\documentclass{beamer}
%\beamertemplateshadingbackground{brown!70}{yellow!10}
\mode<presentation>
{
    %\usetheme{Warsaw}
    \usecolortheme{crane}
    % or ...

    \setbeamercovered{transparent}
    % or whatever (possibly just delete it)
}
\setbeamertemplate{navigation symbols}{}
\setbeamertemplate{footline}[frame number]{}

\usepackage{tikz,pgfplots}
\pgfplotsset{compat=newest}
\usepackage{mathtools}
\usepackage{relsize}
\usepackage[utf8]{inputenc}
\usepackage{amssymb}
\usepackage{amsmath}
\usepackage{colortbl}
%\usepackage{multicol}
\usepackage{cancel}
\usepackage{multirow}
\usepackage{algorithm}
\usepackage{algorithmic}
\usepackage{xfrac}
\usepackage{forloop}% http://ctan.org/pkg/forloop
\AtBeginSection[]
{
\begin{frame}<beamer>
\frametitle{Outline}
\tableofcontents[currentsection]
\end{frame}
}

\def\opt{{\textsc{OPT}_k}}
\def\const{{\mathrm{const}}}
\def\nnz{{\mathrm{nnz}}}
\def\r{\sfrac{\sigma_{\w}^2}{\sigma_{\xib}^2}}
\def\rm{\sfrac{\sigma_{\xib}^2}{\sigma_{\w}^2}}
\def\cmark{\Green{\checkmark}}
\def\xmark{\Red{\large\sffamily x}}
\newcommand{\pdet}{{\mathrm{pdet}}}
\newcommand{\MSPE}[1] {{\mathrm{MSPE}\big[#1\big]}}
\newcommand{\MSE}[1] {{\mathrm{MSE}\big[#1\big]}}
\def\Poisson{{\operatorname{Poisson}}}
\def\PB{{\operatorname{PB}}}
\newcommand{\DP}[1]{\mathcal{DP}^{#1}}
\def\Ic{\mathcal{I}}
\def\Jc{\mathcal{J}}
\def\Mc{\mathcal M}
\def\Ec{\mathcal E}
\def\sr{{\mathrm{sr}}}
\def\ktd{{k^{\underline{d}}}}
\def\Det{{\mathrm{Det}}}
\def\detu{{\widecheck{\mathrm{Det}}_\mu^\gamma}}
\def\deto{{\widehat{\mathrm{Det}}_\mu^\gamma}}
\def\Zu{{\widecheck{Z}_\mu^{\gamma}}}
\def\Zo{{\widehat{Z}_\mu^{\gamma}}}
\def\Zun{{\widecheck{Z}_\mu^{\gamma_n}}}
\def\Zon{{\widehat{Z}_\mu^{\gamma_n}}}
\newcommand{\Er}{\mathrm{Er}}
\newif\ifDRAFT
\DRAFTtrue
\ifDRAFT
\newcommand{\marrow}{\marginpar[\hfill$\longrightarrow$]{$\longleftarrow$}}
\newcommand{\niceremark}[3]
   {\textcolor{red}{\textsc{#1 #2:} \marrow\textsf{#3}}}
\newcommand{\ken}[2][says]{\niceremark{Ken}{#1}{#2}}
\newcommand{\manfred}[2][says]{\niceremark{Manfred}{#1}{#2}}
\newcommand{\michael}[2][says]{\niceremark{Michael}{#1}{#2}}
\newcommand{\michal}[2][says]{\niceremark{Michal}{#1}{#2}}
\newcommand{\feynman}[2][says]{\niceremark{Feynman}{#1}{#2}}
%\usepackage[inline]{showlabels}
\else
\newcommand{\ken}[1]{}
\newcommand{\michael}[1]{}
\newcommand{\michal}[1]{}
\newcommand{\feynman}[1]{}
\fi
\newcommand{\norm}[1]{{\| #1 \|}}

\newcommand{\deff}{d_{\textnormal{eff}}}
\def\ee{\mathrm{e}}
\newcommand\mydots{\makebox[1em][c]{.\hfil.\hfil.}}
\def\Sd{\mathscr{S}_{\!d}}
\newcommand{\dx}{\dxy_{\!\cal X}}
\newcommand{\dxk}{\dxy_{\!\cal X}^k}
\newcommand{\dk}{\dxy^k}
\newcommand{\dxy}{\mathrm{D}}
\def\simiid{\overset{\textnormal{\fontsize{6}{6}\selectfont
i.i.d.}}{\sim}}
%\newcommand{\Dxy}{D_{\!\cal X\!,\cal Y}}
\def\vskx{{\mathrm{VS}_{\!\dx}^k}}
\def\vsk{{\mathrm{VS}_{\!D}^k}}
\def\vskxm{{\mathrm{VS}_{\!\dx}^{k-1}}}
\def\vskm{{\mathrm{VS}_{\!D}^{k-1}}}
\def\vsdx{{\mathrm{VS}_{\!\dx}^d}}
\def\vsd{{\mathrm{VS}_{\!D}^d}}
\newcommand{\vs}[1]{{\mathrm{VS}_{\!D}^{#1}}}
\newcommand{\sigd}{\boldsymbol\Sigma_{\!\dx}}
\def\wols{\w_{\mathrm{LS}}}
\def\wds{\boldsymbol\w_{\!D}^*}
\def\kd{K_{\!\dx}}

\def\poly{{\mathrm{poly}}}
\def\polylog{{\mathrm{polylog}}}
\def\DPP{{\mathrm{DPP}}}
\def\DPPcor{{\DPP_{\!\mathrm{cor}}}}
\def\DPPens{{\DPP_{\!\mathrm{ens}}}}
\newcommand{\DPPreg}[1]{{\DPP_{\!\mathrm{reg}}^{#1}}}
\def\Vol{{\mathrm{VS}}}
\def\Lev{{\mathrm{Lev}}}
\newcommand\todod[1]{\Red{\# DH: #1}}
\newcommand{\explain}[2]{\mathrel{\overset{\makebox[0pt]{\text{\tiny
#1}}}{#2}}}
\def\tot {{\mathrm{tot}}}
\def\checkmark{\tikz\fill[scale=0.4](0,.35) -- (.25,0) --
(1,.7) -- (.25,.15) -- cycle;}
\newcommand{\mnote}[1]{{\bf\large \Magenta{*}}\marginpar{\small \Magenta{#1}}}
\newcommand{\bnote}[1]{{\bf #1}}

\newcommand{\sqrtshort}[1]{{\sqrt{\white{\Big|}\!\!\smash{\text{\fontsize{9}{9}\selectfont$#1$}}}}}
\newenvironment{proofof}[2]{\par\vspace{2mm}\noindent\textbf{Proof of {#1} {#2}}\ }{\hfill\BlackBox}
\newcommand{\sets}[2]
{{\hspace{-0.3mm}[\hspace{-0.3mm}#1\hspace{-0.3mm}]\hspace{-0.3mm}\choose
\hspace{-0.3mm}#2\hspace{-0.3mm}}}
\DeclareMathOperator{\sgn}{\textnormal{sgn}}
\DeclareMathOperator{\adj}{\textnormal{adj}}
\def\Rb{{\mathbf{R}}}
\DeclareMathOperator{\ws}{\widetilde{\w}}
\newcommand{\inote}[1]{{\bf {#1}}}
\def\xib{\boldsymbol\xi}
\def\Sigmab{\mathbf{\Sigma}}
\def\Sigmabh{\widehat{\Sigmab}}
\def\Sigmabt{\widetilde{\Sigmab}}
\def\S{\mathbf{S}}
\def\T{\mathbf{T}}
\def\xt{\tilde{x}}
\def\xbt{\widetilde{\x}}
\def\xbh{\widehat{\x}}
\def\ubh{\widehat{\u}}
\def\dom {{\mathrm{dom}}}
\def\val {{\mathrm{val}}}
\def\out {{\mathrm{out}}}
\def\iin  {{\mathrm{iin}}}
\def\s {\mathbf{s}}
\def\q {\mathbf{q}}
\def\qt{\tilde{q}}
\def\itld {j}
\def\ubt {\tilde{\u}}
\def\n{\{1..n\}}
\def\cb {\mathbf{c}}
\def\cW{\mathcal W}
\def\Xt{\widetilde{X}}
\def\Dbt{\widetilde{\D}}
\def\xtb{\tilde{\mathbf{x}}}
\def\ytb{\tilde{\mathbf{y}}}
\def\Xtb{\widetilde{\mathbf{X}}}
\def\Xbb{\overline{\X}}
\def\Xb{{\bar{\X}}}
\def\ybb{\overline{\y}}
\def\f{{\mathbf{f}}}
\def\g{{\mathbf{g}}}
\def\fbb{{\overline{\f}}}
\def\fb{{\overline{f}}}
\def\Xc{\mathcal{X}}
\def\W{\mathbf W}
\def\L{\mathbf{L}}
\def\Rb{\mathbf R}
\def\Pc{\mathcal{P}}
\def\Nc{\mathcal{N}}
\def\Pt{\widetilde{P}}
\def\Hc{\mathcal{H}}
\def\Wc{\mathcal{W}}
\def\Cc{\mathcal{C}}
\def\p{\mathbf p}
%\def\r{\mathbf r}
\def\Y{\mathbf Y}
\def\H{\mathbf H}
\def\K{\mathbf K}
\def\Kh{\widehat{K}}
\def\Kbh{{\widehat{\K}}}
\def\Q{\mathbf Q}
\def\Qbar{{\bar{\mathbf Q}}}
\def\Ytb{\widetilde{\mathbf{Y}}}
\def\c{{n-d\choose s-d}}
\DeclareMathOperator{\Proj}{Proj}
\newcommand{\Span}{\mathrm{span}}
\newcommand{\ofsubt}[1]{\mbox{\scriptsize \raisebox{0.25pt}{$(#1)$}}}
%\raisebox{0.5pt}{$($}}#1\mbox{\tiny \raisebox{0.5pt}{$)$}}}
\newcommand{\ofsub}[1]{\mbox{\small \raisebox{0.0pt}{$(#1)$}}}
%\newcommand{\ofsubb}[1]{\mbox{\footnotesize \raisebox{0.5pt}{$(#1)$}}}
%\newcommand{\ofsub}[1]{(#1)}
%\newcommand{\ofsub}[1]{\mbox{\tiny$|$\hspace{-0.5pt}\raisebox{-0.5pt}{$#1$}}}
\newcommand{\of}[2]{{#1{\!\ofsub{#2}}}}
\newcommand{\oft}[2]{{#1{\!\ofsubt{#2}}}}
\newcommand{\fof}[2]{{#1({#2})}}
\newcommand{\yof}[2]{{#1{\ofsub{#2}}}}
%\newcommand{\yofb}[2]{{#1{\ofsubb{#2}}}}
\newcommand{\lazy}{FastRegVol}
\newcommand{\volsamp}{RegVol}

\newcommand{\Sm}{{S_{-i}}}
\newcommand{\Sp}{{S_{+i}}}
\ifx\BlackBox\undefined
\newcommand{\BlackBox}{\rule{1.5ex}{1.5ex}}  % end of proof
\fi
%\renewcommand{\dagger}{+}
\DeclareMathOperator*{\argmin}{\mathop{\mathrm{argmin}}}
\DeclareMathOperator*{\argmax}{\mathop{\mathrm{argmax}}}
\DeclareMathOperator*{\diag}{\mathop{\mathrm{diag}}}
\def\x{\mathbf x}
\def\y{\mathbf y}
\def\ybh{\widehat{\mathbf y}}
\def\ybb{\bar{\mathbf y}}
\def\xbb{\bar{\mathbf x}}
\def\yb{{\bar y}}
\def\ybt{\widetilde{\mathbf y}}
\def\yh{\widehat{y}}
\def\yhb{\widehat{\y}}
\def\yt{\widetilde{y}}
\def\z{\mathbf z}
\def\a{\mathbf a}
\def\b{\mathbf b}
\def\w{\mathbf w}
\def\v{\mathbf v}
\def\m{\mathbf m}
\def\wbh{\widehat{\mathbf w}}
\def\wh{\widehat{\mathbf w}}
\def\vbh{\widehat{\mathbf v}}
\def\wbt{\widetilde{\mathbf w}}
\def\e{\mathbf e}
\def\zero{\mathbf 0}
\def\one{\mathbf 1}
\def\u{\mathbf u}
\def\ubbar{\bar{\mathbf u}}
\def\f{\mathbf f}
\def\ellb{\boldsymbol\ell}

\def\X{\mathbf X}
\def\Xs{\widetilde{\X}}
\def\B{\mathbf B}
\def\A{\mathbf A}
\def\C{\mathbf C}
\def\U{\mathbf U}
\def\Ubt{\widetilde{\mathbf U}}
\def\Ubh{\widehat{\mathbf U}}
\def\Ubbar{\bar{\mathbf U}}
\def\F{\mathbf F}
\def\D{\mathbf D}
\def\V{\mathbf V}
\def\M{\mathbf M}
\def\Mh{\widehat{\mathbf M}}
%\def\S{\mathbf S}
\def\Stb{\widetilde{\mathbf{S}}}
\def\Sbh{\widehat{\mathbf{S}}}
\def\St{\widetilde{\S}}
\def\Sh{\widehat{S}}
\def\Sc{\mathcal{S}}
\def\Fc{\mathcal{F}}
\def\Vc{\mathcal{V}}
\def\Bc{\mathcal{B}}
\def\Dc{\mathcal{D}}
\def\Z{\mathbf Z}
\def\Zbh{\widehat{\mathbf Z}}
\def\Zbt{\widetilde{\mathbf Z}}
\def\Abh{\widehat{\mathbf A}}
\def\I{\mathbf I}
\def\Ic{\mathcal I}
\def\II{\mathbf {I \!\,I}}
%\def\II{\boldsymbol {\mathbb I}}
\def\A{\mathbf A}
\def\P{\mathbf P}
\def\Ph{\widehat{\mathbf P}}
\def\cP{\mathcal P}
\def\cR{\mathcal R}
\def\Xt{\widetilde{\mathbf{X}}}
\def\Xh{\widehat{\mathbf{X}}}
\def\Rh{\widehat{R}}
\def\Ot{\widetilde{O}}
\def\At{\widetilde{\A}}


\def\E{\mathbb E}
\def\R{\mathbb R}
\def\N{\mathbb N}
\def\Pr{\mathrm{Pr}}
%\def\C{\mathbb C}
\def\tr{\mathrm{tr}}
\def\Sbar{{\bar{S}}}
\def\cS{{\mathcal{S}}}
\def\Tbar{{\bar{T}}}
\def\Tt{{\widetilde{T}}}
\def\rank{\mathrm{rank}}
\def\Prob{\mathrm{Prob}}
\def\Var{\mathrm{Var}}
\def\Xinv{(\X^\top\X)^{-1}}
\def\XinvS{(\X_S\X_S^\top)^{-1}}
\def\ABinvS{(\A_S\B_S^\top)^{-1}}
\def\ABinv{(\A\B^\top)^{-1}}
\def\xinv{\x_i^\top\Xinv\x_i}
\def\Xinvr{(\lambda\I+\X_{-1}^\top\X_{-1})^{-1}}
\def\pdet{\mathrm{pdet}}
\newcommand{\vol}{\mathrm{vol}}
%\newcommand{\defeq}{:=}
\newcommand{\defeq}{\stackrel{\textit{\tiny{def}}}{=}}
\newcommand{\di}{{[d+1]_{-i}}}
\newcommand{\cov}{\mathrm{cov}}
\let\origtop\top
\renewcommand\top{{\scriptscriptstyle{\origtop}}} % this makes transpose not so big

\definecolor{silver}{cmyk}{0,0,0,0.3}
\definecolor{yellow}{cmyk}{0,0,0.9,0.0}
\definecolor{reddishyellow}{cmyk}{0,0.22,1.0,0.0}
\definecolor{black}{cmyk}{0,0,0.0,1.0}
\definecolor{darkYellow}{cmyk}{0.2,0.4,1.0,0}
\definecolor{orange}{cmyk}{0.0,0.7,0.9,0}
\definecolor{darkSilver}{cmyk}{0,0,0,0.1}
\definecolor{grey}{cmyk}{0,0,0,0.5}
\definecolor{darkgreen}{cmyk}{0.6,0,0.8,0}
\newcommand{\Red}[1]{{\color{red}  {#1}}}
\newcommand{\Purple}[1]{{\color{purple}  {#1}}}
\newcommand{\Magenta}[1]{{\color{magenta}{#1}}}
\newcommand{\Green}[1]{{\color{darkgreen}  {#1}}}
\newcommand{\Blue}[1]{\color{blue}{#1}\color{black}}
\newcommand{\Orange}[1]{\textcolor{orange}{#1}\color{black}}
\newcommand{\Brown}[1]{{\color{brown}{#1}\color{black}}}
\newcommand{\Grey}[1]{{\color{grey}{#1}\color{black}}}
\newcommand{\white}[1]{{\textcolor{white}{#1}}}
\newcommand{\yellow}[1]{{\textcolor{reddishyellow}{#1}}}
\newcommand{\darkYellow}[1]{{\textcolor{darkYellow}{#1}}}
\newcommand{\grey}[1]{{\textcolor{grey}{#1}}}

\DeclareMathOperator{\half}{\frac{1}{2}}

\ifx\proof\undefined
\newenvironment{proof}{\par\noindent{\bf Proof\ }}{\hfill\BlackBox\\[2mm]}
\fi

\ifx\theorem\undefined
\newtheorem{theorem}{Theorem}
\fi

\ifx\example\undefined
\newtheorem{example}{Example}
\fi

\ifx\condition\undefined
\newtheorem{condition}{Condition}
\fi
\ifx\property\undefined
\newtheorem{property}{Property}
\fi

\ifx\lemma\undefined
\newtheorem{lemma}{Lemma}
\fi

\ifx\proposition\undefined
\newtheorem{proposition}{Proposition}
\fi

\ifx\remark\undefined
\newtheorem{remark}{Remark}
\fi

\ifx\corollary\undefined
\newtheorem{corollary}{Corollary}
\fi

\ifx\definition\undefined
\newtheorem{definition}{Definition}
\fi

\ifx\conjecture\undefined
\newtheorem{conjecture}{Conjecture}
\fi

\ifx\axiom\undefined
\newtheorem{axiom}{Axiom}
\fi

\ifx\claim\undefined
\newtheorem{claim}{Claim}
\fi

\ifx\assumption\undefined
\newtheorem{assumption}{Assumption}
\fi

\ifx\condition\undefined
\newtheorem{condition}{Condition}
\fi

\renewcommand{\Xinv}{(\X^\top\X)^{-1}}
\renewcommand{\Xinvr}{(\lambda\I+\X_{-1}^\top\X_{-1})^{-1}}

\title[]{Minimax experimental design}

\author[]{Micha{\l } Derezi\'{n}ski}

\begin{document}

\begin{frame}
    \titlepage
\end{frame}

\section{Overview}

\begin{frame}
  \frametitle{Classical statistical regression}
  We consider $n$ parameterized experiments:
  $\x_1,\dots,\x_n\in\R^d$.\\
  Each experiment has a real random outcome $Y_i$ for $i=1..n$.\\[2mm]
  \textbf{Classical setup:}\\
$Y_i=\x_i^\top\w^*+\xi_i$,\quad
  $\E[\xi_i]=0$,\quad $\Var[\xi_i]=\sigma^2$,\quad $\cov[\xi_i,\xi_j]=0$,
\  $i\neq j$\\[3mm]
The \textit{ordinary least squares} estimator $\wols=\X^+ Y$ satisfies:
  \begin{align*}
    \text{\small(unbiasedness)}  && \quad \E[\wols]
  \  &=\ \w^*,\\[2mm]
    \text{\small(mean squared error)} &&\quad
\overbrace{\E\,\|\wols-\w^*\|^2}^{\text{MSE}(\wols)}
                                         \   &=\ \sigma^2\tr\big((\X^\top\X)^{-1}\big)\\
    \text{letting }b=\tr\big((\X^\top\X)^{-1}\big)&&&=\frac bn\cdot \E\,\|\xib\|^2\\
\hspace{-8mm}    \text{\small(mean squared prediction error)}
&&\quad \overbrace{\E\,\|\X(\wols-\w^*)\|^2}^{\text{MSPE}(\wols)}
   \  &=\ \sigma^2d \\
    &&&=\frac dn \cdot\E\,\|\xib\|^2
\end{align*}
%   Suppose we have a budget of $k$ experiments. \textit{Goal:}\\
% Construct a sequence of experiments $\pi\in[n]^k$ and
%   estimator $\wbh_\pi$ s.t.
%   \begin{align*}
% \text{(unbiasedness)},\quad    \E[\wbh_\pi] &= \w^*,\quad\text{and}\\
% \text{(mean squared error)},\quad
%     \sum_{i=1}^d\Var\big[(\wbh_\pi)_i\big]&\text{ is small} \\
%                                             &\text{ or }\\
% \text{(m. sq. prediction e.)},\quad
%     \sum_{i=1}^n\Var\big[\x_i^\top\wbh_\pi\big]&\text{ is small} 
%   \end{align*}
  % \textbf{Our general setup:}\\
  % No assumptions on $Y_i$. We define $\w^* \defeq
  % \E\big[\wbh\big]=\X^+\E[Y]$\\
  % We consider the same problem with budget $k$\\
  % (except we allow $\pi$ to be random to avoid adversarial scenarios)
\end{frame}

\begin{frame}
  \frametitle{Experimental design in classical setting}
  Suppose we have a budget of $k$ experiments out of the $n$
  choices.\\
  \textit{Goal:} Select a subset of $k$ experiments
  $S\subseteq [n]$\\
  \textbf{Question:} How large does $k$ need to be so that:
  \[\overbrace{\text{Excess estimation error}}^{\text{MSE or MSPE}}
    \ \leq\  \epsilon\,\cdot\,
    \overbrace{\text{Total noise}}^{\E\,\|\xib\|^2}\quad ?\]
 Denote $L^*=\E\,\|\xib\|^2 = n\sigma^2$.\\[3mm]
 \textbf{Prior result:} \\
   There is a design $(S,\wbh)$ of size $k$ s.t.~$\E[\wbh_S]=\w^*$ and:
  \begin{align*}
\text{MSE}(\wbh_S) - \text{MSE}(\wols)
    \ &\leq\ \epsilon\cdot
      L^*,\quad\text{for }k\geq  d+ b/\epsilon,\\
\text{MSPE}(\wbh_S)-\text{MSPE}(\wols)
\ &\leq\ \epsilon\cdot L^*,\quad\text{for }k\geq  d + d/\epsilon,
  \end{align*}
  where $b = \tr((\X^\top\X)^{-1})$.
\end{frame}

\begin{frame}
  \frametitle{Experimental design in general setting}
  %   We consider $n$ parameterized experiments:
  % $\x_1,\dots,\x_n\in\R^d$.\\
  % Each experiment has a real random outcome $Y_i$ for
  % $i=1..n$.\\[2mm]
  
No assumptions on $Y_i$. We define
  $\w^* \defeq\E[\wols]=\X^+\E[Y]$\\
Define ``total noise'' as $L^* \defeq \E\,\|\xib\|^2$, where $\xib\defeq
  \X^\top\w^*\!-\!Y$.\\[6mm] 

  \textbf{Theorem 1 (MSE).}\\
  There is a random design $(S,\wbh)$ such that $\E[\wbh_S]=\w^*$ and
\begin{align*}
\text{MSE}(\wbh_S) - \text{MSE}(\wols)
    \ &\leq\ \epsilon\cdot
      L^*,\quad\text{for }k=O(d\log n + b/\epsilon),
\end{align*}
where $b=\tr((\X^\top\X)^{-1})$.
\vspace{6mm}

  \textbf{Theorem 2 (MSPE).}\\
  There is a random design $(S,\wbh)$ such that $\E[\wbh_S]=\w^*$ and
\begin{align*}
\text{MSPE}(\wbh_S) - \text{MSPE}(\wols)
    \ &\leq\ \epsilon\cdot
  L^*,\quad\text{for }k=O(d\log n + d/\epsilon).
\end{align*}
\end{frame}

\begin{frame}
  \frametitle{Examples}
  1. Statistical setting (homoscedastic noise):\\
  $\E[\xi_i]=0$,\quad$\Var[\xi_i]=\sigma^2$,\qquad
$L^* = n\sigma^2$\\[6mm]
  2. Statistical setting (any mean-zero noise):\\
  $\E[\xi_i]=0$,\quad$\Var[\xi_i] = \sigma_i^2$,\qquad
  $L^*=\sum_i\sigma_i^2$ \\[6mm]
  3. Computer science setting (fixed residuals):\\
$\E[\xi_i]=r_i$,\quad \!$\Var[\xi_i] = 0$,\qquad
\ $L^*=\sum_ir_i^2$\\[6mm]
4. Anything in between...
\end{frame}

\begin{frame}
  \frametitle{Minimax experimental design}
  Given matrix $\X\in\R^{n\times d}$ and budget $k$ we define: 
  \begin{align*}
R_k^*(\X)\quad =\quad \min_{S,\wbh}\quad\overbrace{\max_{Y}\quad
    \frac{\text{MSE}(\wbh_S)-\text{MSE}(\wols)}
    {  \E\big[\|\X\w^*- Y\|^2\big]}}^{R_k(\wbh_S\,|\,\X)}
  \end{align*}
  \vspace{3mm}
  
  \textbf{Conjecture 1.} For any estimator $\wbh_S$, there is an
 estimator $\wbt_S$ s.t.~$R_k(\wbt_S\,|\,\X)\leq
 R_k(\wbh_S\,|\,\X)$ and $\E[\wbt_S]=\w^*$ for all $Y$.\\[5mm] 
  \textbf{Conjecture 2.} Maximum over all
  variables $Y$ is always equal to maximum over all constant vectors $\y$
\end{frame}

\begin{frame}
  \frametitle{Tractable minimax design}
    Given matrix $\X\in\R^{n\times d}$ and budget $k$ we
    define:
    \begin{align*}
      \widehat{R}_k(\X)\quad =\quad \min_{q}\ R_k(\w_\pi^*\,|\,\X),
    \end{align*}
    where $\w_\pi^*$ is obtained by $q$-rescaled volume sampling.\\[5mm]
    Advantages:
    \begin{enumerate}
    \item Efficient sampling,
    \item Always unbiased,
    \item $R_k(\w_\pi^*\,|\,\X)$ can be easily estimated.
      % Defining $P_\X\defeq\X(\X^\top\X)^{-1}\X^\top$,
      % \begin{align*}
      %   R_k(\w_\pi^*\,|\,\X) =
      %   \big\|\E\big[\Q_\pi^{\sfrac12}(\Q_\pi^{\sfrac12}\P_\X
      %   \Q_\pi^{\sfrac12})^+\Q_\pi^{\sfrac12}\big]-\P_\X\big\|
      % \end{align*}
    \end{enumerate}
    \vspace{3mm}
    
    \textbf{Question:} Is there an efficient iterative
   optimizer for finding $q^*$?
\end{frame}

\section{Averaging unbiased estimators}


% \begin{frame}
% \frametitle{Subsampling for linear regression}
% \only<1->
% {
% {\bf Given}: $n$ points $\x_i\in\R^d$ with hidden labels $y_i\in \R$\\
% {\bf Goal}: Minimize loss $L(\w)= \sum_i (\x_i^\top\w-y_i)^2$ over
% all $n$ points
% }
% \only<1>
% {
% \begin{columns}
% \begin{column}{0.3\textwidth}
% \end{column}
% \begin{column}{0.5\textwidth}
% \begin{center}
% 	\begin{tikzpicture}[scale=0.9]
%           \draw [fill=brown!30] (-2,0) rectangle (0,3);
%           \draw [decorate,decoration={brace}] (-2.1,0) -- (-2.1,3);
%           \draw (-2.4,1.51) node {\mbox{\fontsize{8}{8}\selectfont $n$}}; 
%           \draw [decorate,decoration={brace}] (-2,3.1) -- (0,3.1);
%           \draw (-1,3.4) node {\mbox{\fontsize{8}{8}\selectfont $d$}}; 

% %          \draw [color=black] (-2,2) -- (0,2);
% %          \draw (-2.25,2) node {\mbox{\footnotesize $\x_4^\top$}}; 
% %          \draw [color=black] (-2,1.5) -- (0,1.5);
% %          \draw (-2.25,1.5) node {\mbox{\footnotesize $\x_6^\top$}}; 
% %          \draw [color=black] (-2,0.5) -- (0,0.5);
% %          \draw (-2.25,0.5) node {\mbox{\footnotesize $\x_9^\top$}}; 
% 	    \draw (-2.5,3) node {$\X$}; 
%             \draw [color=lightgray,line width =0.5mm] (1,0) -- (1,3);
%             \draw [color=lightgray] (0.75,3) node {$\y$};
% %            \draw (0.75,2) node {\mbox{\footnotesize $y_4$}}; 
% %            \draw (0.75,1.5) node {\mbox{\footnotesize $y_6$}}; 
% %            \draw (0.75,0.5) node {\mbox{\footnotesize $y_9$}}; 
% 	\end{tikzpicture}
% \end{center}
% \end{column}
% \end{columns}
% }
% \only<2->
% {
% \begin{columns}
% \begin{column}{0.3\textwidth}
% \\\vspace{0.8cm}
% Sample $S=\{4,6,9\}$\\
% \vspace{1cm}
% Receive $y_4, y_6, y_9$
% \end{column}
% \begin{column}{0.5\textwidth}
% \begin{center}
% 	\begin{tikzpicture}[scale=0.9]
%           \draw [fill=brown!30] (-2,0) rectangle (0,3);
%           \draw [color=black] (-2,2) -- (0,2);
%           \draw (-2.25,2) node {\mbox{\footnotesize $\x_4^\top$}}; 
%           \draw [color=black] (-2,1.5) -- (0,1.5);
%           \draw (-2.25,1.5) node {\mbox{\footnotesize $\x_6^\top$}}; 
%           \draw [color=black] (-2,0.5) -- (0,0.5);
%           \draw (-2.25,0.5) node {\mbox{\footnotesize $\x_9^\top$}}; 
% 	   \draw (-2.5,3) node {$\X$}; 
%            \draw [decorate,decoration={brace}] (-2,3.1) -- (0,3.1);
%           \draw (-1,3.4) node {\mbox{\fontsize{8}{8}\selectfont $d$}}; 
%             \draw [color=lightgray,line width =0.5mm] (1,0) -- (1,3);
%             \draw [color=lightgray] (0.75,3) node {$\y$};
%             \draw (0.75,2) node {\mbox{\footnotesize $y_4$}}; 
%             \draw (1,2) node {.}; 
%             \draw[mark=*,mark size=1.5pt] plot coordinates{(1,2)};
%             \draw (0.75,1.5) node {\mbox{\footnotesize $y_6$}}; 
%             \draw (1,1.5) node {.}; 
%             \draw[mark=*,mark size=1.5pt] plot coordinates{(1,1.5)};
%             \draw (0.75,0.5) node {\mbox{\footnotesize $y_9$}}; 
%             \draw[mark=*,mark size=1.5pt] plot coordinates{(1,.5)};
% 	\end{tikzpicture}
% \end{center}
% \end{column}
% \end{columns}
% }
% \vspace{1cm}
% \pause
% \pause
% {\bf Goal}: Design the sampling of $S$ and estimator $\wbh(S)$ such that:
% \begin{align*}
% \E\Big[L\big(\wbh(S)\big)\Big] \leq (1+\epsilon)
%   L(\w^*),\quad\text{where  }\w^* = \argmin_\w L(\w).
% \end{align*}
% \end{frame}

\begin{frame}
  \frametitle{Distributed sampling for regression}
  Central server has the matrix $\X$ but no labels.\\

  There is $m$ learners, each can query a small number of labels.\\[3mm]
  % The learners are not allowed to share the labels with anyone.\\[3mm]

  We consider the following distributed learning procedure:
  \begin{enumerate}
    \item For each $t$, server sends learner $t$ a small sample of
      input points along with some side information,
    \item Learner $t$ queries the labels and computes estimator $\wbh_t$
      \item Server collects all estimators $\wbh_t$ and
        produces:
  \end{enumerate}
\begin{align*}
\wbh = \frac{1}{m}\sum_{t=1}^m\wbh_t
\end{align*}
\textbf{Goal:} Minimizing computation,
communication and number of queries per learner, obtain $\wbh$ such that
\begin{align*}
\E\big[ L(\wbh)\big] \leq \Big(1+\frac1m\Big)\,L(\w^*).
\end{align*}
\end{frame}

\begin{frame}
  \frametitle{Applying volume sampling in the distributed setting}
  We generate $m$ independent volume sampled estimators $\wbh_t$.\\
Volume sampling implies $\E[\wbh_t] = \w^*$, so for $\wbh =
  \frac1m\sum_t\wbh_t$ we have
  \begin{align*}
    \E\big[L(\wbh)\big] -L(\w^*) = \frac1m\Big(\E\big[L(\wbh_t)\big]
    - L(\w^*)\Big),
  \end{align*}
  for any fixed $t$.\\[2mm]
  So if $\E[L(\wbh_t)]\leq 2\,L(\w^*)$, then $\E[L(\wbh)]\leq
  (1+\frac1m)\,L(\w^*)$\\[5mm]
  
  \textbf{Note:} Tail bounds do not imply expected bounds!\\
  \textbf{Example:} For an i.i.d.~leverage score sample of size
  $O(d\log d+d/\epsilon)$, 
  \begin{align*}
L(\wbh_t)\leq (1+\epsilon) L(\w^*)\text{ w.p. 0.9,\quad but}\quad
    \frac{\E\big[L(\wbh_t)\big]}{L(\w^*)} \text{ cannot be bounded}
  \end{align*}
  (a simple lower bound can be shown here)
\end{frame}

\begin{frame}
  \frametitle{Comparison to prior results}
  For volume sampling it was shown by [DW17] that
  \begin{align*}
    \E\big[L(\wbh_t)\big]\leq (d+1)\,L(\w^*)
  \end{align*}
So we need $d$ independent volume samples (of size $d$) per machine to
get estimator $\wbh_t$ such that $\E\big[L(\wbh_t)\big]\leq
2\,L(\w^*)$.\\[3mm] 
To obtain an averaged estimator s.t. $\E[L(\wbh)]\leq
(1+\frac1m)L(\w^*)$, the cost per machine is:
\begin{enumerate}
\item $d^2$ labels
\item $ O(d^5)$ runtime
\end{enumerate}
\vspace{5mm}
\hrule
\vspace{6mm}

The cost per machine obtained in \textbf{our new results} is:
\begin{enumerate}
\item $O(d\log n)$ labels \qquad (new expected loss bound)
  \item $O(d^3\log n)$ runtime \qquad (new sampling algorithm)
\end{enumerate}
\end{frame}


\section{Rescaled volume sampling [DWH18a]}

\begin{frame}
\frametitle{Rescaled volume sampling [DWH18a]}
Let $\X\in\R^{n\times d}$ be any matrix and $q = (q_1...q_n)$ be any distribution\\
For any $\pi\in[n]^k$, define matrix $\Q_\pi\in\R^{n\times n}$ s.t.
\vspace{-2mm}
\[
\Q_\pi=\sum_{i=1}^k\frac{1}{q_{\pi_i}}\e_{\pi_i}\e_{\pi_i}^\top=\!\begin{pmatrix}
\sfrac{s_{\pi\!,1}}{q_1}\!\!&&0\\
&\ddots&\\
0&&\!\!\sfrac{s_{\pi\!,n}}{q_n}
\end{pmatrix}\quad\text{where}\quad s_{\pi\!,j}=|\{i:\pi_i=j\}|.
\]
% \[\Xt = \begin{pmatrix}\frac{1}{\sqrt{q_{1}}}\x_{1}^\top\\
% \cdots\\
% \frac{1}{\sqrt{q_{n}}}\x_{n}^\top
% \end{pmatrix}.\]
We define $q$-rescaled size $k$ volume sampling of $\X$:
\begin{align*}
\text{for $\pi\in[n]^k$,}\quad
P(\pi) \propto \det\big(
%\overbrace{\sum \frac{1}{q_{\pi_i}}\x_{\pi_i}\x_{\pi_i}^\top}^{
\X^\top\!\Q_\pi\X\big) \prod_{i=1}^kq_{\pi_i}
\end{align*}
% \begin{align*}
% P(\pi) \propto \det\big(
% %\overbrace{\sum \frac{1}{q_{\pi_i}}\x_{\pi_i}\x_{\pi_i}^\top}^{
% \Xt_{\pi}^\top\Xt_{\pi}\big) \prod_{i=1}^kq_{\pi_i}
% \end{align*}
% \textbf{Simple implementation using rejection sampling}:\\
% 1. Sample $\hat{\pi}_1...\hat{\pi}_k\in[n]$ i.i.d. from $q$,\\
% 2. w.p. $\det(\X^\top\!\Q_{\hat{\pi}}\X)\,/ D$, set $\pi=\hat{\pi}$ (otherwise go to step 1), where
% \[D \geq \max_{\tilde{\pi}\in[n]^k}\det(\X^\top\!\Q_{\tilde{\pi}}\X)\]
\end{frame}

\begin{frame}
\frametitle{Square loss under rescaled volume sampling}
We define the least squares solution after sampling $\pi\in[n]^k$:
\begin{align*}
\w_\pi^* &\defeq \argmin_\w L_\pi(\w)\\
\text{where }
L_\pi(\w)&\defeq\|\Q_\pi^{\sfrac{1}{2}}(\X\w - \y)\|^2=\sum_{i=1}^ns_i\,\frac{1}{q_i}(\x_i^\top\w-\y_i)^2
\end{align*}
Then, the least squares solution can be computed as:
\begin{align*}
\w_\pi^* = (\Q_\pi^{\sfrac{1}{2}}\X)^+\Q_\pi^{\sfrac{1}{2}}\y
\end{align*}
\end{frame}


\begin{frame}
\frametitle{Prior results [DWH18a]}
Unbiasedness for any $q$ with full support:
\begin{align*}
\E[\w_\pi^*] = \w^*
\end{align*}

\textbf{Leveraged volume sampling:}\\
Use $q = (\frac{\ell_1}{d},...,\frac{\ell_n}{d})$, where $\ell_i=\x_i^\top(\X^\top\X)^{-1}\x_i$\\[4mm]
\textit{Tail bounds for leveraged volume sampling:}
\begin{align*}
L(\w_\pi^*)\leq (1+\epsilon)\,L(\w^*)\quad\text{ w.p.~$0.9$ for }k\geq C\Big(d\log d + \frac{d}{\epsilon}\Big)
\end{align*}
\textit{Cost:} $O(d^4)$ per sample, \\
plus computing leverage scores, $O(nd\log n + d^4\log d)$\\[3mm]
\end{frame}


\section{New expected loss bound}

\begin{frame}
  \frametitle{From random noise to fixed noise}
  Recall that $\wbh=\X^+Y$ and $\w^*=\X^+\E[Y]$. Let
  $\P\defeq\X\X^+$.\\
  Note that if $\E[\w_\pi^*|Y]=\wbh$, then $\E[\w_\pi^*] = \w^*$. Also,
  \begin{align*}
    \E\big[\|\X\wbh-Y\|^2\big]
    &= \E\big[\|\X(\wbh-\w^*)\|^2\big] +
    \E\big[\|\X\w^*-Y\|^2\big] \\
&= \E\big[\|\P(Y-\E[Y])\|^2\big] + \sum_i\E[\xi_i^2]\\
    &\leq \E\big[\|Y-\E[Y]\|^2\big]+ L^*\ \leq\ 2\,L^*.
  \end{align*}
  Note that if $\Var[Y]=\zero$ then $\wbh=\w^*$ and $\|\X\wbh-Y\|^2=L^*$.\\
  So, w.l.o.g.~we can assume that noise is fixed.
\end{frame}

\begin{frame}
  \frametitle{Main result}
  Let $q =(q_1..q_n)$ be a distribution s.t. $q_i\geq
  \max\{\frac1{2n},\frac{\ell_i}{2d}\}$ for all $i$.\\ [5mm]
  \textbf{Theorem.} \ 
For a $q$-rescaled volume sample of size $k=O(d \log n)$:
  \begin{align*}
    \E\big[L(\w_\pi^*)\big] \leq 2 L(\w^*).
  \end{align*}
  \vspace{8mm}
  
  \textbf{Preliminaries:} Let $\U = \X(\X^\top\X)^{-\frac12}$. Note
  that
  \begin{align*}
    L(\w_\pi^*) - L(\w^*) = \|(\U^\top\Q_\pi\U)^{-1}\U^\top\Q_\pi\r\|^2,
  \end{align*}
  where $\r=\X\w^*-\y$ and $\|\r\|^2=L(\w^*)$.
\end{frame}

\begin{frame}
  \frametitle{Bounding the expected loss}
We use \underline{closure under augmentation} with i.i.d. samples [DWH18b]:
\begin{align*}
  \pi = [\underbrace{\pi_1,\ \dots,\ \pi_d}_{\text{volume
  sampling}}\ ,\ \underbrace{\pi_{d+1},\ \dots,\ \pi_k}_{\text{i.i.d.~samples}}].
\end{align*}
Let $A$ be the event that
$\|(\frac1s\U^\top\Q_{\pi_{s:k}}\U)^{-1}\|\leq 2$ for $s=k/2$.
\begin{align*}
\E\big[L(\w_\pi^*)\big] = \Pr(A)\,\E\big[L(\w_\pi^*)\,|\,A\big] +
\Pr(\neg A)\,\E\big[L(\w_\pi^*)\,|\,\neg A\big].
\end{align*}
For $k=\Theta(d\log \frac{d}{\delta})$ we have $\Pr(\neg A)\leq
\delta$ and:  
\begin{align*}
 \Pr(A)\E\big[L(\w_\pi^*)\,|\,A\big]&= \frac32 L(\w^*),\\
  \E\big[L(\w_\pi^*)\,|\,\neg A\big]& = Cnd\, L(\w^*).
\end{align*}
For $\delta=\frac1{2Cnd}$, we get
$\E\big[L(\w_\pi^*)\big] \leq 2 L(\w^*)$ for $k=\Theta(d\log n)$.
\end{frame}

\begin{frame}
  \frametitle{Bounding expected loss when A succeeds}
Recall that for $\r=\X\w^*-\y$, we have $\|\r\|^2=L(\w^*)$ and
$\E\big[\|\frac1k\U^\top\Q_\pi\r\|^2\big] \leq C\frac{d}{k}\|\r\|^2$
(Theorem 8 [DWH18a]). If $A$, then
\begin{center} $\|(\frac1k\U^\top\Q_\pi\U)^{-1}\|\leq
  2\,\|(\frac1s\U^\top\Q_{\pi_{s:k}}\U)^{-1}\|\leq 4$,\quad so\end{center} 
  \begin{align*}
    \hspace{-5mm}
    \Pr(A)\,\Big(\E\big[L(\w_\pi^*)\,|\,A\big] - L(\w^*)\Big)
    &=
      \Pr(A)\,
      \E\Big[\big\|(\frac1k\U^\top\Q_\pi\U)^{-1}
      \frac1k\U^\top\Q_\pi\r\big\|^2\,|\,A\Big]
    \\
    &\leq 16\ \Pr(A)\,\E\Big[\|\frac1k\U^\top\Q_\pi\r\|^2\,|\,A\Big],
  \end{align*}
Now, we just have to get rid of the conditioning on $A$:
  \begin{align*}
    \hspace{-0.7cm}\Pr(A)\, \E\Big[\|\frac1k\U^\top\Q_\pi\r\|^2\,|\,A\Big]
    &=
    \E\Big[\|\frac1k\U^\top\Q_\pi\r\|^2\Big] - \Pr(\neg A)
    \E\Big[\|\frac1k\U^\top\Q_\pi\r\|^2\,|\,\neg A\Big]\\
    &\leq \E\Big[\|\frac1k\U^\top\Q_\pi\r\|^2\Big] \leq  \frac1{32}L(\w^*),
  \end{align*}
  for $k\geq 32Cd$, so we get $\Pr(A)\,\E[L(\w_\pi^*)\,|\,A]\leq
  \frac32 L(\w^*)$.
\end{frame}

\begin{frame}
  \frametitle{Bounding expected loss when A fails}
  For this we use that $q_i\geq
  \max\{\frac1{2n},\frac{\ell_i}{2d}\}$, so that $\Q_\pi\preceq 2nk\,\I$.
  \begin{align*}
    L(\w_\pi^*)& - L(\w^*)
= \|(\U^\top\Q_\pi\U)^{-1}\U^\top\Q_\pi\r\|^2 = \r^\top\Q_\pi\U (\U^\top\Q_\pi\U)^{-2}\U^\top\Q_\pi\r\\
&\leq \|\r\|^2 \big\|\Q_\pi\U (\U^\top\Q_\pi\U)^{-2}\U^\top\Q_\pi\big\|\\
&\leq L(\w^*)\tr\big(\Q_\pi\U
(\U^\top\Q_\pi\U)^{-2}\U^\top\Q_\pi\big)\\
&= L(\w^*)
\tr\big(\U^\top\Q_\pi^2\U(\U^\top\Q_\pi\U)^{-2}\big)\\
&\leq L(\w^*)2nk\, \tr\big(
\U^\top\Q_\pi\U(\U^\top\Q_\pi\U)^{-2}\big)\\
&=2nk\,L(\w^*)\,\tr\big((\U^\top\Q_\pi\U)^{-1}\big)\leq 2nk\,L(\w^*)\,\tr\big((\U^\top\Q_{\pi_{1:s-1}}\U)^{-1}\big).
  \end{align*}
Note that $\pi_{1:s-1}$ is distributed according to leveraged volume
sampling (up to permutation), and it is independent of event $A$, so:
\begin{align*}
  \E\Big[\tr\big((\U^\top\Q_{\pi_{1:s-1}}\U)^{-1}\big)\,|\,\neg A\Big]
  &=\E\Big[\tr\big((\U^\top\Q_{\pi_{1:s-1}}\U)^{-1}\big)\Big]\\
\big(\text{Theorem 4 [DWH18a]}\big)\quad  &\leq \frac{\tr\big((\U^\top\U)^{-1}\big)}{s-d}= \frac{2d}{k-2d}
\end{align*}
So we get $\E\big[L(\w_\pi^*)\,|\,\neg A\big]= O\big(nd\,L(\w^*)\big)$.
\end{frame}

\section{New Variance bound}

\begin{frame}
  \frametitle{Bounding the variance}
We use \underline{closure under augmentation} with i.i.d. samples [DWH18b]:
\begin{align*}
  \pi = [\underbrace{\pi_1,\ \dots,\ \pi_d}_{\text{volume
  sampling}}\ ,\ \underbrace{\pi_{d+1},\ \dots,\ \pi_k}_{\text{i.i.d.~samples}}].
\end{align*}
Let $A$ be the event that
$\|(\frac1s\X^\top\Q_{\pi_{s:k}}\X)^{-1}\X^\top\X\|\leq 2$ for $s=k/2$.
\begin{align*}
  \E\big[\|\w_\pi^*-\w^*\|^2\big]
  &= \Pr(A)\,\E\big[\|\w_\pi^*-\w^*\|^2\,|\,A\big] \\
  &\quad + \Pr(\neg A)\,\E\big[\|\w_\pi^*-\w^*\|^2\,|\,\neg A\big].
\end{align*}
% For $k=\Theta(d\log \frac{d}{\delta})$ we have $\Pr(\neg A)\leq
% \delta$ and:  
% \begin{align*}
%  \Pr(A)\E\big[L(\w_\pi^*)\,|\,A\big]&= \frac32 L(\w^*),\\
%   \E\big[L(\w_\pi^*)\,|\,\neg A\big]& = Cnd\, L(\w^*).
% \end{align*}
% For $\delta=\frac1{2Cnd}$, we get
% $\E\big[L(\w_\pi^*)\big] \leq 2 L(\w^*)$ for $k=\Theta(d\log n)$.
\end{frame}

\begin{frame}
  \frametitle{If A fails}
  Assume that $q_i\geq \frac1{2n}$, so that $\Q_\pi\preceq 2nk\I$.
  \begin{align*}
    \|\w_\pi^*-\w^*\|^2
    &= \big\|(\X^\top\Q_\pi\X)^{-1}\X^\top\Q_\pi(\y -
      \X\w^*)\big\|^2\\
    &\leq
      \|\xib\|^2\,\|\Q_\pi\X(\X^\top\Q_\pi\X)^{-2}\X^\top\Q_\pi\|\\
    &\leq
      \|\xib\|^2\tr\big(\X^\top\Q_\pi^2\X(\X^\top\Q_\pi\X)^{-2}\big)\\
    &\leq 2nk  \|\xib\|^2\tr\big((\X^\top\Q_\pi\X)^{-1}\big)
      \leq 2nk  \|\xib\|^2\tr\big((\X^\top\Q_{\pi_{1:s-1}}\X)^{-1}\big)
  \end{align*}
  Note that $\pi_{1:s-1}$ is distributed according to leveraged volume
sampling (up to permutation), and it is independent of event $A$, so:
\begin{align*}
  \E\Big[\tr\big((\X^\top\Q_{\pi_{1:s-1}}\X)^{-1}\big)\,|\,\neg A\Big]
  &=\E\Big[\tr\big((\X^\top\Q_{\pi_{1:s-1}}\X)^{-1}\big)\Big]\\
\big(\text{Theorem 4 [DWH18a]}\big)\quad  &\leq \frac{\tr\big((\X^\top\X)^{-1}\big)}{s-d}= \frac{2b}{k-2d}
\end{align*}
So we get $\E\big[\|\w_\pi^*-\w^*\|^2|\,\neg A\big]\leq Cnb\,\|\xib\|^2$.
\end{frame}

\begin{frame}
  \frametitle{If A succeeds}
  \begin{align*}
\hspace{-6mm}    \E\big[\|\w_\pi^*-\w^*\|\,|\,A\big]
    &= \E\big[\big\|(\X^\top\Q_\pi\X)^{-1}\X^\top\Q_\pi(\y - \X\w^*)\big\|\,|\,A\big]\\
    &\leq
      \E\Big[\Big\|\Big(\frac1k\X^\top\Q_\pi\X\Big)^{-1}\X^\top\X\Big\|\cdot
      \Big\|\frac1k(\X^\top\X)^{-1}\X^\top\Q_\pi
      \xib\Big\|\,|\,A\Big]\\
    &\leq 2\ \E\Big[\Big\|\frac1k(\X^\top\X)^{-1}\X^\top\Q_\pi
      \xib\Big\|\,|\,A\Big]
  \end{align*}
  So, we have:
  \begin{align*}
    \Pr(A)\E\big[\|\w_\pi^*-\w^*\|\,|\,A\big]
    &\leq 2\ \Pr(A)\ \E\Big[\Big\|\frac1k(\X^\top\X)^{-1}\X^\top\Q_\pi
    \xib\Big\|\,|\,A\Big]\\
    &\leq2\, \E\Big[\Big\|\frac1k(\X^\top\X)^{-1}\X^\top\Q_\pi
    \xib\Big\|^2\Big]
    \end{align*}
  So again we need \textit{matrix multiplication}.
  \end{frame}

\begin{frame}
  \frametitle{Sampling with pseudo-inverse square norms}
  Consider a distribution $q=(q_1,\dots,q_n)$ such that
  \begin{align*}
    q_i \geq \frac{\|(\X^+)_{:,i}\|^2}{2\,\|\X^+\|_F^2} =
    \frac{\x_i^\top(\X^\top\X)^{-2}\x_i}{2\,\tr\big((\X^\top\X)^{-1}\big)}.
  \end{align*}
  From standard matrix multiplication, if $\X^\top\r = \zero$, then:
  \begin{align*}
    \E\  \Big\|\frac1k\X^+\Q_\pi \r\Big\|^2 \leq
    C\,\frac1k \|\X^+\|_F^2\,\|\r\|^2.
  \end{align*}
  Note that $\|\X^+\|_F^2=\tr\big((\X^\top\X)^{-1}\big)=b$. \\
  So for $k=O(b/\epsilon)$ we have $\E\|\frac1k\X^+\Q_\pi\r\|^2\leq
  \epsilon\cdot L^*$.\\[5mm]
  \textbf{Question:} Can we get subspace embedding with this sampling?
\end{frame}

\begin{frame}
  \frametitle{Chernoff bound for pseudo-inverse square norms}
Let $q$ be as before and $\U=\X(\X^\top\X)^{-\frac12}$.  Observe that:
  \begin{align*}
\|\u_i\|^2=\x_i^\top(\X^\top\X)^{-1}\x_i \leq
    \lambda_{\max}(\X^\top\X)\cdot \x_i^\top(\X^\top\X)^{-2}\x_i. 
  \end{align*}
  Let $\Z_i = \frac1{q_i}\u_i\u_i^\top$.
  Sample $\pi_1,\dots,\pi_k\sim q$.\\[1mm]
  Matrix Chernoff for matrices
  $\|\Z_{\pi_i}\|\leq \lambda_{\max}(\X^\top\X)\cdot b$ says we
  need
  \[k=O\big( \lambda_{\max}(\X^\top\X)\cdot b\log
    (d/\delta)\big)=O\big(\kappa d\log(d/\delta)\big),\]
  and for this $k$ we get $Pr(\neg A)\leq \delta$.
  \end{frame}

\begin{frame}
  \frametitle{Sampling with a mixture}
  Consider a distribution $q=(q_1,\dots,q_n)$ such that
  \begin{align*}
    q_i = \frac{\beta\ell_i+ \| (\X^+)_{:,i}\|^2}{\beta d +b}.
  \end{align*}
where $\ell_i=\x_i^\top(\X^\top\X)^{-1}\x_i$. Let $\z_i =
\frac{r_i}{q_i}(\X^+)_{:,i}$. If $\X^\top\r = \zero$, then
$\E[\z_{\pi_1}] = \X^+\r = (\X^\top\X)^{-1}\X^\top\r = \zero$ and:
  \begin{align*}
    \E\, \Big\|\frac1k\X^+\Q_\pi \r\Big\|^2 &= \E\,\Big\|\frac1k
    \sum_{i=1}^k\z_{\pi_i}\Big\|^2 = \frac1k
    \E\,\|\z_{\pi_1}\|^2\\
    &=\sum_{i=1}^nq_i \frac{r_i^2 \| (\X^+)_{:,i}\|^2}{q_i^2}\leq
\frac{\beta d\!+\! b}{k}\|\r\|^2
  \end{align*}
\end{frame}


\begin{frame}
  \frametitle{Chernoff bound for mixture}
  Consider a distribution $q=(q_1,\dots,q_n)$ such that
  \begin{align*}
    q_i = \frac{\beta\ell_i+ \| (\X^+)_{:,i}\|^2}{\beta d +b}.
  \end{align*}
  Let $\Z_i = \frac1{q_i}\u_i\u_i^\top$.
  Sample $\pi_1,\dots,\pi_k\sim q$.\\[1mm]
  Matrix Chernoff for matrices
  $\|\Z_{\pi_i}\|\leq \frac{\beta d + b}{\beta}= d + b/\beta$ says we
  need $k\geq (d+b/\beta)\log d/\delta$.
%   \begin{align*}
%     \Pr\bigg(\lambda_{\min}\Big(\frac1k\sum_{i=1}^k\z_{\pi_i}\z_{\pi_i}^\top\Big)
%     \leq \frac12\mu_{\min}\bigg)\leq d \text{e}^{-k\mu_{\min}/b}
%   \end{align*}
% where  $\mu_{\min}=\lambda_{\min}\big(\X^+\X^{+\top}\big)=\lambda_{\min}\big((\X^\top\X)^{-1}\big)$.
\end{frame}

\begin{frame}
  \frametitle{Selecting the optimal mixture}
  To get subspace embedding and matrix multiplication we need:
  \begin{align*}
    k = O\big( (d+b/\beta)\log n + (\beta d + b)/\epsilon\big)
  \end{align*}
  Selecting $\beta = \sqrt{\frac{b\epsilon}{d}\log n}$, we get:
  \begin{align*}
    (d+b/\beta)\log n + (\beta d + b)/\epsilon
    &= d\log n +
    2\sqrt{(b/\epsilon)d\log n} + b/\epsilon\\
&=\big(\sqrt{d\log n} + \sqrt{b/\epsilon}\big)^2\\
&\leq 2\,(d\log n + b/\epsilon)
    \end{align*}
  \end{frame}

\begin{frame}
  \frametitle{Leverage score sampling analysis}
  \begin{align*}
\hspace{-6mm}    \|\w_\pi^*-\w^*\|
    &= \big\|(\X^\top\Q_\pi\X)^{-1}\X^\top\Q_\pi
      \overbrace{(\y - \X\w^*)}^{\r}\big\|\\
    &\leq
      \Big\|\Big(\frac1k\X^\top\Q_\pi\X\Big)^{-1}\X^\top\X\Big\|\cdot
      \Big\|\frac1k(\X^\top\X)^{-1}\X^\top\Q_\pi \r\Big\|\\
    &\leq
      \Big\|\Big(\frac1k\X^\top\Q_\pi\X\Big)^{-1}\X^\top\X\Big\|\cdot
      \big\|(\X^\top\X)^{-\frac12}\big\|\cdot
      \Big\|\frac1k\underbrace{(\X^\top\X)^{-\frac12}\X^\top}_{\U^\top}\Q_\pi\r\Big\|
  \end{align*}
  For $k\geq C\cdot d\log d/\delta$ we can bound the first term by a
  constant.\\
  For $k\geq C\cdot d/(\delta\epsilon)$ we can bound the third term by
  $\sqrt{\epsilon}\cdot \|\r\|$. So:
  \begin{align*}
    \|\w_\pi^*-\w^*\|^2\leq \frac{\epsilon}{\lambda_{\min}(\X^\top\X)}
    L^*\leq \epsilon\,b\cdot L^*.
  \end{align*}

\end{frame}


\section{New sampling algorithm}

\begin{frame}
  \frametitle{Main algorithmic result and comparison to prior work}
  Previous volume sampling algorithms:
  \begin{enumerate}
  \item Classical bottom-up sampling: $O(nd^2)$ per sample [...]\\[2mm]
  \item Reverse iterative sampling: $O(nd^2)$ per sample [DW18b]\\[2mm]
  \item Det. rejection sampling: $O(d^4)$ per sample
    [DWH18a]\\
    One-time preprocessing: $O(nd^2)$ or $O(nd\log n + d^4\log d)$
  \end{enumerate}
  \vspace{5mm}
  
  \textbf{Theorem.} \  Classical bottom-up sampling can be implemented in
  time $O(d^3\log d)$ per sample (after $O(nd^2)$ preprocessing).
  
\end{frame}

  \begin{frame}
    \frametitle{Classical bottom-up algorithm in $nd^2$ time}
    Preprocessing: we compute matrix $\U=\X(\X^\top\X)^{-\frac12}$.\\[2mm]
    
    $\verb~Bottom-up~(\U)$:\vspace{0mm}
    
  \begin{algorithmic}[1]
    \STATE \textbf{input:} $\U\in\R^{n\times d}$, such that $\U^\top\U=\I$
    % \STATE $\forall _{i=1}^n\ \ubt_i \leftarrow\u_i$
    \STATE $\U^{(0)} \leftarrow \U$
    \STATE \textbf{for }$k=1..d$
    \STATE  \quad Sample $\sigma_k \sim \big(\|\u_1^{(k-1)}\|^2,\dots,\|\u_n^{(k-1)}\|^2\big)$
    \STATE \quad$\U^{(k)}\leftarrow \U^{(k-1)}\Big(\I -
    \frac{\u_{\sigma_k}\u_{\sigma_k}^\top}{\|\u_{\sigma_k}\|^2}\Big)$
    \STATE \textbf{end for}
    \RETURN $S = \{\sigma_1,\dots,\sigma_d\}$
  \end{algorithmic}
  Note that $\U^{(k) \top}\U^{(k)}$ is a $(d-k)$-dimensional projection
  so
  \begin{align*}
    \sum_{i=1}^n \big\|\u_i^{(k-1)}\big\|^2
    &= \tr\big(\U^{(k)\top}\U^{(k)}\big) = d-k,\\
\Pr(S)
&=d!\cdot\prod_{k=0}^{d-1}\frac{\|\u_{\sigma_k}^{(k-1)}\|^2}{d-k}
=\det(\U_S)^2.       
  \end{align*}
\end{frame}

\begin{frame}
  \frametitle{Sampling from projected leverage scores}
Given $\X$ and $\U=\X(\X^\top\X)^{-\frac12}$ and set $S$ of size $k$
let:
\begin{align*}
  \P_S = \prod_{i\in S}\bigg(\I -
  \frac{\u_i\u_i^\top}{\|\u_i\|^2}\bigg)\quad (\text{note:
  }\text{rank}(\P_S) = d-k)
\end{align*}
Consider rejection sampling procedure:\\[1mm]
1. Sample $i\sim
\big(\frac{\|\u_1\|^2}{d},\dots,\frac{\|\u_n\|^2}{d}\big)$\\
2. Accept with probability $\frac{\|(\U\P_S)_i\|^2}{\|\u_i\|^2}$
(trivially bounded by 1)\\[5mm]
This samples $i\sim \|(\U\P_S)_i\|^2$. Expected acceptance probability is:
\begin{align*}
  \E\bigg[\frac{\|(\U\P_S)_i\|^2}{\|\u_i\|^2}\bigg]
  = \sum_{i=1}^n\frac{\|\u_i\|^2}{d}\,
  \frac{\|(\U\P_S)_i\|^2}{\|\u_i\|^2} = \frac{\tr(\P_S\U^\top\U\P_S)}{d} = \frac{d-k}{d}.
\end{align*}
\end{frame}

\begin{frame}
  \frametitle{Bottom-up sampling using leverage scores}
  $\verb~Bottom-up~(\U,\P)$,\quad \big(preprocessing computes
  $\U=\X(\X^\top\X)^{-\frac12}$\big)
  \begin{enumerate}
  \item Let $r=\text{rank}(\P)$. If $r = 0$ then return $\{\}$
  \item Repeat until accepted
    \begin{enumerate}
      \item Sample $i \sim
        \big(\frac{\|\u_1\|^2}{d},\dots,\frac{\|\u_n\|^2}{d}\big)$
      \item Accept w.p. $\frac{\|(\U\P)_i\|^2}{\|\u_i\|^2}$
      \end{enumerate}
    \item $S \sim \verb~Bottom-up~\bigg(\U,\ \  \P\cdot \Big(\I -
      \frac{\P\u_i\u_i^\top\P}{\|(\U\P)_i\|^2}\Big)\bigg)$
    \item Return $\{i\} \cup S$
    \end{enumerate}
    Expected number of rejection sampling trials is:
    \begin{align*}
      \sum_{k=0}^{d-1}\frac{d}{d-k} = d\sum_{i=1}^d\frac1i= O\big(d\ln(d)\big)
    \end{align*}
Trial is $d^2$, updating $\P$ is $d^2$, so expected
runtime is $O(d^3\log d)$\\
\textbf{Note:} \ There is a one-time preprocessing cost of $O(nd^2)$.
\end{frame}


\end{document}
