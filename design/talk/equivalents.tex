
\documentclass[10pt]{beamer}
%\beamertemplateshadingbackground{brown!70}{yellow!10}
\mode<presentation>
{
  %\usetheme{Warsaw}
  \usecolortheme{crane}
  % or ...

%  \setbeamercovered{transparent}
  % or whatever (possibly just delete it)
}
\setbeamertemplate{navigation symbols}{}
\setbeamertemplate{footline}[frame number]{}
\usepackage{tikz,pgfplots}
\pgfplotsset{compat=newest}
\usepackage[utf8]{inputenc}
\usetikzlibrary{patterns}
\usepackage{amssymb}
\usepackage{amsmath}
\usepackage{colortbl}
%\usepackage{multicol}
\usepackage{cancel}
\usepackage{ulem}
\usepackage{multirow}
\usepackage{relsize}
\usepackage{algorithm}
\usepackage{algorithmic}
\usepackage{forloop}% http://ctan.org/pkg/forloop
\newcounter{loopcntr}
\newcommand{\rpt}[2][1]{%
  \forloop{loopcntr}{0}{\value{loopcntr}<#1}{#2}%
}
%\pagestyle{plain}
%\input{defs2}
\def\opt{{\textsc{OPT}_k}}
\def\const{{\mathrm{const}}}
\def\nnz{{\mathrm{nnz}}}
\def\r{\sfrac{\sigma_{\w}^2}{\sigma_{\xib}^2}}
\def\rm{\sfrac{\sigma_{\xib}^2}{\sigma_{\w}^2}}
\def\cmark{\Green{\checkmark}}
\def\xmark{\Red{\large\sffamily x}}
\newcommand{\pdet}{{\mathrm{pdet}}}
\newcommand{\MSPE}[1] {{\mathrm{MSPE}\big[#1\big]}}
\newcommand{\MSE}[1] {{\mathrm{MSE}\big[#1\big]}}
\def\Poisson{{\operatorname{Poisson}}}
\def\PB{{\operatorname{PB}}}
\newcommand{\DP}[1]{\mathcal{DP}^{#1}}
\def\Ic{\mathcal{I}}
\def\Jc{\mathcal{J}}
\def\Mc{\mathcal M}
\def\Ec{\mathcal E}
\def\sr{{\mathrm{sr}}}
\def\ktd{{k^{\underline{d}}}}
\def\Det{{\mathrm{Det}}}
\def\detu{{\widecheck{\mathrm{Det}}_\mu^\gamma}}
\def\deto{{\widehat{\mathrm{Det}}_\mu^\gamma}}
\def\Zu{{\widecheck{Z}_\mu^{\gamma}}}
\def\Zo{{\widehat{Z}_\mu^{\gamma}}}
\def\Zun{{\widecheck{Z}_\mu^{\gamma_n}}}
\def\Zon{{\widehat{Z}_\mu^{\gamma_n}}}
\newcommand{\Er}{\mathrm{Er}}
\newif\ifDRAFT
\DRAFTtrue
\ifDRAFT
\newcommand{\marrow}{\marginpar[\hfill$\longrightarrow$]{$\longleftarrow$}}
\newcommand{\niceremark}[3]
   {\textcolor{red}{\textsc{#1 #2:} \marrow\textsf{#3}}}
\newcommand{\ken}[2][says]{\niceremark{Ken}{#1}{#2}}
\newcommand{\manfred}[2][says]{\niceremark{Manfred}{#1}{#2}}
\newcommand{\michael}[2][says]{\niceremark{Michael}{#1}{#2}}
\newcommand{\michal}[2][says]{\niceremark{Michal}{#1}{#2}}
\newcommand{\feynman}[2][says]{\niceremark{Feynman}{#1}{#2}}
%\usepackage[inline]{showlabels}
\else
\newcommand{\ken}[1]{}
\newcommand{\michael}[1]{}
\newcommand{\michal}[1]{}
\newcommand{\feynman}[1]{}
\fi
\newcommand{\norm}[1]{{\| #1 \|}}

\newcommand{\deff}{d_{\textnormal{eff}}}
\def\ee{\mathrm{e}}
\newcommand\mydots{\makebox[1em][c]{.\hfil.\hfil.}}
\def\Sd{\mathscr{S}_{\!d}}
\newcommand{\dx}{\dxy_{\!\cal X}}
\newcommand{\dxk}{\dxy_{\!\cal X}^k}
\newcommand{\dk}{\dxy^k}
\newcommand{\dxy}{\mathrm{D}}
\def\simiid{\overset{\textnormal{\fontsize{6}{6}\selectfont
i.i.d.}}{\sim}}
%\newcommand{\Dxy}{D_{\!\cal X\!,\cal Y}}
\def\vskx{{\mathrm{VS}_{\!\dx}^k}}
\def\vsk{{\mathrm{VS}_{\!D}^k}}
\def\vskxm{{\mathrm{VS}_{\!\dx}^{k-1}}}
\def\vskm{{\mathrm{VS}_{\!D}^{k-1}}}
\def\vsdx{{\mathrm{VS}_{\!\dx}^d}}
\def\vsd{{\mathrm{VS}_{\!D}^d}}
\newcommand{\vs}[1]{{\mathrm{VS}_{\!D}^{#1}}}
\newcommand{\sigd}{\boldsymbol\Sigma_{\!\dx}}
\def\wols{\w_{\mathrm{LS}}}
\def\wds{\boldsymbol\w_{\!D}^*}
\def\kd{K_{\!\dx}}

\def\poly{{\mathrm{poly}}}
\def\polylog{{\mathrm{polylog}}}
\def\DPP{{\mathrm{DPP}}}
\def\DPPcor{{\DPP_{\!\mathrm{cor}}}}
\def\DPPens{{\DPP_{\!\mathrm{ens}}}}
\newcommand{\DPPreg}[1]{{\DPP_{\!\mathrm{reg}}^{#1}}}
\def\Vol{{\mathrm{VS}}}
\def\Lev{{\mathrm{Lev}}}
\newcommand\todod[1]{\Red{\# DH: #1}}
\newcommand{\explain}[2]{\mathrel{\overset{\makebox[0pt]{\text{\tiny
#1}}}{#2}}}
\def\tot {{\mathrm{tot}}}
\def\checkmark{\tikz\fill[scale=0.4](0,.35) -- (.25,0) --
(1,.7) -- (.25,.15) -- cycle;}
\newcommand{\mnote}[1]{{\bf\large \Magenta{*}}\marginpar{\small \Magenta{#1}}}
\newcommand{\bnote}[1]{{\bf #1}}

\newcommand{\sqrtshort}[1]{{\sqrt{\white{\Big|}\!\!\smash{\text{\fontsize{9}{9}\selectfont$#1$}}}}}
\newenvironment{proofof}[2]{\par\vspace{2mm}\noindent\textbf{Proof of {#1} {#2}}\ }{\hfill\BlackBox}
\newcommand{\sets}[2]
{{\hspace{-0.3mm}[\hspace{-0.3mm}#1\hspace{-0.3mm}]\hspace{-0.3mm}\choose
\hspace{-0.3mm}#2\hspace{-0.3mm}}}
\DeclareMathOperator{\sgn}{\textnormal{sgn}}
\DeclareMathOperator{\adj}{\textnormal{adj}}
\def\Rb{{\mathbf{R}}}
\DeclareMathOperator{\ws}{\widetilde{\w}}
\newcommand{\inote}[1]{{\bf {#1}}}
\def\xib{\boldsymbol\xi}
\def\Sigmab{\mathbf{\Sigma}}
\def\Sigmabh{\widehat{\Sigmab}}
\def\Sigmabt{\widetilde{\Sigmab}}
\def\S{\mathbf{S}}
\def\T{\mathbf{T}}
\def\xt{\tilde{x}}
\def\xbt{\widetilde{\x}}
\def\xbh{\widehat{\x}}
\def\ubh{\widehat{\u}}
\def\dom {{\mathrm{dom}}}
\def\val {{\mathrm{val}}}
\def\out {{\mathrm{out}}}
\def\iin  {{\mathrm{iin}}}
\def\s {\mathbf{s}}
\def\q {\mathbf{q}}
\def\qt{\tilde{q}}
\def\itld {j}
\def\ubt {\tilde{\u}}
\def\n{\{1..n\}}
\def\cb {\mathbf{c}}
\def\cW{\mathcal W}
\def\Xt{\widetilde{X}}
\def\Dbt{\widetilde{\D}}
\def\xtb{\tilde{\mathbf{x}}}
\def\ytb{\tilde{\mathbf{y}}}
\def\Xtb{\widetilde{\mathbf{X}}}
\def\Xbb{\overline{\X}}
\def\Xb{{\bar{\X}}}
\def\ybb{\overline{\y}}
\def\f{{\mathbf{f}}}
\def\g{{\mathbf{g}}}
\def\fbb{{\overline{\f}}}
\def\fb{{\overline{f}}}
\def\Xc{\mathcal{X}}
\def\W{\mathbf W}
\def\L{\mathbf{L}}
\def\Rb{\mathbf R}
\def\Pc{\mathcal{P}}
\def\Nc{\mathcal{N}}
\def\Pt{\widetilde{P}}
\def\Hc{\mathcal{H}}
\def\Wc{\mathcal{W}}
\def\Cc{\mathcal{C}}
\def\p{\mathbf p}
%\def\r{\mathbf r}
\def\Y{\mathbf Y}
\def\H{\mathbf H}
\def\K{\mathbf K}
\def\Kh{\widehat{K}}
\def\Kbh{{\widehat{\K}}}
\def\Q{\mathbf Q}
\def\Qbar{{\bar{\mathbf Q}}}
\def\Ytb{\widetilde{\mathbf{Y}}}
\def\c{{n-d\choose s-d}}
\DeclareMathOperator{\Proj}{Proj}
\newcommand{\Span}{\mathrm{span}}
\newcommand{\ofsubt}[1]{\mbox{\scriptsize \raisebox{0.25pt}{$(#1)$}}}
%\raisebox{0.5pt}{$($}}#1\mbox{\tiny \raisebox{0.5pt}{$)$}}}
\newcommand{\ofsub}[1]{\mbox{\small \raisebox{0.0pt}{$(#1)$}}}
%\newcommand{\ofsubb}[1]{\mbox{\footnotesize \raisebox{0.5pt}{$(#1)$}}}
%\newcommand{\ofsub}[1]{(#1)}
%\newcommand{\ofsub}[1]{\mbox{\tiny$|$\hspace{-0.5pt}\raisebox{-0.5pt}{$#1$}}}
\newcommand{\of}[2]{{#1{\!\ofsub{#2}}}}
\newcommand{\oft}[2]{{#1{\!\ofsubt{#2}}}}
\newcommand{\fof}[2]{{#1({#2})}}
\newcommand{\yof}[2]{{#1{\ofsub{#2}}}}
%\newcommand{\yofb}[2]{{#1{\ofsubb{#2}}}}
\newcommand{\lazy}{FastRegVol}
\newcommand{\volsamp}{RegVol}

\newcommand{\Sm}{{S_{-i}}}
\newcommand{\Sp}{{S_{+i}}}
\ifx\BlackBox\undefined
\newcommand{\BlackBox}{\rule{1.5ex}{1.5ex}}  % end of proof
\fi
%\renewcommand{\dagger}{+}
\DeclareMathOperator*{\argmin}{\mathop{\mathrm{argmin}}}
\DeclareMathOperator*{\argmax}{\mathop{\mathrm{argmax}}}
\DeclareMathOperator*{\diag}{\mathop{\mathrm{diag}}}
\def\x{\mathbf x}
\def\y{\mathbf y}
\def\ybh{\widehat{\mathbf y}}
\def\ybb{\bar{\mathbf y}}
\def\xbb{\bar{\mathbf x}}
\def\yb{{\bar y}}
\def\ybt{\widetilde{\mathbf y}}
\def\yh{\widehat{y}}
\def\yhb{\widehat{\y}}
\def\yt{\widetilde{y}}
\def\z{\mathbf z}
\def\a{\mathbf a}
\def\b{\mathbf b}
\def\w{\mathbf w}
\def\v{\mathbf v}
\def\m{\mathbf m}
\def\wbh{\widehat{\mathbf w}}
\def\wh{\widehat{\mathbf w}}
\def\vbh{\widehat{\mathbf v}}
\def\wbt{\widetilde{\mathbf w}}
\def\e{\mathbf e}
\def\zero{\mathbf 0}
\def\one{\mathbf 1}
\def\u{\mathbf u}
\def\ubbar{\bar{\mathbf u}}
\def\f{\mathbf f}
\def\ellb{\boldsymbol\ell}

\def\X{\mathbf X}
\def\Xs{\widetilde{\X}}
\def\B{\mathbf B}
\def\A{\mathbf A}
\def\C{\mathbf C}
\def\U{\mathbf U}
\def\Ubt{\widetilde{\mathbf U}}
\def\Ubh{\widehat{\mathbf U}}
\def\Ubbar{\bar{\mathbf U}}
\def\F{\mathbf F}
\def\D{\mathbf D}
\def\V{\mathbf V}
\def\M{\mathbf M}
\def\Mh{\widehat{\mathbf M}}
%\def\S{\mathbf S}
\def\Stb{\widetilde{\mathbf{S}}}
\def\Sbh{\widehat{\mathbf{S}}}
\def\St{\widetilde{\S}}
\def\Sh{\widehat{S}}
\def\Sc{\mathcal{S}}
\def\Fc{\mathcal{F}}
\def\Vc{\mathcal{V}}
\def\Bc{\mathcal{B}}
\def\Dc{\mathcal{D}}
\def\Z{\mathbf Z}
\def\Zbh{\widehat{\mathbf Z}}
\def\Zbt{\widetilde{\mathbf Z}}
\def\Abh{\widehat{\mathbf A}}
\def\I{\mathbf I}
\def\Ic{\mathcal I}
\def\II{\mathbf {I \!\,I}}
%\def\II{\boldsymbol {\mathbb I}}
\def\A{\mathbf A}
\def\P{\mathbf P}
\def\Ph{\widehat{\mathbf P}}
\def\cP{\mathcal P}
\def\cR{\mathcal R}
\def\Xt{\widetilde{\mathbf{X}}}
\def\Xh{\widehat{\mathbf{X}}}
\def\Rh{\widehat{R}}
\def\Ot{\widetilde{O}}
\def\At{\widetilde{\A}}


\def\E{\mathbb E}
\def\R{\mathbb R}
\def\N{\mathbb N}
\def\Pr{\mathrm{Pr}}
%\def\C{\mathbb C}
\def\tr{\mathrm{tr}}
\def\Sbar{{\bar{S}}}
\def\cS{{\mathcal{S}}}
\def\Tbar{{\bar{T}}}
\def\Tt{{\widetilde{T}}}
\def\rank{\mathrm{rank}}
\def\Prob{\mathrm{Prob}}
\def\Var{\mathrm{Var}}
\def\Xinv{(\X^\top\X)^{-1}}
\def\XinvS{(\X_S\X_S^\top)^{-1}}
\def\ABinvS{(\A_S\B_S^\top)^{-1}}
\def\ABinv{(\A\B^\top)^{-1}}
\def\xinv{\x_i^\top\Xinv\x_i}
\def\Xinvr{(\lambda\I+\X_{-1}^\top\X_{-1})^{-1}}
\def\pdet{\mathrm{pdet}}
\newcommand{\vol}{\mathrm{vol}}
%\newcommand{\defeq}{:=}
\newcommand{\defeq}{\stackrel{\textit{\tiny{def}}}{=}}
\newcommand{\di}{{[d+1]_{-i}}}
\newcommand{\cov}{\mathrm{cov}}
\let\origtop\top
\renewcommand\top{{\scriptscriptstyle{\origtop}}} % this makes transpose not so big

\definecolor{silver}{cmyk}{0,0,0,0.3}
\definecolor{yellow}{cmyk}{0,0,0.9,0.0}
\definecolor{reddishyellow}{cmyk}{0,0.22,1.0,0.0}
\definecolor{black}{cmyk}{0,0,0.0,1.0}
\definecolor{darkYellow}{cmyk}{0.2,0.4,1.0,0}
\definecolor{orange}{cmyk}{0.0,0.7,0.9,0}
\definecolor{darkSilver}{cmyk}{0,0,0,0.1}
\definecolor{grey}{cmyk}{0,0,0,0.5}
\definecolor{darkgreen}{cmyk}{0.6,0,0.8,0}
\newcommand{\Red}[1]{{\color{red}  {#1}}}
\newcommand{\Purple}[1]{{\color{purple}  {#1}}}
\newcommand{\Magenta}[1]{{\color{magenta}{#1}}}
\newcommand{\Green}[1]{{\color{darkgreen}  {#1}}}
\newcommand{\Blue}[1]{\color{blue}{#1}\color{black}}
\newcommand{\Orange}[1]{\textcolor{orange}{#1}\color{black}}
\newcommand{\Brown}[1]{{\color{brown}{#1}\color{black}}}
\newcommand{\Grey}[1]{{\color{grey}{#1}\color{black}}}
\newcommand{\white}[1]{{\textcolor{white}{#1}}}
\newcommand{\yellow}[1]{{\textcolor{reddishyellow}{#1}}}
\newcommand{\darkYellow}[1]{{\textcolor{darkYellow}{#1}}}
\newcommand{\grey}[1]{{\textcolor{grey}{#1}}}

\DeclareMathOperator{\half}{\frac{1}{2}}

\ifx\proof\undefined
\newenvironment{proof}{\par\noindent{\bf Proof\ }}{\hfill\BlackBox\\[2mm]}
\fi

\ifx\theorem\undefined
\newtheorem{theorem}{Theorem}
\fi

\ifx\example\undefined
\newtheorem{example}{Example}
\fi

\ifx\condition\undefined
\newtheorem{condition}{Condition}
\fi
\ifx\property\undefined
\newtheorem{property}{Property}
\fi

\ifx\lemma\undefined
\newtheorem{lemma}{Lemma}
\fi

\ifx\proposition\undefined
\newtheorem{proposition}{Proposition}
\fi

\ifx\remark\undefined
\newtheorem{remark}{Remark}
\fi

\ifx\corollary\undefined
\newtheorem{corollary}{Corollary}
\fi

\ifx\definition\undefined
\newtheorem{definition}{Definition}
\fi

\ifx\conjecture\undefined
\newtheorem{conjecture}{Conjecture}
\fi

\ifx\axiom\undefined
\newtheorem{axiom}{Axiom}
\fi

\ifx\claim\undefined
\newtheorem{claim}{Claim}
\fi

\ifx\assumption\undefined
\newtheorem{assumption}{Assumption}
\fi

\ifx\condition\undefined
\newtheorem{condition}{Condition}
\fi


\edef\polishl{\l}
\setlength{\columnsep}{0.7em}
\setlength{\columnseprule}{0mm}
\setlength{\arrayrulewidth}{1pt} 

\newcommand{\svr}[1]{{\textcolor{darkSilver}{#1}}}
\definecolor{brightyellow}{cmyk}{0,0,0.7,0.0}
\definecolor{lightyellow}{cmyk}{0,0,0.3,0.0}
\definecolor{lighteryellow}{cmyk}{0,0,0.1,0.0}
\definecolor{lightestyellow}{cmyk}{0,0,0.05,0.0}
\AtBeginSection[]
{
\begin{frame}<beamer>
\frametitle{Outline}
\tableofcontents[currentsection]
\end{frame}
}
\def\layersep{2.5cm}


%  \fboxsep=3pt
% %\fboxsep=0mm%padding thickness
% \fboxrule=2pt%border thickness

\setkeys{Gin}{width=0.7\textwidth}

\title[]{Deterministic equivalents}

\author[]{Micha{\l} Derezi\'{n}ski\\
UC Berkeley}

\begin{document}
\begin{frame}
  \titlepage
\end{frame}

\linespread{1.3}

\section{Resolvent}

\begin{frame}
  \frametitle{Setup: Resolvent}
  Let $\X\sim \Nc_{n,d}(\zero,\I_n\otimes\Sigmab)$.\\
  More generally, we can consider $\X = \Z\Sigmab^{\frac12}$ where
  $\Z$ has i.i.d. entries with variance 1.\\[5mm]

 For some $\lambda>0$, define
  \begin{align*}
    \Q(\lambda) = (\tfrac1{\lambda n}\X^\top\X+\I)^{-1},\qquad \Qbar(\lambda) =
    (\tfrac{\bar\gamma}{\lambda}\Sigmab + \I)^{-1}
  \end{align*}
When $\lambda$ is fixed, we will use $\Q$ instead of $\Q(\lambda)$\\
  We wish to show that for an appropriate choice of $\bar\gamma$:
  \begin{align*}
    \E[\Q] \simeq \Qbar.
    \end{align*}
  \end{frame}

  \begin{frame}
    We use: $\A^{-1}-\B^{-1} = \A^{-1}(\B-\A)\B^{-1}$
    \begin{align*}
      \Q-\Qbar &=
                 \Q(\tfrac{\bar\gamma}{\lambda}\Sigmab-
                 \tfrac1{\lambda n}\X^\top\X)\Qbar\\
      &=\tfrac{\bar\gamma}{\lambda}\Q\Sigmab\Qbar - \tfrac1{\lambda n}\Q\X^\top\X\Qbar
    \end{align*}
    Let  $\Q_{-i} =
  (\tfrac1{\lambda n}\X_{-i}^\top\X_{-i}+\I)^{-1}$ and
  $\gamma_i=(1+\frac1{\lambda n}\x_i^\top\Q_{-i}\x_i)^{-1}$.\\
  Sherman-Morrison implies that:
  \begin{align*}
    \Q\x_i\x_i^\top=\gamma_i\Q_{-i}\x_i\x_i^\top =
    \bar\gamma\Q_{-i}\x_i\x_i^\top + \Q_{-i}\x_i\x_i^\top(\gamma_i-\bar\gamma)
  \end{align*}
It follows that:
  \begin{align*}
    \E[\Q]-\Qbar &= \tfrac{\bar\gamma}{\lambda} \E[\Q]\Sigmab\bar\Q -
    \tfrac1{\lambda n}\sum_{i=1}^n\big(\bar\gamma\E[\Q_{-i}]\E[\x_i\x_i^\top] +
    \E[\Q_{-i}(\gamma_i-\bar\gamma)\x_i\x_i^\top]\big)\bar\Q \\
    &=\underbrace{\tfrac{\bar\gamma}{\lambda}\E[\Q-\Q_{-n}]\Sigmab\bar\Q}_{\A} +
\underbrace{\E[\Q_{-n}(\tfrac{\bar\gamma}{\lambda}-\tfrac{\gamma_n}{\lambda})\x_n\x_n^\top]\bar\Q}_{\B}
  \end{align*}
\end{frame}

\begin{frame}
  \frametitle{Analyzing the first term}
  Note that $\A\preceq\zero$, so $\A$ can be dropped if we only care
  about an upper bound. To get a lower bound we bound $-\A$ as follows:
  \begin{align*}
    -\A &=
    \E[\Q_{-n}-\Q]\tfrac{\bar\gamma}{\lambda}\Sigmab
          (\tfrac{\bar\gamma}{\lambda}\Sigmab+\I)^{-1} \\
        &\preceq \E[\Q_{-n}-\Q]\\
    &\!\!\underset{\lambda\rightarrow 0}{\longrightarrow} \
      \E\bigg[(\I-\P_{n-1}) -
      (\I-\P_{n-1})\Big(\I-\frac{\x_n\x_n^\top}{\x_n^\top\x_n}\Big)\bigg]\\
        &=\E[\I-\P_{n-1}]\,\E\bigg[\frac{\x_n\x_n^\top}{\x_n^\top\x_n}\bigg]\\
    &\preceq \E\bigg[\frac{\x_n\x_n^\top}{\x_n^\top\x_n}\bigg]
  \end{align*}
\end{frame}

\begin{frame}
\frametitle{Analyzing second term}
Let $\alpha_n=\frac1n\x_n^\top\Q_{-n}\x_n$ and let
$\bar\gamma=\frac1{1+\frac1{\lambda}\bar\alpha}$. We have:
\begin{align*}
  \tfrac{\gamma_n}\lambda = \frac1{\lambda + \alpha_n}
  \ \underset{\lambda\rightarrow 0}{\longrightarrow}\ \frac
  1{\alpha_n}
\end{align*}
and similarly, we have $\frac{\bar\gamma}{\lambda}\rightarrow \bar\alpha^{-1}$, so
\begin{align*}
  \|\B\|
  & \rightarrow
    \|\E[(\alpha_n^{-1}-\bar\alpha^{-1})(\I-\P_{n-1})\x_n\x_n^\top]\bar\Q\|\\
  &\leq \big\|\E\big[|\alpha_n^{-1}-\bar\alpha^{-1}|\cdot
    \x_n\x_n^\top \big]\bar\Q \big\|\\
  &=
    \big\|\E[|\tfrac{\bar\alpha}{\alpha_n}-1|\cdot
    \tfrac1{\bar\alpha}\x_n\x_n^\top](\tfrac1{\bar\alpha}\Sigmab+\I)^{-1}\big\| 
\end{align*}
Hanson-Wright implies that $|\frac{\bar\alpha}{\alpha_n}-1| =
O(n^{-1/2})$ with high probability\\
Then we have
$\frac1{\bar\alpha}\E[\x_n\x_n^\top](\tfrac1{\bar\alpha}\Sigmab+\I)^{-1}
=\frac1{\bar\alpha}\Sigmab (\tfrac1{\bar\alpha}\Sigmab+\I)^{-1}\preceq \I$
\end{frame}

\begin{frame}
  \frametitle{Hanson-Wright for projections}
  Let $\P$ be a projection matrix. Then for $\z\sim\Nc(\zero,\I)$,
  \begin{align*}
    \Pr\big\{|\z^\top\Sigmab^{\frac12}\P\Sigmab^{\frac12}\z -
    \tr(\P\Sigmab)|>t\big\}
    \leq 2\exp\big(-c\min\big\{\tfrac{t^2}{\|\Sigmab\|_F^2},\tfrac t{\|\Sigmab\|_2}\big\}\big)
  \end{align*}
\end{frame}

\section{Projection}

\begin{frame}
  \frametitle{Setup}
  Let $\X\sim \Nc_{n,d}(\zero,\I_n\otimes\Sigmab)$.\\
  More generally, let $\X = \Z\Sigmab^{\frac12}$ where
  $\Z$ has i.i.d. entries with variance 1.\\[5mm]

We define a deterministic equivalent for the projection matrix:
  \begin{align*}
    \P =\I-\X^\dagger\X,\qquad \bar\P =
    (\bar\gamma\Sigmab + \I)^{-1}
  \end{align*}
  We wish to show that for an appropriate choice of $\bar\gamma$:
  \begin{align*}
    \E[\P] \simeq \bar\P.
    \end{align*}
  \end{frame}

  \begin{frame}
    \frametitle{Reduction}
    Let $\X = \Z\Sigmab^{\frac12}$ and
    $\X_\alpha=\Z\Sigmab_\alpha^{\frac12}$, where
    $\Sigmab_\alpha=\alpha\Sigmab$.\\
Let $\P_\alpha=\I-\X_\alpha^\dagger\X_\alpha$ and
    $\bar\P_\alpha$ be the deterministic equivalent for $\P_\alpha$\\[3mm]

    Note that $\X_\alpha^\dagger\X_\alpha=\X^\dagger\X$ so
    $\P_\alpha=\P$. \\
    This implies that also $\bar\P_\alpha=\bar\P$, and
    \[\E[\P_\alpha]-\bar\P_\alpha = \E[\P]-\bar\P,\]
    so, setting $\alpha=\frac1{\|\Sigmab\|}$, we can replace $\Sigmab$
    with $\Sigmab_\alpha$ such that
    $\|\Sigmab_\alpha\|=1$.\\[5mm]
    \textbf{Conclusion:} W.l.o.g.~we can assume that $\|\Sigmab\|=1$.
  \end{frame}

  \begin{frame}
    \frametitle{The scale of $\bar\gamma$}
    Let $r = \tr(\Sigmab)/\|\Sigmab\|$ denote the nuclear rank of $\Sigmab$.\\
    W.l.o.g., assume that $\|\Sigmab\|=1$ (otherwise, scale $\bar\gamma$ by
    $1/\|\Sigmab\|$)\\
    We define $\bar\gamma$ so that $\tr\,(\bar\gamma\Sigmab+\I)^{-1}=d-n$
    \begin{lemma}
      Assuming that $\|\Sigmab\|=1$, we have $\bar\gamma\geq \frac nr$, and if $n<r$ then also
      $\bar\gamma\leq \frac n{r-n}$.
      \end{lemma}
      \begin{proof}
        Let $1=\lambda_1\geq\lambda_2\geq...$ be the eigenvalues of
        $\Sigmab$. Then:
        \begin{align*}
          n &= \sum_{i=1}^d
          \frac{\bar\gamma\lambda_i}{\bar\gamma\lambda_i+1}\leq
          \bar\gamma\sum_{i=1}^d\lambda_i=\bar\gamma\cdot r,\\
          n &= \sum_{i=1}^d
          \frac{\bar\gamma\lambda_i}{\bar\gamma\lambda_i+1}\geq
              \frac{\bar\gamma}{\bar\gamma+1}\sum_{i=1}^d\lambda_i
              =\frac{\bar\gamma}{\bar\gamma+1}\cdot r.
        \end{align*}
      \end{proof}
  \end{frame}

  \begin{frame}
    \frametitle{Rank one update for pseudoinverse}
    \begin{lemma}
      We have the following Sherman-Morrison type formula:
      \begin{align*}
        (\X^\top\X)^\dagger\x_i =
        \frac{(\I-\X_{-i}^\dagger\X_{-i})\x_i}{\x_i^\top(\I-\X_{-i}^\dagger\X_{-i})\x_i}. 
        \end{align*}
      \end{lemma}
      We prove it by taking a limit.\\
    Let $\Q = (\X^\top\X+\lambda\I)^{-1}$ and note that
    $\Q\x_i\rightarrow (\X^\top\X)^\dagger\x_i$ as
    $\lambda\rightarrow 0$\\
Moreover, let $\x_i=\u_i+\v_i$, where
$\u_i=\X_{-i}^\dagger\X_{-i}\x_i$ and $\v_i=(\I-\X_{-i}^\dagger\X_{-i})\x_i$.
    Sherman-Morrison implies that:
  \begin{align*}
    \Q\x_i
    &=\frac{\Q_{-i}\x_i}{1+\x_i^\top\Q_{-i}\x_i} 
    =\frac{\Q_{-i}\u_i + \frac1\lambda\v_i}{1+
      \u_i^\top\Q_{-i}\u_i + \frac1\lambda\v_i^\top\v_i}\\
    &\!\!\!\underset{\lambda\rightarrow0}{\longrightarrow}
      \ \frac{\v_i}{\v_i^\top\v_i}
      =\frac{(\I-\X_{-i}^\dagger\X_{-i})\x_i}{\x_i^\top(\I-\X_{-i}^\dagger\X_{-i})\x_i}
  \end{align*}
\end{frame}

\begin{frame}
  \frametitle{Two terms}
In place of the resolvent identity (since $\P$ is not invertible) we use:
  \begin{align*}
\P\bar\P^{-1}-\I &= \P(\bar\gamma\Sigmab+\I) -
    \I
    =\P\bar\gamma\Sigmab - (\I-\P) 
  \end{align*}
  \textit{Note:} To prove second conjecture we should bound $\|\E[\P]\bar\P^{-1}-\I\|$\\[2mm]
  Let $\gamma_i = n/\x_i^\top\P_{-i}\x_i$, where
$\P_{-i}=\I-\X_{-i}^\dagger\X_{-i}$. \\
Since $\I-\P=(\X^\top\X)^\dagger\X^\top\X$, it follows that
  \begin{align*}
    \E[\P]\bar\P^{-1}-\I
&=\E[\P\bar\gamma\Sigmab] -
\sum_{i=1}^n\E[(\X^\top\X)^\dagger\x_i\x_i^\top]\\
&=\E[\P]\bar\gamma\Sigmab-n\,\E[(\X^\top\X)^\dagger\x_n\x_n^\top]\\
&=\E[\P]\bar\gamma\Sigmab-\E[\P_{-n}\x_n\x_n^\top\gamma_n]\\
&=\underbrace{\E[\P-\P_{-n}]\bar\gamma\Sigmab}_{\A} -
\underbrace{\E[\P_{-n}\x_n\x_n^\top(\gamma_n-\bar\gamma)]}_{\B}.
  \end{align*}
\end{frame}

\begin{frame}
  \frametitle{Analyzing the first term}
Note that $\A\preceq \zero$. Moreover, we have
  \begin{align*}
    -\A
    &=\E[\P_{-n}-\P]\bar\gamma\Sigmab
    \\
    &=\E\bigg[\frac{\P_{-n}\x_n\x_n^\top\P_{-n}}{\x_n^\top\P_{-n}\x_n}\bigg]
      \,\bar\gamma\Sigmab\\
    &=\E\bigg[\frac{\P_{-n}\Sigmab\P_{-n}}{\tr(\P_{-n}\Sigmab\P_{-n})}\bigg]\bar\gamma\Sigmab\\
    &=\E\bigg[\frac{\P_{-n}\Sigmab}{\tr(\P_{-n}\Sigmab)}
      \P_{-n}\bar\gamma\Sigmab\bigg]\\
    &\preceq \frac1{r_n}\cdot\E[\P_{-n}]\bar\gamma\Sigmab,
  \end{align*}
  where $r_n = \sum_{i\geq n}\lambda_i$ and $\lambda_i$ are the
  eigenvalues of $\Sigmab$\\
  Note that we should be able to show that
  $\E[\P_{-n}]\bar\gamma\Sigmab= O(1)\cdot \I$
\end{frame}

\begin{frame}
  \frametitle{Bounding a rank-one projection}
  Let $\x\sim\Nc(\zero,\Sigmab)$. Note that $\x=\Sigmab^{\frac12}\z$
  for $\z\sim\Nc(\zero,\I)$\\
  Also, let $\lambda_{\min}$ denote the smalles eigenvalue of
  $\Sigmab$. It follows that:
  \begin{align*}
    \E\Big[\frac{\x\x^\top}{\x^\top\x}\Big]
&= \Sigmab^{\frac12}\E\Big[\frac{\z\z^\top}{\z^\top\Sigmab\z}\Big]\Sigmab^{\frac12}
\preceq \frac1{\lambda_{\min}}\Sigmab^{\frac12}\E\Big[\frac{\z\z^\top}{\z^\top\z}\Big]\Sigmab^{\frac12} 
  \\ &  =\frac{1}{\lambda_{\min}}\Sigmab^{\frac12}\,\frac1d\I\,\Sigmab^{\frac12}
  =\frac1{d\lambda_{\min}}\Sigmab \preceq \frac{\kappa}{d}\cdot \I
  \end{align*}
  where $\kappa$ is the condition number of $\Sigmab$.\\[5mm]
  
\textit{Note:} The dependence on $\lambda_{\min}$ may be avoidable
\end{frame}

\begin{frame}
  \frametitle{Better bound for a rank-one projection}
Assume that $\Sigmab=\diag(\lambda_{i})$,
  $\|\Sigmab\|=1$, let $Z_i=\sum_{j\neq i}\lambda_jz_j^2$ and $r=\tr(\Sigmab)$.
  \begin{align*}
    \bigg[\E\Big[\frac{\x\x^\top}{\x^\top\x}\Big]\bigg]_{ii}
    &=\E\bigg[\E\Big[\frac{\lambda_iz_i^2}{\lambda_iz_i^2+Z_i}\mid Z_i\Big]\bigg]
   \leq
      \E\Big[\frac{\lambda_i\E[z_i^2]}{\lambda_i\E[z_i^2]+Z_i}\Big]\\
    &\leq \frac{\lambda_i}{\lambda_i+\frac12\E[Z_i]} + \Pr\big\{|Z_i-\E[Z_i]|>\tfrac12\E[Z_i]\big\}
  \end{align*}
  Using Hanson-Wright, we have:
  \begin{align*}
    \Pr\big\{|Z_i-\E[Z_i]|>\tfrac12\E[Z_i]\big\}\leq
    2\ee^{-c\min\big\{\frac{(\sum_{j\neq i}\lambda_j)^2}{\sum_{j\neq
    i}\lambda_j^2},\frac{\sum_{j\neq
    i}\lambda_j}{\lambda_1}\big\}} \leq 2\ee^{-c(r-1)}
  \end{align*}
  So  $\big[\E\frac{\x\x^\top}{\x^\top\x}\big]_{ii}\leq
  \frac{2\lambda_i}{r}+2\ee^{-c(r-1)}$. Also, 
  $\big[\E\frac{\x\x^\top}{\x^\top\x}\big]_{ij}=0$ for $i\neq j$. So,
  \begin{align*}
    \E\Big[\frac{\x\x^\top}{\x^\top\x}\Big]\preceq
    \frac2{r}\cdot\Sigmab + 2\ee^{-c(r-1)}\cdot \I
  \end{align*}
\end{frame}

\begin{frame}
  \frametitle{Bounding higher moments}
  Let $\z\sim\Nc(\zero,\I)$, with $\bar\z=\frac{\z}{\|\z\|}$ and
  $r=\|\z\|$.\\
  Since $\bar\z$ is independent of $r$, for any $\v\in\R^d$ we have
  \begin{align*}
\E[(\v^\top\z)^{2k}] = \E[(\v^\top\bar\z)^{2k}r^{2k}] =
    \E[(\v^\top\bar\z)^{2k}]\cdot \E[r^{2k}].
  \end{align*}
  Note that $r^2\sim\chi^2(d)$ and
  $\v^\top\z\sim\Nc(\zero,\v^\top\v)$, so
  \begin{align*}
    \E[(\v^\top\bar\z)^{2k}] = \E[(\v^\top\z)^{2k}]\frac1{\E[r^{2k}]}
    = (2k-1)!! (\v^\top\v)^k\frac{(d-2)!!}{(d+2k-2)!!}.
  \end{align*}
  In particular, for $k=2$ we get
  $\E[(\v^\top\bar\z)^4]=\frac{3(\v^\top\v)^2}{d(d+2)}$. If
  $\x=\Sigmab^{\frac12}\z$, then
  \begin{align*}
    \E\bigg[\Big(\frac{\v^\top\x\x^\top\v}{\x^\top\x}\Big)^2\bigg]\leq
    \frac1{\lambda_{\min}^2}\E\big[(\v^\top\Sigmab^{\frac12}\bar\z)^4\big]
    =\frac{3(\v^\top\Sigmab\v)^2}{\lambda_{\min}^2d(d+2)}.
  \end{align*}
\end{frame}

\begin{frame}
  \frametitle{Better bound for the second moment}
Assume that $\Sigmab=\diag(\lambda_{i})$ and $r=\tr(\Sigmab)$. Fix
$\v\in\R^d$.\\
Let $Z_i=\sum_{t\neq i}\lambda_tz_t$, $Z_{ij}=\sum_{t\neq
  i,j}\lambda_tz_t$, and $\delta = 2\ee^{-c(r-2)}$ be H-W bound.
\begin{align*}
\hspace{-8mm}  
&\E\bigg[\frac{(\v^\top\x)^4}{\|\x\|^4}\bigg] =
    \E\bigg[\frac{(\sum_i \sqrt{\lambda_i}v_iz_i)^4}{\|\x\|^4}\bigg]
  = \sum_iv_i^4
\E\bigg[\frac{\lambda_i^2z_i^4}{\|\x\|^4}\bigg]+
    \sum_{i\neq
    j}v_i^2v_j^2\E\bigg[\frac{\lambda_i\lambda_jz_i^2z_j^2}{\|\x\|^4}\bigg]\\
\hspace{-8mm}  &=  \sum_iv_i^4
    \E\Bigg[\E\bigg[\frac{\lambda_i^2z_i^4}{(\lambda_iz_i^2+Z_i)^2}\mid z_i\bigg]\Bigg] +
\sum_{i\neq j}v_i^2v_j^2
\E\Bigg[\E\bigg[\frac{\lambda_i\lambda_jz_i^2z_j^2}
{(\lambda_iz_i^2+\lambda_jz_j^2+Z_{ij})^2}\mid 
    z_i,z_j\bigg]\Bigg]\\
\hspace{-8mm}    &\leq  \sum_iv_i^4
\Bigg(\E\bigg[\frac{\lambda_i^2z_i^4}{(\lambda_iz_i^2+\frac12\E[Z_i])^2}\bigg] \!+ \delta\!\Bigg) +
    \sum_{i\neq
    j}v_i^2v_j^2\Bigg(\E\bigg[\frac{\lambda_i\lambda_jz_i^2z_j^2}{(\lambda_iz_i^2+\lambda_jz_j^2+\frac12\E[Z_{ij}])^2}\bigg]
                   \!+\!\delta\Bigg)\\
  \hspace{-8mm}
  &\leq \sum_iv_i^4
\bigg(\frac{\lambda_i^2\E[z_i^4]}{(\frac12\E[Z_i])^2} \!+ \delta\!\bigg) +
    \sum_{i\neq j}v_i^2v_j^2
    \Bigg(\frac{\lambda_i\lambda_j\E[z_i^2]\E[z_j^2]}{(\frac12\E[Z_{ij}])^2}
    \!+\!\delta\Bigg)\\
  \hspace{-8mm}
  &\leq \sum_{i,j}v_i^2v_j^2\Big(\frac{12\lambda_i\lambda_j}{(r-2)^2}
    + \delta\Big) = \frac{12}{(r-2)^2}\big(\v^\top\Sigmab\v\big)^2 + 2\ee^{-c(r-2)}\|\v\|^4
\end{align*}
\end{frame}

  \begin{frame}
    \frametitle{Analyzing the second term}
Let $\alpha_n = \x_n^\top\P_{-n}\x_n$ and $\bar\alpha=n/{\bar\gamma}$
%    \begin{align*}
% \|\B\| 
% &= \big\|\E[\P_{-n}\x_n\x_n^\top(\gamma_n-\bar\gamma)]\bar\P\big\|\\
% &\preceq \big\|\E[|\gamma_n-\bar\gamma|\cdot\x_n\x_n^\top]\bar\P\big\|\\
% &=\Big\|\E\Big[\tfrac{|\bar\alpha-\alpha_n|}{\alpha_n}\cdot\bar\gamma\x_n\x_n^\top\Big]\bar\P\Big\|
%    \end{align*}
% Here, we have to deal with $1/\alpha_n$, which blows up when
% $\alpha_n\rightarrow 0$.\\[4mm]
% Alternatively, we can use
\begin{align*}
\hspace{-8mm}\|\B\| &=
\Bigg\|\E\bigg[\E\Big[\frac{\P_{-n}\x_n\x_n^\top}{\x_n^\top\P_{-n}\x_n}
(\bar\alpha-\alpha_n) \bar\gamma\mid\P_{-n}\Big]\bigg]\Bigg\|\\
  &\leq \E\Bigg[\bigg\|\E\Big[\frac{\P_{-n}\x_n\x_n^\top}{\x_n^\top\P_{-n}\x_n}
(\bar\alpha-\alpha_n)
    \bar\gamma\mid\P_{-n}\Big]\bigg\|\Bigg]\\
  &=\E\Bigg[\bigg\|\E\Big[\frac{\P_{-n}\x_n\x_n^\top\P_{-n}}{\x_n^\top\P_{-n}\x_n}
(\bar\alpha-\alpha_n)
    \bar\gamma\mid\P_{-n}\Big]\bigg\|\Bigg]\\
  &=\E\Bigg[\max_{\v:\,\|\v\|=1}\v^\top\E\bigg[\frac{\P_{-n}\x_n\x_n^\top\P_{-n}}{\x_n^\top\P_{-n}\x_n}
  (\bar\alpha-\alpha_n)
    \bar\gamma\mid\P_{-n}\bigg]\v\Bigg]
  \\
&\leq \bar\gamma\, \E\Bigg[\max_{\v:\,\|\v\|=1}\underbrace{\sqrt{\E\bigg[
\Big(\v^\top
       \frac{\P_{-n}\x_n\x_n^\top\P_{-n}}{\x_n^\top\P_{-n}\x_n}\v\Big)^2\mid\P_{-n}\bigg]}}_{B_1}
       \cdot\underbrace{\sqrt{\E\big[(\bar\alpha-\alpha_n)^2 \mid\P_{-n}\big]}}_{B_2}\Bigg]
\end{align*}
\end{frame}

\begin{frame}
  \frametitle{Bounding $B_1$}
  First we lower bound the nuclear rank of $\P_{-n}\Sigmab\P_{-n}$:
  \begin{align*}
    \frac{\tr(\P_{-n}\Sigmab\P_{-n})} {\|\P_{-n}\Sigmab\P_{-n}\|}\geq
    \tr(\P_{-n}\Sigmab) =
    \|\Sigmab^{\frac12}-\Sigmab^{\frac12}\X_{-n}^\dagger\X_{-n}\|_F^2\geq
    \overbrace{\sum_{i\geq n}\lambda_i}^{r_n}
  \end{align*}
  % Consider the eigenvector $\v$ of matrix $\P_{-n}\Sigmab\P_{-n}$
  % associated with the smallest non-zero eigenvalue. Since
  % $\P_{-n}\Sigmab\P_{-n}\v=\lambda\v\neq \zero$, vector $\v$ has to
  % lie in the subspace of $\P_{-n}$, so:
  % \begin{align*}
  % \lambda_{\min}(\P_{-n}\Sigmab\P_{-n})=\v^\top\P_{-n}\Sigmab\P_{-n}\v=\v^\top\Sigmab\v\geq \lambda_{\min}.
  % \end{align*}
Note that $r_n\geq (d-n)\lambda_{\min}$.  Let $\x \sim\Nc(\zero, \P_{-n}\Sigmab\P_{-n})$.\\
  Using the bound on  the rank one Gaussian projection, we get:
  \begin{align*}
B_1^2&=\E\bigg[\Big(\v^\top\tfrac{\P_{-n}\x_n\x_n^\top\P_{-n}}{\x_n^\top\P_{-n}\x_n}\v\Big)^2\mid
    \P_{-n}\bigg]
    =\E\bigg[\Big(\frac{\v^\top\x\x^\top\v}{\x^\top\x}\Big)^2\bigg] \\
     &\leq
      \frac{12(\v^\top\P_{-n}\Sigmab\P_{-n}\v)^2} {(r_n-2)^2} + 2\ee^{-c(r_n-2)}\|\v\|^4
  \end{align*}
  So, letting $\|\Sigmab\|=1$, we get $\max_{\v:\|\v\|=1}B_1= O(\frac1{r_n})$
\end{frame}

\begin{frame}
  \frametitle{Decomposing $B_2$}
Recall $\alpha_n = \x_n^\top\P_{-n}\x_n$. Let
$\bar\alpha=\tr(\E[\P_{-n}]\Sigmab)$ and
$\tilde\alpha=\tr(\P_{-n}\Sigmab)$.
\begin{align*}
\E[B_2]&=\E\Big[\sqrt{\E\big[(\bar\alpha-\tilde\alpha+\tilde\alpha-\alpha_n)^2
         \mid\P_{-n}\big]}\Big]\\
&=\E\Big[\sqrt{(\bar\alpha-\tilde\alpha)^2 +
                                    \E\big[(\tilde\alpha-\alpha_n)^2\mid\P_{-n}\big]}\Big]\\
&\leq \sqrt{\E\big[(\bar\alpha-\tilde\alpha)^2\big]
+ \E\big[(\tilde\alpha-\alpha_n)^2\big] },
\end{align*}
where the last step follows from Jensen's inequality.
\end{frame}

\begin{frame}
  \frametitle{Putting everything together}
  Burkholder's inequality implies that:
  \begin{align*}
    \E\big[(\bar\alpha-\tilde\alpha_n)^2\big]
    &= \E\big[\big(\tr\,\Sigmab(\E[\P_{-n}]-\P_{-n})\big)^2\big] \leq C\cdot n\, \|\Sigmab\|^2
  \end{align*}
  \vspace{5mm}
  
Bai-Silverstein lemma implies
that: 
  \begin{align*}
\E\big[(\tilde\alpha-\alpha_n)^2\mid\P_{-n}\big] \leq C\cdot
    \|\P_{-n}\Sigmab\|_F^2\leq C\cdot d\,\|\P_{-n}\Sigmab\|^2\leq
    C\cdot d\,\|\Sigmab\|^2.
  \end{align*}
  \vspace{5mm}
  
  Combining, we get:   $\E[B_2] \leq C\cdot \sqrt{n+d}\,\|\Sigmab\|$\\
  Putting everything together, assuming $\|\Sigmab\|=1$, we have:

  \begin{align*}
    \|\B\|=O\Big(\tfrac{\bar\gamma\sqrt{\!d}}{r_n}\Big)\overset{(*)}{=}
O\Big(\tfrac{\kappa\bar\gamma}{\sqrt{\!d}}\Big),
  \end{align*}
  where $(*)$ holds when $n\leq d/2$
\end{frame}

\begin{frame}
  \frametitle{Next steps}
  % Note that $ \E[\|\P_{-n}\Sigmab\|_F^2] =
  % \tr(\E[\P_{-n}]\Sigmab^2)\leq   \tr(\E[\P_{-n}]\Sigmab) $.

  % Moreover,
  We would like to replace $r_n$ with:
  \begin{align*}
\tr(\P_{-n}\Sigmab)\simeq
\tr(\bar\P\Sigmab) =n/\bar\gamma \geq r-n,
  \end{align*}
  where $r=\frac{\tr(\Sigmab)}{\|\Sigmab\|}$ is the nuclear
  rank. Overall, our goal is to show that:
  \begin{align*}
    \big\|\E[\P]\bar\P^{-1}-\I\big\| \leq
    C\cdot\frac{\bar\gamma}{\sqrt{n}}
\leq
    \begin{cases}
      \frac C{\sqrt r} &\text{for }n\leq r/2,\\
      \frac {C\kappa}{\sqrt d} &\text{for }n\leq d/2,
    \end{cases}  
  \end{align*}
  where $C$ is an \underline{absolute} constant, and $\kappa$ is the
  condition number of $\Sigmab$.
\end{frame}

\begin{frame}
  \frametitle{New strategy for A}
  Assume that $n< r$. Then, using $\bar\gamma\leq \frac{n}{r-n}$
  and $\|\P_{-n}\Sigmab\|\leq 1$, we have
\begin{align*}
   \|\A\| & =
          \Bigg\|\E\bigg[\frac{\P_{-n}\Sigmab}{\tr\,\P_{-n}\Sigmab}
           \P_{-n}\bar\gamma\Sigmab\bigg]\Bigg\|
  \leq \bar\gamma\,\E\bigg[\frac{\|\P_{-n}\Sigmab\|}{\tr\,\P_{-n}\Sigmab}\bigg]\\
  &\leq \frac{n}{r-n}\,\E\bigg[\frac{\|\P_{-n}\Sigmab\|}{\tr\,\P_{-n}\Sigmab}\bigg]
\end{align*}
Let $s=\tr\,\E[\P_{-n}]\Sigmab$. Burkholder gives:
$\Pr\{\tr\,\P_{-n}\Sigmab\leq  \frac12 s\}\leq\ee^{-\frac{cs}{\sqrt n}}$, so
\begin{align*}
  \E\bigg[\frac{\|\P_{-n}\Sigmab\|}{\tr\,\P_{-n}\Sigmab}\bigg]
  &\leq
    \E\bigg[\frac{\|\P_{-n}\Sigmab\|}{\tr\,\P_{-n}\Sigmab}\mid
    \tr\,\P_{-n}\Sigmab\geq\frac12s\bigg]
    +\Pr\big\{\tr\,\P_{-n}\Sigmab\leq \frac12 s\big\}\\
  &\leq \frac2{s}+\ee^{-\frac{cs}{\sqrt n}} = O(\sqrt n/s)
\end{align*}
Then, we close the loop to replace $s$ with $\tr\,\bar\P\Sigmab$.
\end{frame}

\begin{frame}
  \frametitle{Concentration using Burkholder}
  Burkholder inequality says that for $k\geq 2$,
  \begin{align*}
    \E\big[(s - \tr\,\P_{-n}\Sigmab)^k\big]^{\frac1k}\leq Ck\,\sqrt{n}
  \end{align*}
  So, Markov's inequality says that:
  \begin{align*}
    \Pr\big\{|s - \tr\,\P_{-n}\Sigmab|\geq \alpha\big\}\leq
    \Big(\frac{Ck\sqrt n}{\alpha}\Big)^k
  \end{align*}
  Setting $k=\frac {\alpha}{\ee C\sqrt n}$ we get:
  \begin{align*}
 \Pr\big\{|s - \tr\,\P_{-n}\Sigmab|\geq \alpha\big\}\leq
    \Big(\frac1e\Big)^k\leq \ee^{-\frac{\alpha}{C\sqrt n}}
    \end{align*}
  \end{frame}

  \begin{frame}
    \frametitle{Bounding the trace}
Let $\alpha_n = \x_n^\top\P_{-n}\x_n$, $\hat\alpha =\E[\alpha_n]=
\tr\,\E[\P_{-n}]\Sigmab$ and $\bar \alpha=\tr\,\bar\P
\Sigmab$. \\
Suppose that $\bar\alpha - \hat\alpha>0$. Since $\|\Sigmab\|=1$ and
$\|\bar\gamma\Sigmab\bar\P\|=\frac{\bar\gamma}{\bar\gamma+1}$, we have:
    \begin{align*}
      \bar\alpha - \hat\alpha &\leq \tr\,\bar\P\Sigmab - \tr\,\E[\P]\Sigmab
      = \tr\,(\I-\E[\P]\bar\P^{-1})\bar\P\Sigmab \\
      &=\tr\, \E[\P_{-n}-\P]\bar\gamma\Sigmab\bar\P\Sigmab +
        \tr\,\E\big[\P_{-n}\x_n\x_n^\top(\gamma_n-\bar\gamma)\big]\bar\P\Sigmab
      \\
      &\leq \tfrac{\bar\gamma}{\bar\gamma+1}\,
        \E\bigg[\tr\,\frac{\P_{-n}\x_n\x_n^\top\P_{-n}}{\x_n^\top\P_{-n}\x_n}\bigg]
        +\tfrac{\bar\gamma}{\bar\gamma+1}\,
        \E\bigg[\tr\,\frac{\P_{-n}\x_n\x_n^\top\P_{-n}}{\x_n^\top\P_{-n}\x_n}
        \cdot|\bar\alpha-\alpha_n|\bigg]\\
&=\tfrac{\bar\gamma}{\bar\gamma+1}\Big(1 + \E\big[|\bar\alpha - \alpha_n|\big]\Big)\\
      &=\tfrac{\bar\gamma}{\bar\gamma+1}\Big(1+ (\bar\alpha - \hat\alpha) +\E\big[|\hat\alpha-\alpha_n|\big]\Big)
    \end{align*}
    From Bei-Silverstein, we have
    $\E\big[|\hat\alpha-\alpha_n|\big]\leq
    C\sqrt{\E[\|\P_{-n}\Sigmab\|_F^2]}\leq C\sqrt{\hat\alpha}$,\\
    so, closing the loop and assuming that $\bar\alpha\geq 1$, we get:
    \begin{align*}
      \tfrac1{\bar\gamma+1}(\bar\alpha-\hat\alpha)\leq
      \tfrac{\bar\gamma} {\bar\gamma+1}
      (1+C\sqrt{\hat\alpha})\leq\tfrac1{\bar\gamma+1}\tfrac{C'\bar\gamma}
      {\sqrt{\bar\alpha}}\cdot\bar\alpha. 
    \end{align*}
    We conclude that $\hat\alpha\geq (1-\frac{C'\bar\gamma}{\sqrt{\bar\alpha}})\bar\alpha$.
  \end{frame}

  \begin{frame}
    \frametitle{Rank-one bound with $y$}
Let $\z\sim\Nc(\zero,\I)$ and $(\z,y)$ are jointly
distributed. \\
Let $\lambda_i$ be the eigenvalues of $\Sigmab$.
\begin{align*}
  \E\Big[\frac{\|\Sigmab^{\frac12}\z y\|^2}{(\z^\top\Sigmab\z)^2}\Big]
  &=\sum_i
   \E\bigg[\frac{\lambda_iz_i^2y^2}{(\sum_j\lambda z_j^2)^2}\bigg]\\
 &\leq \sqrt{\E\bigg[\frac1{(\sum_i\lambda_iz_i^2)^4}\bigg]}\cdot\sum_i\sqrt{\E[\lambda_i^2 z_i^4y^4]}
\end{align*}
\end{frame}

\begin{frame}
  \frametitle{Bounding inverse moments}
  Let $Z_S=\sum_{i\in S}\lambda_iz_i^2$, $r=\sum_i\lambda_i$
  and  $1=\lambda_1\geq ...\geq\lambda_d$. Fix $a=2k+1$
  \begin{align*}
    \E\bigg[\frac1{(\sum_i\lambda_iz_i^2)^k}\bigg]
    &=\E\Bigg[\E\bigg[
      \frac1{(Z_{\leq a}  + Z_{>a})^k}\mid  Z_{\leq a}\bigg]\Bigg]\\
    &\leq \frac1{(\frac12\E[Z_{>a}])^k} +
      \ee^{-c(r-a)}\E\bigg[\frac1{(Z_{\leq a})^k}\bigg]\\
    &\leq \Big(\frac 2{r-a}\Big)^k +
      \ee^{-c(r-a)}\Big(\frac{1}{\lambda_{a}}\Big)^k\E\bigg[\frac1{(\sum_{i=1}^{a}z_i^2)^k}\bigg]\\
    &=\Big(\frac 2{r-a}\Big)^k +
      \ee^{-c(r-a)}\Big(\frac{1}{\lambda_{a}}\Big)^k\frac1{(a-2)!!},
  \end{align*}
  where we used $k$th inverse moment of the chi-squared
  distribution.\\
  Note that $r-a\leq (d-a)\lambda_a$, so if we assume
  that $r\geq C(k+\ln d)$ then:
  \begin{align*}
    \Bigg(\E\bigg[\frac1{(\sum_i\lambda_iz_i^2)^k}\bigg]\Bigg)^{\!1/k}=
    O\Big(\frac1r\Big).
  \end{align*}
  
\end{frame}

\begin{frame}
  \frametitle{Inverse moments for general sub-Gaussians}
  Let $z_1,...,z_d$ be i.i.d. sub-Gaussian and let $\A$ be
  p.s.d.~with r = $\frac{\tr\,\A}{\|\A\|}$.\\ 
There is an event $E$ such that $\Pr(E)\geq
1-\ee^{-c\cdot r}$ and for any $k$:
\begin{align*}
  \Bigg(\E\bigg[\frac1{(\z^\top\A\z)^k}\mid E\bigg]\Bigg)^{1/k}
  =O(1/r). 
\end{align*}
In particular, if $z_i$ is a Rademacher, the condition number of $\A$ is
$\kappa<\infty$ and $r\geq c\ln\kappa$, then
\begin{align*}
  \Bigg(\E\bigg[\frac1{(\z^\top\A\z)^k}\bigg]\Bigg)^{1/k}
  = O(1/r).
\end{align*}
Dependence on condition number (or failure event) is inevitable in general, because if
$z_i$ is a Rademacher and $\one_d$ is an eigenvector of $\A$
corresponding to $\lambda_{\min}$, then
\begin{align*}
    \Bigg(\E\bigg[\frac1{(\z^\top\A\z)^k}\bigg]\Bigg)^{1/k}\geq
  \frac1{\one_d^\top\A\one_d}\cdot \Pr(\z = \one_d) =
  \frac1{\lambda_{\min}d}\cdot 2^{-d}.
\end{align*}
\end{frame}

\begin{frame}
  \frametitle{Rank-one projection bound for sub-Gaussians}
  Assume that $z_i$ are i.i.d.~sub-Gaussian with mean zero and variance 1.\\
  Let $E$ denote the event that
  $\z^\top\A\z\geq\frac12\tr\,\A$. Hanson-Wright says:
  \begin{align*}
\Pr(\neg E)\leq 2\exp\Big(-c\min\Big\{\frac{(\tr\,\A)^2}{\tr\,\A^2},\frac{\tr\,\A}{\|\A\|}\Big\}\Big).
  \end{align*}  
Note that $\frac{(\tr\,\A)^2}{\tr\,\A^2}\geq
\frac{\tr\,\A}{\|\A\|}=r$, so $\Pr(\neg E)\leq 2\ee^{-c\cdot r}$ and:
  \begin{align*}
    \E\bigg[\frac{\A^{\frac12}\z\z^\top\A^{\frac12}}{\z^\top\A\z}\bigg]
    &\preceq \frac2{\tr\,\A} \E\big[\A^{\frac12}\z\z^\top\A^{\frac12}\mid E]\Pr(E)      + 
      \E\bigg[\frac{\A^{\frac12}\z\z^\top\A^{\frac12}}{\z^\top\A\z}\mid \neg E\bigg]\Pr(\neg E)\\
    &\preceq \frac 2{\tr\,\A}\E[\A^{\frac12}\z\z^\top \A^{\frac12}] +
\E\Big[\frac{\z^\top\A\z\cdot \I}{\z^\top\A\z}\mid
      \neg E\Big]2\ee^{-c\cdot r}\\
    &= \frac {2}{\tr\,\A}\cdot \A + 2\ee^{-c\cdot r}\cdot \I =
      O(1/r)\cdot \I.
  \end{align*}
\end{frame}

\begin{frame}
  \frametitle{Rank-one projection for sub-Gaussians: precise analysis}
  We apply Hanson-Wright more carefully:
  \begin{align*}
    \hspace{-6mm}\Pr\Big(|\z^\top\A\z-\tr\,\A|\geq \epsilon\,\tr\,\A\Big)\leq 
    2\exp\Big(-c\min\Big\{\frac{\epsilon^2(\tr\,\A)^2}{\tr\,\A^2},
    \frac{\epsilon\,\tr\,\A}{\|\A\|}\Big\}\Big)
    \leq 2\ee^{-c\epsilon^2 r}.
  \end{align*}
  Using an analogous argument as in the previous slide, we get:
  \begin{align*}
    \Bigg\|\E\bigg[\frac{\A^{\frac12}\z\z^\top\A^{\frac12}}{\z^\top\A\z}\bigg]
    -\frac{\A}{\tr\,\A}\Bigg\|
    \leq\frac\epsilon r +2\ee^{-c\epsilon^2 r}.
  \end{align*}
  Setting $\epsilon =C\cdot \sqrt{\frac{\ln r}{r}}$ with appropriate $C$, we get:
  \begin{align*}
   \Bigg\|\E\bigg[\frac{\A^{\frac12}\z\z^\top\A^{\frac12}}{\z^\top\A\z}\bigg]
    -\frac{\A}{\tr\,\A}\Bigg\|\leq C \sqrt{\frac{\ln r}{r}}\cdot \bigg\|\frac{\A}{\tr\,\A}\bigg\|
    \end{align*}
  \end{frame}

  \begin{frame}
    \frametitle{Rank-one oblique projection}
Set $\P_\x=\frac{\A\x\x^\top}{\x^\top\A\x}$,
$\epsilon_\x=\frac{\tr\,\A\Sigmab}{\x^\top\A\x}-1$, and
$E=\big[|\epsilon_\x|\leq \epsilon\big]$, with $\delta=\Pr(\neg E)$
    \begin{align*}
\hspace{-4mm}\Big\|\E\big[\P_\x\mid E\big]\Big\|
&\leq\bigg\|\E\big[\P_\x\mid E\big] -
        \E\Big[\frac{\A\x\x^\top}{\tr\,\A\Sigmab}\mid E\Big]\bigg\|
        + \bigg\|\E\Big[\frac{\A\x\x^\top}{\tr\,\A\Sigmab}\mid E\Big]\Bigg\|\\
      &=\bigg\|\frac{\A}{\tr\,\A\Sigmab}
        \E\big[\epsilon_\x\x\x^\top\mid E\big]\bigg\|
        +\bigg\|\frac{\A}{\tr\,\A\Sigmab}\E\big[\x\x^\top\mid E\big]\bigg\|
    \end{align*}
    Note that we have $\E[\x\x^\top\mid E]\preceq \big(1+
    \frac{\delta}{1-\delta}\big)\cdot\Sigmab$ and
    \begin{align*}
      \big\|\E\big[\epsilon_\x\x\x^\top\mid E\big]\big\| \leq
      \big\|\E\big[\epsilon_\x^+\x\x^\top\mid E\big]\big\|+
      \big\|\E\big[\epsilon_\x^-\x\x^\top\mid E\big]\big\|\leq
      2\epsilon\cdot \|\E[\x\x^\top\mid E]\|
    \end{align*}
    Overall, setting $\epsilon=C\cdot \sqrt{\frac{\ln r}{r}}$ we get
    $\Pr(\neg E)\leq \epsilon$ and moreover:
    \begin{align*}
      \Big\|\E\big[\P_\z\mid E\big]\Big\|\leq
      \big(1+O(\epsilon)\big)\cdot\frac{\|\A\|\,\|\Sigmab\|}{\tr\,\A\Sigmab}.
      \end{align*}
  \end{frame}


  \begin{frame}
    \frametitle{A new analysis of Theorem 1: high probability event}
Let $\X = \Z\Sigmab^{\frac12}$, with $\|\Sigmab\|=1$ and
$r=\tr\,\Sigmab$. $\Z$ has i.i.d.~sub-Gaussian rows\\
    Let $E_i =
    \Big[\Big|\frac{\tr\,\P_{-i}\Sigmab}{\x_i^\top\P_{-i}\x_i}-1\Big|\leq
\epsilon\Big]$ and $F_i= \big[\x_i^\top\x_i\geq r/2\big]$ be
    events for $i\in[n]$.\\
    Hanson-Wright for the event $F_i$ gives: $\Pr(\neg F_i)\leq
    \ee^{-c r}$.\\
Denoting
$r_i=\frac{\tr\,\Sigmab^{\frac12}\P_{-i}\Sigmab^{\frac12}}{\|\Sigmab^{\frac12}\P_{-i}\Sigmab^{\frac12}\|}\geq
\tr\,\P_{-i}\Sigmab\geq r-n$, 
we have:
\begin{align*}
\Pr(\neg E_i) = \E\big[\Pr(\neg E_i\mid \P_{-i})\big]\leq
  \E\big[\ee^{-c\epsilon^2r_i}\big]\leq \ee^{-c\epsilon^2(r-n)}
\end{align*}
Now, we let $E = \bigwedge_{i=1}^n(E_i \wedge F_i)$ and union bound
implies that:
\begin{align*}
  \Pr(\neg E)\leq 2n\,\ee^{-c\epsilon^2(r-n)}.
\end{align*}
Let $\rho = r/n$. Note that setting $\epsilon = C_\rho\cdot
\sqrt{\frac{\ln r}{r}}$, we get $\Pr(\neg E)\leq \epsilon/r$
  \end{frame}
  
  \begin{frame}
    \frametitle{A new analysis of Theorem 1: decomposition}
Recall that $\P=\I-\X^\dagger\X$ and $\bar\P=(\bar\gamma\Sigmab+\I)^{-1}$
\begin{align*}
\hspace{-5mm}\I-\E[\P]\bar\P^{-1}
  &= \E[\X^\dagger\X] - \bar\gamma\E[\P]\Sigmab\\
  &= \E[\X^\dagger\X\mid E]\Pr(E) +
    \E[\X^\dagger\X\mid \neg E]\Pr(\neg E)
    -\bar\gamma\E[\P]\Sigmab\\
  &=n\,\E\bigg[\frac{\P_{-n}\x_n\x_n^\top}{\x_n^\top\P_{-n}\x_n}\mid
    E\bigg]\Pr(E)-\bar\gamma\E[\P]\Sigmab \ +\     \E[\X^\dagger\X\mid \neg
    E]\Pr(\neg E)\\
  &=n\,\E\bigg[\frac{\P_{-n}\Sigmab}{\tr\,\P_{-n}\Sigmab}\bigg] +
    n\bigg(\E\bigg[\frac{\P_{-n}\x_n\x_n^\top}{\x_n^\top\P_{-n}\x_n}\one_E-\frac{\P_{-n}\Sigmab}{\tr\,\P_{-n}\Sigmab}\bigg]\bigg)
  \\
  &\quad-\bar\gamma\E[\P]\Sigmab \ +\  \E[\X^\dagger\X\mid \neg E]\Pr(\neg E)
\end{align*}
Note that $\big\|\E[\X^\dagger\X\mid \neg E]\Pr(\neg
E)\big\|\leq \Pr(\neg E)\leq \epsilon$. Next, we have:
\begin{align*}
\hspace{-8mm}n\,\bigg\|\E\bigg[\frac{\P_{-n}\x_n\x_n^\top}{\x_n^\top\P_{-n}\x_n}\one_E
-\frac{\P_{-n}\Sigmab}{\tr\,\P_{-n}\Sigmab}\bigg]\bigg\|
&\leq n\,\E\Bigg[\bigg\|\E\bigg[\frac{\P_{-n}\x_n\x_n^\top}{\x_n^\top\P_{-n}\x_n}\one_E\mid \P_{-n}\bigg] -\frac{\P_{-n}\Sigmab}{\tr\,\P_{-n}\Sigmab}\bigg\|\Bigg]\\
&\overset{(*)}{\leq} n \, \E\bigg[\frac{O(\epsilon)}{\tr\,\P_{-n}\Sigmab}\bigg]
\leq O(\epsilon)\cdot \frac{n}{r-n} \leq C_\rho' \cdot \sqrt{\frac{\ln r}{r}},
\end{align*}
where $(*)$ is shown on the next slide.
\end{frame}

  \begin{frame}
    \frametitle{A new analysis of Theorem 1: oblique projection}
let $\hat\C=\frac{\P_{-n}\x_n\x_n^\top}{\x_n^\top\P_{-n}\x_n}$, $\C=\frac{\P_{-n}\Sigmab}{\tr\,\P_{-n}\Sigmab}$ and
$\varepsilon=\frac{\tr\,\P_{-n}\Sigmab}{\x_n^\top\P_{-n}\x_n}-1$.
Condition on $\P_{-n}$. 
    \begin{align*}
\hspace{-5mm}\big\|\E\big[\hat\C \cdot \one _E\big]-\C\big\|
&\leq\bigg\|\E\Big[\hat\C \cdot \one _E -
\frac{\P_{-n}\x_n\x_n^\top}{\tr\,\P_{-n}\Sigmab} \one _E\Big]\bigg\|
        + \bigg\|\E\Big[\frac{\P_{-n}\x_n\x_n^\top}{\tr\,\P_{-n}\Sigmab} \one_E\Big]
-\C\Bigg\|\\ 
      &=\bigg\|\frac{\P_{-n}}{\tr\,\P_{-n}\Sigmab}
        \E\big[\varepsilon\x_n\x_n^\top\one_E\big]\bigg\|
        +\bigg\|\frac{\P_{-n}}{\tr\,\P_{-n}\Sigmab}\Big(\E\big[\x_n\x_n^\top\one_E\big]-\Sigmab\Big)\bigg\|
    \end{align*}
Denoting $\delta = \Pr(\neg E)\leq \epsilon/r$, we have:
\begin{align*}
\Sigmab = \E[\x_n\x_n^\top] \succeq  \E[\x_n\x_n^\top\one_E]
  &= \Sigmab - \delta\,\E[\x_n\x_n^\top\mid\neg E]\\
  &\succeq \Sigmab - \delta\,
    \E[\x_n^\top\x_n\,\I\mid\neg E]
    \succeq \Sigmab  - \frac{\delta r}{2}\I\succeq \Sigmab - \epsilon\,\I,
\end{align*}
so $\|\E[\x_n\x_n\one_E]-\Sigmab\|\leq \epsilon$ and moreover it
follows that:
    \begin{align*}
      \hspace{-4mm}
      \big\|\E\big[\varepsilon\x_n\x_n^\top\one_E\big]\big\| \leq
      \big\|\E\big[\varepsilon^+\x_n\x_n^\top\one_E\big]\big\|+
      \big\|\E\big[\varepsilon^-\x_n\x_n^\top\one_E\big]\big\|\leq
      2\epsilon\cdot \|\E[\x_n\x_n^\top\one_E]\|\leq 2\epsilon.
    \end{align*}
We conclude that $\|\E[\hat\C\cdot\one_E]-\C\|\leq O(\epsilon)\cdot
\frac{\|\P_{-n}\|}{\tr\,\P_{-n}\Sigmab}=\frac{O(\epsilon)}{\tr\,\P_{-n}\Sigmab}$. 
  \end{frame}



\begin{frame}
  \frametitle{A new analysis of Theorem 1: returning to the old proof}
  In the previous slides we showed that:
  \begin{align*}
    \|\I-\E[\P]\bar\P^{-1}\|\leq
    \bigg\|n\,\E\bigg[\frac{\P_{-n}\Sigmab}{\tr\,\P_{-n}\Sigmab}\bigg]
    - \bar\gamma\E[\P]\Sigmab\bigg\| + C_\rho\cdot \sqrt{\frac{\ln r}{r}}
  \end{align*}
  It remains to use the old analysis, with $\bar s=\tr\bar\P\Sigmab$
  and $\tilde s=\tr\,\P_{-n}\Sigmab$.
  \begin{align*}
   \bigg\| n\,\E\bigg[\frac{\P_{-n}\Sigmab}{\tr\,\P_{-n}\Sigmab}\bigg]
    - \bar\gamma\E[\P]\Sigmab\bigg\|
&=\bar\gamma\bigg\|\E\bigg[(\bar s - \tilde s)\cdot
\frac{\P_{-n}\Sigmab}{\tr\,\P_{-n}\Sigmab}\bigg]
+ \E[\P_{-n}-\P]\Sigmab\bigg\|\\
&\leq \bar\gamma \E\bigg[|\bar s-\tilde s|\cdot
\frac{\|\P_{-n}\Sigmab\|}{\tr\,\P_{-n}\Sigmab}\bigg]
+\bar\gamma\bigg\|\E\bigg[\frac{\P_{-n}\x_n\x_n\P_{-n}}{\x_n^\top\P_{-n}\x_n}\bigg]\bigg\|
    \\
    &\leq \bar\gamma\,\Big(\E\big[|\bar s-\tilde s|\big] + 2\Big)\cdot
      \frac1{r-n}\leq C_\rho\cdot \frac1{\sqrt r}
  \end{align*}
\end{frame}

  \begin{frame}
    \frametitle{Eliminating the $\ln(r)$: high probability event}
Let $\X = \Z\Sigmab^{\frac12}$, with $\|\Sigmab\|=1$ and
$r=\tr\,\Sigmab$. $\Z$ has i.i.d.~sub-Gaussian rows\\
    Let $E_i =
    \Big[\Big|\frac{\tr\,\P_{-i}\Sigmab}{\x_i^\top\P_{-i}\x_i}-1\Big|\leq\frac12\Big]$ and $F_i= \big[\frac{\tr\Sigmab}{\x_i^\top\x_i}\leq 2\big]$ be
    events for $i\in[n]$.\\
    Hanson-Wright for the event $F_i$ gives: $\Pr(\neg F_i)\leq
    \ee^{-c r}$.\\
Denoting
$r_i=\frac{\tr\,\Sigmab^{\frac12}\P_{-i}\Sigmab^{\frac12}}{\|\Sigmab^{\frac12}\P_{-i}\Sigmab^{\frac12}\|}\geq
\tr\,\P_{-i}\Sigmab\geq r-n$, 
we have:
\begin{align*}
\Pr(\neg E_i) = \E\big[\Pr(\neg E_i\mid \P_{-i})\big]\leq
  \E\big[\ee^{-cr_i}\big]\leq \ee^{-c(r-n)}
\end{align*}
Now, we let $E = \bigwedge_{i=1}^n(E_i \wedge F_i)$ and a union bound
implies that:
\begin{align*}
  \Pr(\neg E)\leq 2n\,\ee^{-c(r-n)}\leq  \frac{C_\rho}{\sqrt{r}}
\end{align*}
where $\rho = r/n>1$, assuming that $r\geq C_\rho$ for appropriately
chosen $C_\rho$.
  \end{frame}


\begin{frame}
  \frametitle{Eliminating $\ln(r)$: decomposition}
  Recall that $\P=\I-\X^\dagger\X$ and
  $\bar\P=(\bar\gamma\Sigmab+\I)^{-1}$ and $\hat s =
  \x_n^\top\P_{-n}\x_n$, $\bar s=n/\bar\gamma$
\begin{align*}
\hspace{-5mm}\I&-\E[\P]\bar\P^{-1}
  = \E[\X^\dagger\X] - \bar\gamma\E[\P]\Sigmab\\
  &= \E[\X^\dagger\X\one_E] +
    \E[\X^\dagger\X \one_{\neg E}]
    -\bar\gamma\E[\P]\Sigmab\\
  &=n\,\E\bigg[\frac{\P_{-n}\x_n\x_n^\top}{\x_n^\top\P_{-n}\x_n}\one_E\bigg]
    -\bar\gamma\E[\P]\Sigmab \ +\     \E[\X^\dagger\X\one_{\neg E}]\\
  &=\bar\gamma\,\underbrace{\E\bigg[(\bar s-\hat
    s)\cdot\frac{\P_{-n}\x_n\x_n^\top}{\x_n^\top\P_{-n}\x_n}\one_E\bigg]}_{\T}
  + \bar\gamma\E[\P_{-n}\x_n\x_n^\top\one_E] -\bar\gamma\E[\P_{-n}]\Sigmab \\[-4mm]
  &\hspace{5cm}+ \E[\P_{-n}-\P]\bar\gamma\Sigmab +\E[\X^\dagger\X\one_{\neg E}]
\end{align*}
Note that $\|\E[\X^\dagger\X\one_{\neg E}]\|\leq \frac{C_\rho}{\sqrt{r}}$ and $\|\E[\x_n\x_n^\top\one_E\mid\P_{-n}]-\Sigmab\|\leq
\frac{C_\rho}{\sqrt{r}}$ so:
\begin{align*}
\big\|\E[\P_{-n}]\Sigmab -
\E[\P_{-n}\x_n\x_n^\top\one_E]\big\|
  =\Big\|\E\Big[\P_{-n}\big(\Sigmab-\E[\x_n\x_n^\top\one_E\mid\P_{-n}]\big)\Big]\Big\|\leq \frac{C_\rho}{\sqrt{r}}
\end{align*}
\textbf{Question:} How do we bound $\|\T\|$?
\end{frame}

\begin{frame}
  \frametitle{Eliminating $\ln(r)$: last term}
\begin{align*}
\hspace{-8mm}   
\|\T\|    &= \bigg\|\E\bigg[\P_{-n}\E\Big[(\bar s-\hat s)\cdot
            \frac{\x_n\x_n^\top}{\x_n^\top\P_{-n}\x_n}\one_E\mid\P_{-n}\Big]\bigg]\bigg\|
  \\
  &\leq\E\bigg[\|\P_{-n}\|\cdot \Big\|\E\Big[|\bar s-\hat s|\cdot
    \frac{\x_n\x_n^\top}{\x_n^\top\P_{-n}\x_n}\one_E\mid\P_{-n}\Big]
    \Big\|\bigg]
  \\
          &\leq \frac2{r-n} \cdot \E\bigg[\sup_{\v\in S^{d-1}}\E\Big[
            |\bar s-\hat s|\cdot \v^\top\x_n\x_n^\top\v\cdot\one_E\mid
            \P_{-n}\Big]\bigg]
  \\
  &\leq \frac{C_\rho}{r}\cdot \E\bigg[\underbrace{\sqrt{\E\big[(\bar s-\hat
    s)^2\mid\P_{-n}\big]}}_{O(\sqrt r)}\cdot
    \underbrace{\sup_{\v\in S^{d-1}}\sqrt{\E\big[(\v^\top\x_n)^4\mid \P_{-n}\big]}}_{O(1)}\bigg]
   \end{align*}
 \end{frame}

 
 \begin{frame}
   \frametitle{A new analysis of Theorem 2: decomposition}
   Define $\Q = \X^\dagger \X$ and
   $\bar\Q=\bar\gamma\Sigmab(\bar\gamma\Sigmab+\I)^{-1}$. We use
   $\E_E$ to mean $\E[\cdot\mid E]$
   \begin{align*}
     \bar\Q^{-\frac12}\E_E[\Q]\bar\Q^{-\frac12} - \I
     &=\bar\Q^{-\frac12}(\E_E[\Q]\bar\Q^{-1} - \I)\bar\Q^{\frac12}
     \\
     &=\bar\Q^{-\frac12}(\I-\E_E[\P]\bar\P^{-1})\bar\gamma^{-1}\Sigmab^{-1}\bar\Q^{\frac12}
     \\
     &=\bar\P^{-\frac12}(\bar\gamma\Sigmab)^{-\frac12}(\I-\E_E[\P]\bar\P^{-1})(\bar\gamma\Sigmab)^{-\frac12}\bar\P^{\frac12}
     \\
     &=\bar\P^{-\frac12}\Sigmab^{-\frac12}
       (\T_1 + \T_2 + \T_3) \Sigmab^{-\frac12}\bar\P^{\frac12},
   \end{align*}
   where we have:
   \begin{align*}
     \T_1
     &= \E_E\bigg[(\bar s - \hat s)\cdot
            \frac{\P_{-n}\x_n\x_n^\top}{\x_n^\top\P_{-n}\x_n}\bigg],
     \\
     \T_2 
     &=\E_E[\P_{-n}\x_n\x_n^\top] - \E_E[\P_{-n}]\Sigmab,
     \\
     \T_3
     &=\E_E[\P_{-n} - \P]\Sigmab.
   \end{align*}
 \end{frame}



 \begin{frame}
   \frametitle{A new analysis of Theorem 2: analyzing $\T_1$}
   \begin{align*}
      \|\Sigmab^{-\frac12}\T_1 \Sigmab^{-\frac12}\| \leq
     \underbrace{\E_E\Big[\big\|\Sigmab^{-\frac12}\P_{-n}\Sigmab^{\frac12}\big\|\Big]}_{=O(\sqrt{d/r}\,) \ ???}
     \cdot
     \underbrace{\E_E\bigg[\Big\|\E_E\Big[(\bar s- \hat s)\cdot
     \frac{\z_n\z_n^\top}{\x_n^\top\P_{-n}\x_n}\mid
     \P_{-n}\Big]\Big\|\bigg]}_{=O(1/\sqrt r\,)} 
   \end{align*}
   Note that we have:
   \begin{align*}
     \big\|\Sigmab^{-\frac12}\P_{-n}\Sigmab^{\frac12}\big\|
     &= \big\|\I -
       \Sigmab^{-\frac12}\X_{-n}^\dagger\X_{-n}\Sigmab^{\frac12}\big\|
     \\
     &\leq 1 +
       \big\|\Sigmab^{-\frac12}\Sigmab^{\frac12}\Z_{-n}^\top(\X_{-n}\X_{-n}^\top)^{\dagger}\Z_{-n}\Sigmab\big\|
     \\
     &= 1 +
       \big\|\Z_{-n}^\top(\X_{-n}\X_{-n}^\top)^{\dagger}\Z_{-n}\Sigmab\big\|
   \end{align*}
Furthermore, we can show that:
\begin{align*}
  \X_{-n}\X_{-n}^\top\simeq r\cdot \I\
  \quad\text{and}\quad
  \Z_{-n}\Z_{-n}^\top \simeq d\cdot\I,
\end{align*}
so we get:
\begin{align*}
  \hspace{-5mm}\big\|\Z_{-n}^\top(\X_{-n}\X_{-n}^\top)^{\dagger}\Z_{-n}\Sigmab\big\|
  &\simeq \tfrac1{r}  \big\|\Z_{-n}^\top\Z_{-n}\Sigmab\big\|
= \tfrac1r \sqrt{\|\Sigmab
    \Z_{-n}^\top\Z_{-n}\Z_{-n}^\top\Z_{-n}\Sigmab\|}
  \\
  &\simeq \tfrac{\sqrt
    d}{r}\sqrt{\|\Sigmab\Z_{-n}^\top\Z_{-n}\Sigmab\|}
    \leq
    \tfrac{\sqrt d}{r}\sqrt{\|\X_{-n}\X_{-n}^\top\|}
    =
    O\big(\sqrt{d/r}\,\big).
  \end{align*}
\end{frame}

\begin{frame}
  \frametitle{Net argument}
  Let $\b\in\R^k$ be a unit vector and 
  $\Z\in \R^{k\times d}$ have
  i.i.d.~$\Nc(0,1)$ entries. Then $\Z^\top\b\sim \Nc(\zero,\I_d)$,
so letting $r=\tr\,\Sigmab$, Hanson-Wright says:
  \begin{align*}
\hspace{-6mm}\Pr\Big(|\b^\top\Z\Sigmab\Z^\top\b-r|\geq
    \epsilon\,r\Big)\leq  
    2\exp\Big(-c\min\Big\{\frac{\epsilon^2(\tr\,\Sigmab)^2}{\tr\,\Sigmab^2},
    \frac{\epsilon\,\tr\,\Sigmab}{\|\Sigmab\|}\Big\}\Big)
    \leq 2\ee^{-c\epsilon^2 r}
  \end{align*}
  Setting $\epsilon = \sqrt{\frac kr} + \frac t{\sqrt r}$, we have with
  probability $\ee^{-c(k+t^2)}$:
  \begin{align*}
    \Big|\frac1r\|\Sigmab^{\frac12}\Z^\top\b\|^2-1\Big|\leq \sqrt{\frac kr} + \frac t{\sqrt r},
  \end{align*}
Using a net argument over unit vectors $\b$, we conclude that $\X=\Z\Sigmab^{\frac12}$ with
probability $1-\ee^{-t^2}$ satisfies:
\begin{align*}
  \sqrt r - C(\sqrt k +t)\leq \sigma_{\min}(\X)\leq
  \sigma_{\max}(\X)\leq\sqrt r+C(\sqrt k+t)
\end{align*}
\end{frame}

\begin{frame}{New strategy}
Note that for any matrices $\A, \B$ of the same dimension, we have that $\A\B$ and $\B \A$ share the same eigenvalue and we denote $\A\B \sim \B\A$. As such
\[
  \bar \Q^{-\frac12} \E_E[\Q] \bar \Q^{-\frac12} - \I \sim (\T_1 + \T_2 + \T_3) \Sigmab^{-1}
\]
so that $\| \bar \Q^{-\frac12} \E_E[\Q] \bar \Q^{-\frac12} - \I \| \le \sum_{k=1}^3 \| \T_k \Sigmab^{-1} \|$ (eigenvalue bounded by spectral norm). Since 
\begin{align*}
  \| \T_1 \Sigmab^{-1} \| & = \left\| \E_E \left[ (\bar s - \hat s) \cdot \frac{ \P_{-n} \x_n \x_n^\top \Sigmab^{-1} }{\x_n^\top \P_{-n} \x_n} \right] \right\| \\ 
  %%%
  &\le \E_E \left\| \E_E \left[ (\bar s - \hat s) \cdot \frac{ \Sigmab^{\frac12} \z_n \z_n^\top \Sigmab^{-\frac12} }{\x_n^\top \P_{-n} \x_n} \mid \P_{-n} \right] \right\| 
\end{align*}
\end{frame}
 

\begin{frame}
    \frametitle{Rank-one projection for sub-Gaussians: higher moments}
Let  $\x=\A^{\frac12}\z$, fix any $\v$ and let $E$ be the event that
$\x^\top\x\geq \frac12\tr\,\A$.
    \begin{align*}
      \hspace{-3mm}\E\bigg[\Big(\v^\top\frac{\x\x^\top}
      {\x^\top\x}\v\Big)^2\bigg]
      &\leq
        \Big(\frac2{\tr\,\A}\Big)^2\E\big[(\v^\top\x)^4\mid
        E\big]\Pr(E) +
        \E\bigg[\Big(\v^\top\frac{\x\x^\top}
      {\x^\top\x}\v\Big)^2 \mid
        E\bigg]\Pr(\neg E)\\
      &\leq
        \Big(\frac2{\tr\,\A}\Big)^2\E\big[(\z^\top\A^{\frac12}\v\v^\top\A^{\frac12}\z)^2\big]
        + 2\ee^{-c  r} \|\v\|^4\\
      &\overset{(*)}{\leq}
        C\cdot\Big(\frac{1}{\tr\,\A}\Big)^2\tr\,(\A^{\frac12}\v\v^\top\A^{\frac12})^2
        + 2\ee^{-c  r} \|\v\|^4 \\
      &\leq C\cdot
        \Big(\frac{\|\A\|}{\tr\,\A}\Big)^2\|\v\|^4+2\ee^{-cr}\|\v\|^4
        = O(1/r^2)\cdot \|\v\|^4,
    \end{align*}
    where $(*)$ uses Bai-Silverstein, so it can easily be extended to
    higher moments
  \end{frame}

  \begin{frame}
    \frametitle{Bounds using nuclear rank}
    Let $\bar s = \tr\,\bar\P\Sigmab = n/\bar\gamma$. Our current bounds are of the form:
    \begin{align*}
      \|\I-\E[\P]\bar\P^{-1}\| =
      O\Big(\bar\gamma^{\alpha}\tfrac{\sqrt{n+\bar s}}{\bar s}\Big)
    \end{align*}
W.l.o.g.~assume that $\lambda_1=1$. We lower-bound $\bar s$ and upper bound $\bar\gamma$:
    \begin{align*}
n= \sum_i \frac{\bar\gamma\lambda_i}{\bar\gamma\lambda_i + 1}
      \geq
      \frac{\bar\gamma\lambda_1}{\bar\gamma\lambda_1+1}\frac{\sum_i\lambda_i}{\lambda_1}
      = \frac{\bar\gamma}{\bar\gamma+1}\cdot r\\
\Rightarrow \quad
      \bar\gamma\leq \frac n{r-n}
\quad\text{and}\quad
      \bar s\geq \frac1{\frac n{r-n}+1}\cdot r = r-n
    \end{align*}
Fixing $\rho = r/n>1$, we have
    $\bar\gamma\leq \frac1{\rho-1}$ and $\bar s\geq r\,\frac{\rho-1}{\rho}$, so:
    \begin{align*}
      \|\I-\E[\P]\bar\P^{-1}\| = O\Big(C_\rho\cdot\frac1{\sqrt r}\Big)
      = O\Big(C_\rho\cdot\sqrt{\frac{\kappa}{d}}\ \Big),
    \end{align*}
because $r\geq d/\kappa$, but note that we require $r/n>1$ instead of $d/n>1$
  \end{frame}

  \begin{frame}
    \frametitle{Bounds using the dimension}
    Let $\bar s = \tr\,\bar\P\Sigmab = n/\bar\gamma$. Our current bounds are of the form:
    \begin{align*}
      \|\I-\E[\P]\bar\P^{-1}\| =
      O\Big(\bar\gamma^{\alpha}\tfrac{\sqrt{n+\bar s}}{\bar s}\Big)
    \end{align*}
W.l.o.g., $\lambda_1=1$. We lower/upper-bound $\bar s$ and $\bar\gamma$:
    \begin{align*}
n= \sum_i \frac{\bar\gamma\lambda_i}{\bar\gamma\lambda_i + 1}
      &\geq
      \frac{\bar\gamma\lambda_{\min}}{\bar\gamma\lambda_{\min}+1}\cdot d\\
\Rightarrow \quad
      \bar\gamma\leq \frac1{\lambda_{\min}}\frac n{d-n}
\quad&\text{and}\quad
      \bar s\geq \frac{\lambda_{\min}}{\frac n{d-n}+1}\cdot d = \lambda_{\min}(d-n)
    \end{align*}
For $\rho = d/n>1$ and $\kappa=\frac1{\lambda_{\min}}$, we have
$\bar\gamma\leq \frac\kappa{\rho-1}$ and $\bar s\geq r\,\frac{\rho-1}{\kappa\rho}$, so: 
    \begin{align*}
      \|\I-\E[\P]\bar\P^{-1}\| = O\Big(C_\rho\cdot \frac{\kappa^{\alpha+\frac12}}{\sqrt d}\Big)
    \end{align*}
\end{frame}
    
    \begin{frame}
      \frametitle{Interpolated bounds}
    Let $\bar s = \tr\,\bar\P\Sigmab = n/\bar\gamma$. Our current bounds are of the form:
    \begin{align*}
      \|\I-\E[\P]\bar\P^{-1}\| =
      O\Big(\bar\gamma^{\alpha}\tfrac{\sqrt{n+\bar s}}{\bar s}\Big)
    \end{align*}
W.l.o.g., $\lambda_1=1$. We lower/upper-bound $\bar s$ and $\bar\gamma$:
  \vspace{-5mm}
    \begin{align*}
      n= \sum_i \frac{\bar\gamma\lambda_i}{\bar\gamma\lambda_i + 1}
      \geq
      \frac{\bar\gamma\lambda_k}{\bar\gamma\lambda_k+1}
      \overbrace{\bigg(k-1+\frac{\sum_{i\geq k}\lambda_i}{\lambda_k}\bigg)}^{r_k}\\
\Rightarrow \quad
      \bar\gamma\leq \frac1{\lambda_k}\frac n{r_k-n}
\quad\text{and}\quad
      \bar s\geq \frac{\lambda_k}{\frac n{r_k-n}+1}\cdot r_k = \lambda_k(r_k-n)
    \end{align*}
For $\rho = \frac{r_k}{n}>1$ and $\kappa_k=\frac1{\lambda_k}$, we have
    $\bar\gamma\leq \frac{\kappa_k}{\rho-1}$ and $\bar s\geq r_k\frac{\rho-1}{\kappa_k\rho}$, so:
    \begin{align*}
      \|\I-\E[\P]\bar\P^{-1}\| = O\Big(C_\rho\cdot\frac{\kappa_k^{\alpha+\frac12}}{\sqrt {r_k}}\Big)
    \end{align*}
For $k=1$ we recover nuclear rank, for $k=d$ we recover the dimension.
\end{frame}

\begin{frame}
  \frametitle{Interpolated bounds: motivating example}
  Let $\lambda_1=1$ and $\lambda_2=...=\lambda_t=\frac1{t^\beta}$ and
  $\lambda_{t+1}=\epsilon$.\\[3mm]
  Let $k=1$: We have $r_1 \simeq  t^{1-\beta}$ and $\kappa_1=1$ so
\begin{align*}
  n=\Omega(t^{1-\beta})\quad\text{and}\quad
  \|\I-\E[\P]\bar\P^{-1}\|= O\Big(C_\rho \frac1{\sqrt{t^{1-\beta}}}\Big). 
\end{align*}
Let $k=2$: We have $r_2 \simeq t$ and $\kappa_2 = t^\beta$ so
\begin{align*}
    n=\Omega(t)\quad\text{and}\quad
  \|\I-\E[\P]\bar\P^{-1}\|= O \Big(C_\rho
  \frac{t^{\beta(\alpha+\frac12)}}{\sqrt{t}}\Big) = O\Big(C_\rho\frac1{\sqrt{t^{1-\beta(2\alpha+1)}}}\Big)
\end{align*}
Conclusions:
\begin{enumerate}
\item The convergence rate seems best for $k=1$
\item But the range of applicable $n$ is better for $k>1$.
\end{enumerate}
\end{frame}

% \begin{frame}
%   \frametitle{Converting from main theorem to sketching}
%   Let $\A\in\R^{m\times n}$ and let $\S$ be a $k\times m$ sketching
%   matrix, $\E[\S^\top\S]=\I$\\
%   Define $\Sigmab=\A^\top\A$. We let $\Z$ be a $k\times n$ random
%   matrix such that:
%   \begin{align*}
% \Z = \S\A\Sigmab^{-\frac12}
%   \end{align*}
%   Note that if $\S$ has i.i.d.~rows then so does $\Z$. Moreover:
%   \begin{align*}
% \Z\Sigmab^{\frac12} = \S\A\quad\text{and}\quad \E[\Z^\top\Z] =
%     \Sigmab^{-\frac12}\A^\top\E[\S^\top\S]\A\Sigmab^{-\frac12} = \I
%   \end{align*}
% Suppose that $\S$ is the leverage score sampling sketch, i.e.,
% $\s_i=\frac1{\sqrt{k p_{j_i}}}\e_{j_i}$, and $p_{j_i}=\frac1n\a_{j_i}^\top\Sigmab^{-1}\a_{j_i}$.
%   Then $\|\z_i\|^2= \frac nk$ with probability $1$ for all $i$.
%   \begin{align*}
%     \E\big[\ee^{(\v^\top\z)^2/t^2}\big]
%     &\leq \E\big[\ee^{\|\v\|^2\|\z\|^2/t^2}\big] =\ee^{n/(k t^2)}
%       \quad\text{for }\v\in S^{n-1}\\
%         \|\z\|_{\psi_2}
%     & = \sup_{\v\in S^{n-1}}\|\v^\top\z\|_{\Psi_2} \\
%     &= \sup_{\v\in
%       S^{n-1}}\inf\Big\{t>0:\E\big[\ee^{(\v^\top\z)^2/t^2}\big]\leq
%       2\Big\}\leq \sqrt{ 2n/k}
%   \end{align*}
% \end{frame}


  \begin{frame}
    \frametitle{Rotational invariance of the sub-gaussian norm}
    Let $\z$ be an $m$-dimensional sub-gaussian random variable.\\
    Let $\U\in\R^{m\times n}$ be a matrix such that $\|\U\v\|=\|\v\|$
    for all $\v\in\R^n$\\
    This means that if $\v\in S^{n-1}$ then $\U\v\in S^{m-1}$, so
    \begin{align*}
      \|\U^\top\z\|_{\psi_2} = \sup_{\v\in
      S^{n-1}}\|(\U\v)^\top\z\|_{\psi_2}
      \leq\sup_{\u\in S^{m-1}}\|\u^\top\z\|_{\psi_2} = \|\z\|_{\psi_2}
    \end{align*}
\textbf{Reduction} from $\Z\Sigmab^{\frac12}$ to $\S\A$ for a
non-square $\A$\\[3mm]
Let $\A\in\R^{m\times n}$ be a rank $n$ matrix and let
$\U=\A\Sigmab^{-\frac12}$, $\Sigmab = \A^\top\A$\\
Let $\S$ be a $k\times m$ matrix with i.i.d.~sub-gaussian rows
and let $\Z=\S\U$.\\[2mm]

Then $\Z$ also has i.i.d.~sub-gaussian rows with the same constant,
so Theorem 1 applies to $\X=\S\A = \Z\Sigmab^{\frac12}$\\[5mm]

\textbf{Todo:}
\begin{enumerate}
\item Check if this reduction works for Gaussian and Rademacher sketches
  (for Gaussians the reduction is not necessary, just a sanity check)
\item Extend this to allow a rank-deficient matrix $\A$
\end{enumerate}
  \end{frame}

  \begin{frame}
    \frametitle{Applying Theorem 1 to the Gaussian sketch}
   Let $\S\sim\Nc_{k\times m}(\zero,\I_k,\I_m)$, so that $\E[\S^\top\S]=k\I$\\
   Fix any $m\times n$ matrix $\A$. Then $\S\A\sim
   \Nc_{k\times n}(\zero,\I_k,\A^\top\A)$\\
   Let $\Sigmab=\A^\top\A$ and $\Z\sim\Nc_{k\times n}(\zero,\I_k,\I_n)$. Then
   $\Z\Sigmab^{\frac12}\overset{d}{=}\S\A$\\
   Each row $\s_i^\top$ of $\S$ as well as $\z_i^\top$ of $\Z$
   has constant sub-Gaussian norm.\\[5mm]

   \textbf{Testing rotational invariance.}
   Let $\Z' = \S\A\Sigmab^{-\frac12}$. Then,
   $\Z'\overset{d}{=}\Z$,\\
   so the sub-Gaussian norm of each row of $\Z'$ is also constant
  \end{frame}
  
  \begin{frame}
    \frametitle{Applying Theorem 1 to the Rademacher sketch}
   Let $\S$ be $k\times m$ with i.i.d.~Rademacher entries, so that $\E[\S^\top\S]=k\I$\\
   Each row $\s_i^\top$ of $\S$ is sub-Gaussian with
   $\|\s_i\|_{\psi_2}\leq 2$.\\
   Fix any $m\times n$ matrix $\A$ and let $\Sigmab=\A^\top\A$. In
   general,   $\S\Sigmab^{\frac12}\overset{d}{\neq}\S\A$\\[5mm] 

   \textbf{Applying rotational invariance.}
   Let $\Z = \S\A\Sigmab^{-\frac12}$. In general,
   $\Z\overset{d}{\neq}\S$,\\
   but the covariance matrix of each row is the $\I$, the same for
   $\Z$ and $\S$\\
   and sub-Gaussian norm of each row of $\Z$ is bounded by $2$\\
   Since $\Z\Sigmab^{\frac12}=\S\A$, we can apply Theorem 1.\\[5mm]

\textbf{Note:} This immediately generalizes to any sketching matrix
$\S$ consisting of i.i.d.~entries with constant sub-Gaussian norm.
 \end{frame}

 \begin{frame}
   \frametitle{Applying Theorem 1 to the i.i.d.~row-sampling sketch}
   Let $\S$ be $k\times m$, with $i$th row $\s_i$ equal to
   $\frac1{\sqrt p_{j}}\e_j$ w.p.~$p_j$, so $\E[\S^\top\S]=k\I$\\
   Each row $\s_i^\top$ of $\S$ is sub-Gaussian because for $\v\in S^{m-1}$:
   \begin{align*}
\E\big[\ee^{(\v^\top\s)^2/t^2}\big] =
     \sum_{j=1}^mp_j\ee^{v_i^2/(p_it^2)}\leq 1 +
     p_{\min}\ee^{1/(p_{\min}t^2)}
   \end{align*}
   This implies that $\|\s\|_{\psi_2}\leq \frac2{\sqrt{p_{\min}}}$\\
   Note that this bound is at least $2\sqrt m$ for any distribution $p_1,...,p_m$\\[5mm]
Thus, rotational invariance reduction is not sufficient because it
leads to at least an additional factor $m$ in the error bound of
Theorem 1.
 \end{frame}

  \begin{frame}
    \frametitle{Applying Theorem 1 to leverage score sampling}
    Let $\S$ be $k\times m$, with $i$th row $\s_i$ equal to
    $\frac1{\sqrt p_{j}}\e_j$ w.p.~$p_j$, so $\E[\S^\top\S]=k\I$\\
    Fix $m\times n$ matrix $\A$ and let $\Sigmab=\A^\top\A$. Set $p_j
    = \frac1n\a_j^\top\Sigmab^{-1}\a_j$.\\
Note that still $\|\s_i\|\geq \sqrt m$, so we define $\Z=\S\A\Sigmab^{-\frac12}$.\\
Since $\|\z_i\|^2=n$, each row of $\Z$ is sub-Gaussian because for $\v\in S^{m-1}$:
  \begin{align*}
    \E\big[\ee^{(\v^\top\z)^2/t^2}\big]
    \leq \E\big[\ee^{\|\v\|^2\|\z\|^2/t^2}\big] =\ee^{n/t^2}
  \end{align*}
  This implies a bound of $\|\z\|_{\psi_2}\leq 2\sqrt n$. This is
  potentially far better then $\frac2{\sqrt{p_{\min}}}\geq 2\sqrt m$ which
  follows via rotational invariance.\\[5mm]

\textbf{Note:}  This bound is still too weak to give non-vacuous
guarantees in Theorem 1. Moreover, it is tight for example when
$\A=\I_n$.\\[4mm]
\textbf{Conclusion:} The only way that row sampling can have a small
sub-Gaussian norm is if the matrix $\A$ itself was generated from
i.i.d.~samples of sub-Gaussian distribution

 \end{frame}


\begin{frame}
  \frametitle{Converting to trace bounds for low-rank approximation}
  Let $\A\in\R^{m\times n}$ and let $\S$ be a $k\times m$ sketching
  matrix, $\E[\S^\top\S]=\I$\\
    Define $\Sigmab=\A^\top\A$, $\Z=\S\A\Sigmab^{-1}$ and $\X=\Z\Sigmab^{\frac12}$. Suppose
    that we show
    \begin{align*}
      (1-\epsilon)\bar\P\preceq \E[\I-\X^\dagger\X]\preceq(1+\epsilon)\bar\P.
    \end{align*}
    Then, it follows that:
    \begin{align*}
      \E\big[\|\A-\A(\S\A)^\dagger(\S\A)\|_F^2\big]
      &= \tr\,\A\E\big[\I-(\S\A)^\dagger(\S\A)\big]\A^\top\\
      &= \tr\,\A\E\big[\I-\X^\dagger\X\big]\A^\top\\
       &\leq (1+\epsilon)\cdot \tr\,\A\bar\P\A^\top,
    \end{align*}
    and the lower bound follows analogously.
  \end{frame}

\begin{frame}
  \frametitle{The Counterexample}

  Suppose that $\X=\Z\Sigmab^{\frac12}$ where $\Z$ has
  i.i.d.~Rademacher entries and $\one_d$ is the eigenvector of
  $\Sigmab$ corresponding to $\lambda_{\min}$. \\
$\one_d$ is an eigenvector of
$\I-\bar\P=\bar\gamma\Sigmab(\bar\gamma\Sigmab+\I)^{-1}$ with
eigenvalue $\frac{\bar\gamma\lambda_{\min}}{\bar\gamma\lambda_{\min}+1}$.\\
Let $E = [\z_1=\one_d]$. If $E$ holds then
  $\x_1=\sqrt{\lambda_{\min}}\one_d$.
  
  \begin{align*}
    \|(\I-\bar\P)^{-\frac12}\E[\X^\dagger\X](\I-\bar\P)^{-\frac12}-\I\|
    &\geq \frac1d\one_d^\top
      (\I-\bar\P)^{-\frac12}\E[\X^\dagger\X](\I-\bar\P)^{-\frac12}\one_d-1\\
    &=\frac{\bar\gamma\lambda_{\min}+1}{d\bar\gamma\lambda_{\min}}
      \one_d^\top\E[\X^\dagger\X]\one_d-1\\
&\geq \frac1{d\bar\gamma\lambda_{\min}}
\E[\one_d^\top\X^\dagger\X\one_d\mid E]\Pr(E)-1\\
&=     \frac1{d\bar\gamma\lambda_{\min}}
\one_d^\top\one_d2^{-d}-1 = \frac{2^{-d}}{\bar\gamma\lambda_{\min}}-1
  \end{align*}
\end{frame}


\begin{frame}
  \frametitle{Inverting the expression: exponential decay}
  Suppose that $\lambda_i = C\cdot a^{i-1}$ for some
  $a\in(0,1)$. W.l.o.g., assume that $C=1$.
  \begin{align*}
    k&=  \sum_{i\geq 1} \frac{\gamma\lambda_i}{\gamma\lambda_i+1}
=\sum_{i\geq 0}\frac1{1 + \gamma^{-1}a^{-i}}
      \\
      & \simeq \int_0^\infty \frac1{1+\gamma^{-1}a^{-x}}dx
        =\frac{\ln(a^x+\gamma^{-1})}{\ln(a)} \Big|_0^\infty
    \\
    &= \frac{\ln(\gamma^{-1})}{\ln(a)} -
      \frac{\ln(1+\gamma^{-1})}{\ln(a)} = \frac{\ln(1/(\gamma+1))}{\ln(a)}.
  \end{align*}
 From this, we get $\gamma \simeq a^{-k}-1$ and the low-rank
 approximation error is:
 \begin{align*}
   \E\big[\|\A-\A\P\|_F^2\big] \simeq k/\gamma\simeq \frac{C\cdot k}{a^{-k}-1}.
 \end{align*}
\end{frame}

\begin{frame}
  \frametitle{Inverting the expression: harmonic decay}
  Suppose that $\lambda_i = 1/i$. Then, we have:
  \begin{align*}
    k&=  \sum_{i=1}^n \frac{\gamma\lambda_i}{\gamma\lambda_i+1}
=\sum_{i=1}^n\frac1{1 + \gamma^{-1}i}
      \\
      & \simeq \int_0^n \frac1{1+\gamma^{-1}x}dx
        =\gamma\ln(1+\gamma^{-1}x) \big|_0^n
    \\
    &= \gamma\ln(1+n/\gamma).
  \end{align*}
  It follows that $E \simeq k/\gamma$ satisfies the equation: $E =
  \ln(1+\frac nk E)$.\\
  As long as $n>2k$, this will converge to the solution starting from
  $E_0=1$, \\
  so we get a series of improving approximations:
  \begin{align*}
    E_1 = \ln(1+\tfrac nk),\quad E_2 = \ln\!\Big(1+\tfrac
    nk\ln\!\big(1+\tfrac nk\big)\Big),\quad ...
  \end{align*}
  It can also be expressed via the Lambert W-function:
  \begin{align*}
    \E\big[\|\A-\A\P\|_F^2\big]\simeq  -W\big(-\tfrac nk\ee^{k/n}\big) - \tfrac kn
  \end{align*}
\end{frame}

\begin{frame}
  \frametitle{Inverting the expression: quadratic decay}
  Suppose that $\lambda_i = 1/i^2$. Then, we have:
  \begin{align*}
    k&=  \sum_{i\geq 1} \frac{\gamma\lambda_i}{\gamma\lambda_i+1}
=\sum_{i\geq 1}\frac1{1 + \gamma^{-1}i^2}
= \frac\pi 2 \sqrt\gamma \underbrace{\coth(\pi\sqrt\gamma)}_{\simeq 1} - \frac12,
       % = \frac\pi 2 \sqrt\gamma \,\frac{\ee^{2 \pi\sqrt\gamma}+1}{\ee^{2
       % \pi\sqrt\gamma}-1} - \frac12
  \end{align*}
  so we obtain that $\gamma \simeq (2k+1)^2/\pi^2$ and so:
  \begin{align*}
    \E\big[\|\A-\A\P\|_F^2\big]\simeq \frac{\pi^2 k}{(2k+1)^2}    
  \end{align*}
\end{frame}

\begin{frame}
  \frametitle{Inverting the expression: polynomial decay}
  Suppose that $\lambda_i = 1/i^\alpha$ for $\alpha>1$. Then, we have:
  \begin{align*}
    k&=  \sum_{i\geq 1} \frac{\gamma\lambda_i}{\gamma\lambda_i+1}
       =-1 + \sum_{i\geq 0}\frac1{1 + \gamma^{-1}i^\alpha}
    \\
&\simeq -\frac12+\int_0^\infty \frac1{1 + \gamma^{-1}x^\alpha}dx = -\frac12 +
         \gamma^{1/\alpha}\,\frac\pi\alpha\mathrm{csc}\Big(\frac\pi\alpha\Big).
  \end{align*}
  It follows that $\gamma \simeq
  \big((k+\frac12)\frac\alpha\pi\sin(\frac\pi\alpha)\big)^\alpha$, and
  \begin{align*}
    \E\big[\|\A-\A\P\|_F^2\big]\simeq \frac
    k{(k+\frac12)^\alpha}\,\bigg(\frac{\pi/\alpha}{\sin(\pi/\alpha)}\bigg)^\alpha. 
  \end{align*}
  For large $\alpha$, 
  $\sin(\frac\pi\alpha)\simeq\frac\pi\alpha$,
  so then we get $\gamma\simeq (k+\frac12)^\alpha$.
\end{frame}

  \begin{frame}
    \frametitle{Spectral norm using Burkholder and $\epsilon$-net}
    Let $\Nc\subseteq S^{d-1}$ be an $\epsilon$-net of the unit
    sphere. Then for any $\C$:
    \begin{align*}
      \|\Sigmab^{\frac12}(\P_{\perp}-\E[\P_{\perp}])\Sigmab^{\frac12}\|
      \overset\epsilon\simeq
      \sup_{\u\in\Nc}\u^\top\Sigmab^{\frac12}
      (\P_{\perp}-\E[\P_{\perp}])\Sigmab^{\frac12}\u.
    \end{align*}
Note that $\u^\top\Sigmab^{\frac12}(\P-\P_{-i})\Sigmab^{\frac12}\u\leq
\u^\top\Sigmab\u\leq \|\Sigmab\|=1$. So Azuma-Hoeffding says that:
    \begin{align*}
      \Pr\big\{|\u^\top\Sigmab^{\frac12}(\P_{\perp}-\E[\P_{\perp}])\Sigmab^{\frac12}\u|\geq t\big\}\leq
      \ee^{-\frac{t^2}{8k}}.
    \end{align*}
    So using a union bound, and noting that $|\Nc|\leq (3/\epsilon)^d$, we have 
    \begin{align*}
      \Pr\Big\{\|\Sigmab^{\frac12}(\P_{\perp}-\E[\P_{\perp}])\Sigmab^{\frac12}\|\Big\}
      &\leq \sum_{\u\in\Nc}
\Pr\big\{|\u^\top\Sigmab^{\frac12}(\P_{\perp}-\E[\P_{\perp}])\Sigmab^{\frac12}\u|\geq
        t\big\}
        \\
      &\leq (3/\epsilon)^d\ee^{-\frac{t^2}{8k}} =
        \ee^{-\frac{t^2}{8k} + d\ln(3/\epsilon)}
    \end{align*}    
  \end{frame}


  \begin{frame}
    \frametitle{Integrating out the tail}
    Let $\x\sim\Nc(\zero,\I_d)$. Note that $\E[\x^\top\x]=d$. Take any
    $\Ec$ s.t. $\Pr(\neg \Ec)=\delta$.\\
    \emph{Goal}:
    Find an upper bound for $\E[\x^\top\x\cdot \one_{\neg \Ec}]$.
    \begin{align*}
      \E[\x^\top\x\cdot \one_{\neg \Ec}]
      &= \int_{0}^\infty\Pr(\x^\top\x\cdot \one_{\neg \Ec}\geq x)\,dx
      \\
      &\leq 2d\delta + \int_{2d}^\infty\Pr(\x^\top\x\geq x)\,dx
      \\
      &\leq 2d\delta + \int_{2d}^\infty\ee^{-cx}\,dx
      \\
      &=2d\delta + \frac1c\ee^{-2cd}.
    \end{align*}
\emph{Note}: In our setting we have roughly $\delta \leq \ee^{-cd}$.
\end{frame}

\begin{frame}
  \frametitle{Using just the variance via Chebychev}
  Assume that $\Var[\x^\top\x]\leq Cd$. Then, Chebychev says that for $x>0$:
  \begin{align*}
    \Pr(\x^\top\x\geq d+x)\leq\frac {Cd}{x^2}.
  \end{align*}
  Using this as the tail bound we get:
    \begin{align*}
      \E[\x^\top\x\cdot \one_{\neg \Ec}]
      &= \int_{0}^\infty\Pr(\x^\top\x\cdot \one_{\neg \Ec}\geq x)\,dx
      \\
      &\leq 2d^2\delta + \int_{2d^2}^\infty\Pr(\x^\top\x\geq x)\,dx
      \\
      &\leq 2d^2\delta +C'd\int_{2d^2}^\infty\frac{1}{x^2}\,dx
      \\
      &\leq 2d^2\delta + C'd\cdot \frac1{2d^2}
      \\
      &=O\Big(\frac1d\Big)\quad\text{for}\quad\delta\leq 1/d^3.
  \end{align*}
\end{frame}

\end{document}