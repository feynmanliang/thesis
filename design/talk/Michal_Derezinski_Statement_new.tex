\documentclass[11pt, oneside]{article}   	% use "amsart" instead of "article" for AMSLaTeX format
\usepackage{fullpage}
% \usepackage[1in]{geometry}                		% See
%geometry.pdf to learn the layout options. There are lots.

%\geometry{letterpaper}                   		% ... or a4paper or a5paper or ... 
%\geometry{landscape}                		% Activate for rotated page geometry
%\usepackage[parfill]{parskip}    		% Activate to begin paragraphs with an empty line rather than an indent
\usepackage{graphicx}				% Use pdf, png, jpg, or eps§ with pdflatex; use eps in DVI mode
								% TeX will automatically convert eps --> pdf in pdflatex		
\usepackage{amssymb}
\usepackage{url}
%SetFonts

%SetFonts


\title{Improved guarantees and a multiple-descent curve for \\
  % the CSSP
  Column Subset Selection %Problem
  and the Nystr\"om method}
\author{Micha{\l} Derezi\'nski (speaker), Rajiv Khanna and Michael W. Mahoney}
 \date{}							% Activate to display a given date or no date

\begin{document}
\maketitle
%\section{}
%\subsection{}
\vspace{-5mm}

Thank you \textbf{very} much. We are \textbf{really} excited to receive this award. My
name is Micha{\l}~Derezi\'nski, I am a postdoctoral researcher at UC Berkeley.

A key motivation for our work is interpretable data summarization,
which is a core challenge in reasoning 
about large datasets and machine learning models.  Data summarization
can be used, for example, to select a representative subset of gene
variants from a genetics dataset, or a collection of most informative
documents from a text database.  When data are represented
numerically, they are often described via matrices, in which case
linear algebra suggests a natural (and in a certain sense optimal) way
of performing data summarization: namely, finding the principal components
corresponding to the largest directions of variance.  These principal
components work well for black-box models that are evaluated only in
terms of prediction quality, but they are generally not interpretable
in terms of the domain from which the data are drawn. They do not
correspond to, say, a particular document or a gene variant, but
rather a complex mixture of them.  Neural networks, on which the
machine learning community increasingly relies, simply exacerbate this
problem of non-interpretability.  A long-standing challenge has been to find summaries of data
which mimic the numerical properties of principal components and which
are also interpretable. 

The cost of interpretability, when formulated this way, has strong
connections to the so-called Column Subset Selection Problem in
Randomized Numerical Linear Algebra. It
has been extensively studied in the literature, resulting in optimal
worst-case guarantees, developed for the first time nearly fifteen
years ago. These results are still relied upon as an important
technical tool in recent works, including the best paper at last
year’s International Conference on Machine Learning, which applied
them to Gaussian Process regression. The worst-case results suggest
that the cost of interpretability should become progressively higher
as the data summaries get larger.
To verify this, we carefully constructed a worst-case example for this problem; and we
showed that, for this example, the cost of interpretability \textbf{does} reach
the worse-case bound for \textbf{one} size of the summary, but then it rapidly
drops down for both larger and smaller sizes. As it turns out, in this and other
worst-case examples, the high cost of interpretability is only a
corner case.

% We empirically tested this claim on
% a worst-case example taken from the literature. In this example, the
% cost of interpretability does reach the bound for one size of the
% summary, but then it rapidly drops down for both larger and smaller
% sizes. So, even in the worst-case example, high cost is only a corner
% case. 

To explain this, we go beyond worst-case analysis, and our results are
able to accurately capture the non-linear behavior of the cost of
interpretability. Specifically, we conclude that, except for certain
pathological examples which we can characterize, the cost of
interpretability is far smaller than suggested by prior work, and it
is often negligible for real-world problems. To construct our
interpretable summaries, we use a randomized subset selection
technique based on Determinantal Point Processes (DPPs for short), which provide
%technique called Determinantal Point Processes, which is 
a probabilistic model of diversity that has emerged across many scientific domains.
DPPs were first discovered in physics as a model of fermions in thermal
equilibrium, and they have since been used in random matrix theory, graph
theory and quantum mechanics. More recently, thanks to the emergence
of efficient algorithms for DPP sampling, they have also become
popular in Randomized Numerical Linear Algebra and Machine~Learning.

Our analysis reveals that
the previously observed worst-case examples are part of a larger
phenomenon, which we call the multiple-descent curve in data
summarization.  This curve represents a family of phase transitions,
observed through a spike in the cost of interpretability, which occur
when the data exhibit an underlying hierarchical structure. The
multiple-descent curve can be observed not only in artificially
constructed pathological examples but also in real-world
problems. However, it can generally be avoided by tuning model
parameters. 

This phenomenon is very much reminiscent of the so-called double
descent curve, which is exhibited by many machine learning models
including deep neural networks. In the context of double descent, we
distinguish two regimes of machine learning: the "classical" regime,
where we have an abundance of training data relative to the number of
model parameters we wish to learn; and the "modern" regime, which
includes deep learning architectures, where there is far more
parameters than training data. While each regime allows for
constructing machine learning models that perform well on unseen data,
the phase transition between the two regimes may lead to a spike in
the error rate, resulting in poor performance. 

Both the double descent curve in machine learning, and the new
multiple-descent curve in data summarization, shown in our work, are
related to fundamental phase transitions observed in the behavior of
high-dimensional random matrices. Obtaining precise characterizations
of these phenomena is of crucial importance to understanding modern
machine learning as well as algorithmic-statistical tradeoffs that are
central to the foundations of data science.
\vfill
\textbf{Summary}:
Interpretable data summarization is a core challenge in reasoning
about large datasets and machine learning models.  Data summarization
can be used, e.g., to select a representative subset of gene variants
from a genetics dataset, or a collection of most informative documents
from a text database.  When data are represented numerically, they are
often described via matrices, in which case linear algebra suggests a
natural (and in certain ways optimal) way of performing data
summarization: find the principal components corresponding to the
largest directions of variance.  These principal components work well
for black-box models that are evaluated only in terms of prediction
quality, but they are generally not interpretable in terms of the
domain from which the data are drawn. They do not correspond to, say,
a particular document or a gene variant, but rather a complex mixture
of them.  Neural networks, on which the machine learning community
increasingly relies, simply exacerbate this problem.  A long-standing
challenge has been to find summaries of data which mimic the numerical
properties of principal components and which are also
interpretable. The added cost of this interpretability can be
significant, as shown by prior worst-case analysis.  In our work, we
exploit the inherent structure of data to show that, except for
pathological examples which we characterize, the cost of
interpretability is far smaller than suggested by prior worst-case
analysis, and it is often negligible for real-world problems.  Our
analysis reveals an intriguing new phenomenon, which we call the
multiple-descent curve in subset selection.  This is a phase
transition, observed through a spike in the cost of interpretability,
which occurs when the data exhibit an underlying hierarchical
structure.  The multiple-descent phenomenon can be observed in
practice, but it can also be avoided by tuning model parameters. 

\newpage
\begin{itemize}
\item Interpretability is important, but Modern ML models are
  black boxes.
  \begin{quote}
    Obtaining interpretable data summaries is an
    important challenge, especially in the context of modern ML and
    Deep Learning models, which are often black boxes.
  \end{quote}
  \begin{quote}
    Our work suggests that obtaining interpretable summaries can be
    done effectively, if we ignore pathological problem instances,
    which do not occur in the real-world.
  \end{quote}
\item Going beyond worst-case analysis to understand different phases
  of learning.
  \begin{quote}
Theoretical work in ML and Data Science, particularly involving
techniques from Randomized Numerical Linear Algebra, often focuses on
the worst-case examples.
\end{quote}
\begin{quote}
  Worst-case analysis does not capture the
fundamental phase transitions associated with learning, such as
double-descent and multiple-descent.
\end{quote}
\begin{quote}
Using techniques from statistical physics and random matrix theory,
such as Determinantal Point Processes, we can go beyond the worst-case
analysis and obtain precise characterizations of these phenomena.
\end{quote}
\item The paper that “forced Timnit Gebru out of Google”
  \begin{quote}
    I don’t know the specifics of the situation with Timnit, so I
    can’t comment on that. However, it's worth noting that Timnit's work underscores the
    importance of interpretability in machine learning, particularly when
    it comes to the implicit bias of machine learning models trained
    on large datasets. Our work suggests that the cost of interpretability is small, even
    negligible, for many real-world applications.
  \end{quote}
\item Connection to text summarization
  \begin{quote}
    While our work is foundational, and not specific to, say natural
    language processing, the method we analyze in this paper, namely
    Determinantal Point Processes, has been used in a number of
    concrete tasks such as document summarization and video
    summarization, and has proven empirically effective. Our
    theoretical analysis provides a better understanding of these
    empirical results.
  \end{quote}
  \end{itemize}
\vspace{2mm}

\noindent
\textbf{Media Contact:}
Micha{\l} Derezi\'nski\\
\underline{Email:} \texttt{mderezin@berkeley.edu}\\
\underline{Phone:} 408-680-4652
%\vspace{5mm}
\subsection*{Author Bios} 
\begin{minipage}{.7\textwidth}
  \underline{Full name:} \textbf{Micha{\l} Derezi\'nski}\\
\underline{Title:} Postdoctoral Researcher\\
\underline{Organization:} University of California, Berkeley\\
\underline{Bio:} Michał Dereziński is a Postdoctoral Researcher in the
Department of Statistics at University of California,
Berkeley. Previously, he was a research fellow at the Simons Institute
for the Theory of Computing. He obtained his Ph.D. in Computer Science
at University of California, Santa Cruz, advised by professor Manfred
Warmuth. Michał's research is focused on developing scalable
randomized algorithms with robust statistical guarantees for machine
learning, data science and optimization. More information is available
at: \url{https://users.soe.ucsc.edu/~mderezin/}.\\
\underline{Headshot:} \url{https://users.soe.ucsc.edu/~mderezin/photo.jpg}
\end{minipage}\hfill
\begin{minipage}{.25\textwidth}
  \includegraphics[width=\textwidth]{figs/photo-michal.jpg}
\end{minipage}
\vspace{5mm}

\noindent
\begin{minipage}{.7\textwidth}
\underline{Full Name:} \textbf{Rajiv Khanna}\\
\underline{Title:} Postdoctoral Researcher\\
\underline{Organization:} University of California, Berkeley\\
\underline{Bio:} Rajiv Khanna is a Postdoctoral Researcher at the
Foundations of Data Analysis Institute at University of California,
Berkeley working with Michael Mahoney. He graduated with a PhD from UT
Austin advised by Professors Joydeep Ghosh, and Alex Dimakis.  His
research interests include theoretical aspects of optimization and
more recently generalization in neural networks.
\end{minipage}
\hfill
\begin{minipage}{.25\textwidth}
  \includegraphics[width=\textwidth]{figs/photo-rajiv.jpg}
\end{minipage}
\vspace{5mm}

\noindent
\begin{minipage}{.7\textwidth}
\underline{Full name:} \textbf{Michael W. Mahoney}\\
\underline{Title:} Associate Professor\\
\underline{Organization:} International Computer Science Institute and
University of California at Berkeley\\
\underline{Bio:} Michael W. Mahoney is at the University of California
at Berkeley in the Department of Statistics and at the International
Computer Science Institute (ICSI).  He works on algorithmic and
statistical aspects of modern large-scale data analysis.  Much of his
recent research has focused on large-scale machine learning, including
randomized matrix algorithms and randomized numerical linear algebra,
geometric network analysis tools for structure extraction in large
informatics graphs, scalable implicit regularization methods, and
applications in genetics, astronomy, medical imaging, social network
analysis, and internet data analysis.  He received him PhD from Yale
University with a dissertation in computational statistical mechanics,
and he has worked and taught at Yale University in the mathematics
department, at Yahoo Research, and at Stanford University in the
mathematics department.  Among other things, he is on the national
advisory committee of the Statistical and Applied Mathematical
Sciences Institute (SAMSI), he was on the National Research Council's
Committee on the Analysis of Massive Data, he co-organized the Simons
Institute's fall 2013 and 2018 programs on the foundations of data
science, and he runs the biennial MMDS Workshops on Algorithms for
Modern Massive Data Sets.  He is currently the Director of the
NSF/TRIPODS-funded FODA (Foundations of Data Analysis) Institute at UC
Berkeley.  He holds several patents for work done at Yahoo Research
and as Lead Data Scientist for Vieu Labs, Inc., a  startup reimagining
consumer video for billions of users.  More information is available
at: \url{https://www.stat.berkeley.edu/~mmahoney/}.\\
\underline{Headshot:} \url{https://www.stat.berkeley.edu/~mmahoney/misc/mwm18-big.png}
 \end{minipage}\hfill
\begin{minipage}[t]{.25\textwidth}
  \includegraphics[width=\textwidth]{figs/photo-mwm.png}
\end{minipage}


\end{document}  