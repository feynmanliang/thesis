\documentclass[10pt]{beamer}
%\beamertemplateshadingbackground{brown!70}{yellow!10}
\mode<presentation>
{
  %\usetheme{Warsaw}
  \usecolortheme{crane}
  % or ...

%  \setbeamercovered{transparent}
  % or whatever (possibly just delete it)
}
\setbeamertemplate{navigation symbols}{}
\setbeamertemplate{footline}[frame number]{}
\usepackage{tikz,pgfplots}
\pgfplotsset{compat=newest}
\usepackage[utf8]{inputenc}
\usetikzlibrary{patterns}
\usepackage{amssymb}
\usepackage{amsmath}
\usepackage{colortbl}
%\usepackage{multicol}
\usepackage{cancel}
\usepackage{ulem}
\usepackage{multirow}
\usepackage{relsize}
\usepackage{algorithm}
\usepackage{algorithmic}
\usepackage{forloop}% http://ctan.org/pkg/forloop
\newcounter{loopcntr}
\newcommand{\rpt}[2][1]{%
  \forloop{loopcntr}{0}{\value{loopcntr}<#1}{#2}%
}
%\pagestyle{plain}
%\input{defs2}
\def\opt{{\textsc{OPT}_k}}
\def\const{{\mathrm{const}}}
\def\nnz{{\mathrm{nnz}}}
\def\r{\sfrac{\sigma_{\w}^2}{\sigma_{\xib}^2}}
\def\rm{\sfrac{\sigma_{\xib}^2}{\sigma_{\w}^2}}
\def\cmark{\Green{\checkmark}}
\def\xmark{\Red{\large\sffamily x}}
\newcommand{\pdet}{{\mathrm{pdet}}}
\newcommand{\MSPE}[1] {{\mathrm{MSPE}\big[#1\big]}}
\newcommand{\MSE}[1] {{\mathrm{MSE}\big[#1\big]}}
\def\Poisson{{\operatorname{Poisson}}}
\def\PB{{\operatorname{PB}}}
\newcommand{\DP}[1]{\mathcal{DP}^{#1}}
\def\Ic{\mathcal{I}}
\def\Jc{\mathcal{J}}
\def\Mc{\mathcal M}
\def\Ec{\mathcal E}
\def\sr{{\mathrm{sr}}}
\def\ktd{{k^{\underline{d}}}}
\def\Det{{\mathrm{Det}}}
\def\detu{{\widecheck{\mathrm{Det}}_\mu^\gamma}}
\def\deto{{\widehat{\mathrm{Det}}_\mu^\gamma}}
\def\Zu{{\widecheck{Z}_\mu^{\gamma}}}
\def\Zo{{\widehat{Z}_\mu^{\gamma}}}
\def\Zun{{\widecheck{Z}_\mu^{\gamma_n}}}
\def\Zon{{\widehat{Z}_\mu^{\gamma_n}}}
\newcommand{\Er}{\mathrm{Er}}
\newif\ifDRAFT
\DRAFTtrue
\ifDRAFT
\newcommand{\marrow}{\marginpar[\hfill$\longrightarrow$]{$\longleftarrow$}}
\newcommand{\niceremark}[3]
   {\textcolor{red}{\textsc{#1 #2:} \marrow\textsf{#3}}}
\newcommand{\ken}[2][says]{\niceremark{Ken}{#1}{#2}}
\newcommand{\manfred}[2][says]{\niceremark{Manfred}{#1}{#2}}
\newcommand{\michael}[2][says]{\niceremark{Michael}{#1}{#2}}
\newcommand{\michal}[2][says]{\niceremark{Michal}{#1}{#2}}
\newcommand{\feynman}[2][says]{\niceremark{Feynman}{#1}{#2}}
%\usepackage[inline]{showlabels}
\else
\newcommand{\ken}[1]{}
\newcommand{\michael}[1]{}
\newcommand{\michal}[1]{}
\newcommand{\feynman}[1]{}
\fi
\newcommand{\norm}[1]{{\| #1 \|}}

\newcommand{\deff}{d_{\textnormal{eff}}}
\def\ee{\mathrm{e}}
\newcommand\mydots{\makebox[1em][c]{.\hfil.\hfil.}}
\def\Sd{\mathscr{S}_{\!d}}
\newcommand{\dx}{\dxy_{\!\cal X}}
\newcommand{\dxk}{\dxy_{\!\cal X}^k}
\newcommand{\dk}{\dxy^k}
\newcommand{\dxy}{\mathrm{D}}
\def\simiid{\overset{\textnormal{\fontsize{6}{6}\selectfont
i.i.d.}}{\sim}}
%\newcommand{\Dxy}{D_{\!\cal X\!,\cal Y}}
\def\vskx{{\mathrm{VS}_{\!\dx}^k}}
\def\vsk{{\mathrm{VS}_{\!D}^k}}
\def\vskxm{{\mathrm{VS}_{\!\dx}^{k-1}}}
\def\vskm{{\mathrm{VS}_{\!D}^{k-1}}}
\def\vsdx{{\mathrm{VS}_{\!\dx}^d}}
\def\vsd{{\mathrm{VS}_{\!D}^d}}
\newcommand{\vs}[1]{{\mathrm{VS}_{\!D}^{#1}}}
\newcommand{\sigd}{\boldsymbol\Sigma_{\!\dx}}
\def\wols{\w_{\mathrm{LS}}}
\def\wds{\boldsymbol\w_{\!D}^*}
\def\kd{K_{\!\dx}}

\def\poly{{\mathrm{poly}}}
\def\polylog{{\mathrm{polylog}}}
\def\DPP{{\mathrm{DPP}}}
\def\DPPcor{{\DPP_{\!\mathrm{cor}}}}
\def\DPPens{{\DPP_{\!\mathrm{ens}}}}
\newcommand{\DPPreg}[1]{{\DPP_{\!\mathrm{reg}}^{#1}}}
\def\Vol{{\mathrm{VS}}}
\def\Lev{{\mathrm{Lev}}}
\newcommand\todod[1]{\Red{\# DH: #1}}
\newcommand{\explain}[2]{\mathrel{\overset{\makebox[0pt]{\text{\tiny
#1}}}{#2}}}
\def\tot {{\mathrm{tot}}}
\def\checkmark{\tikz\fill[scale=0.4](0,.35) -- (.25,0) --
(1,.7) -- (.25,.15) -- cycle;}
\newcommand{\mnote}[1]{{\bf\large \Magenta{*}}\marginpar{\small \Magenta{#1}}}
\newcommand{\bnote}[1]{{\bf #1}}

\newcommand{\sqrtshort}[1]{{\sqrt{\white{\Big|}\!\!\smash{\text{\fontsize{9}{9}\selectfont$#1$}}}}}
\newenvironment{proofof}[2]{\par\vspace{2mm}\noindent\textbf{Proof of {#1} {#2}}\ }{\hfill\BlackBox}
\newcommand{\sets}[2]
{{\hspace{-0.3mm}[\hspace{-0.3mm}#1\hspace{-0.3mm}]\hspace{-0.3mm}\choose
\hspace{-0.3mm}#2\hspace{-0.3mm}}}
\DeclareMathOperator{\sgn}{\textnormal{sgn}}
\DeclareMathOperator{\adj}{\textnormal{adj}}
\def\Rb{{\mathbf{R}}}
\DeclareMathOperator{\ws}{\widetilde{\w}}
\newcommand{\inote}[1]{{\bf {#1}}}
\def\xib{\boldsymbol\xi}
\def\Sigmab{\mathbf{\Sigma}}
\def\Sigmabh{\widehat{\Sigmab}}
\def\Sigmabt{\widetilde{\Sigmab}}
\def\S{\mathbf{S}}
\def\T{\mathbf{T}}
\def\xt{\tilde{x}}
\def\xbt{\widetilde{\x}}
\def\xbh{\widehat{\x}}
\def\ubh{\widehat{\u}}
\def\dom {{\mathrm{dom}}}
\def\val {{\mathrm{val}}}
\def\out {{\mathrm{out}}}
\def\iin  {{\mathrm{iin}}}
\def\s {\mathbf{s}}
\def\q {\mathbf{q}}
\def\qt{\tilde{q}}
\def\itld {j}
\def\ubt {\tilde{\u}}
\def\n{\{1..n\}}
\def\cb {\mathbf{c}}
\def\cW{\mathcal W}
\def\Xt{\widetilde{X}}
\def\Dbt{\widetilde{\D}}
\def\xtb{\tilde{\mathbf{x}}}
\def\ytb{\tilde{\mathbf{y}}}
\def\Xtb{\widetilde{\mathbf{X}}}
\def\Xbb{\overline{\X}}
\def\Xb{{\bar{\X}}}
\def\ybb{\overline{\y}}
\def\f{{\mathbf{f}}}
\def\g{{\mathbf{g}}}
\def\fbb{{\overline{\f}}}
\def\fb{{\overline{f}}}
\def\Xc{\mathcal{X}}
\def\W{\mathbf W}
\def\L{\mathbf{L}}
\def\Rb{\mathbf R}
\def\Pc{\mathcal{P}}
\def\Nc{\mathcal{N}}
\def\Pt{\widetilde{P}}
\def\Hc{\mathcal{H}}
\def\Wc{\mathcal{W}}
\def\Cc{\mathcal{C}}
\def\p{\mathbf p}
%\def\r{\mathbf r}
\def\Y{\mathbf Y}
\def\H{\mathbf H}
\def\K{\mathbf K}
\def\Kh{\widehat{K}}
\def\Kbh{{\widehat{\K}}}
\def\Q{\mathbf Q}
\def\Qbar{{\bar{\mathbf Q}}}
\def\Ytb{\widetilde{\mathbf{Y}}}
\def\c{{n-d\choose s-d}}
\DeclareMathOperator{\Proj}{Proj}
\newcommand{\Span}{\mathrm{span}}
\newcommand{\ofsubt}[1]{\mbox{\scriptsize \raisebox{0.25pt}{$(#1)$}}}
%\raisebox{0.5pt}{$($}}#1\mbox{\tiny \raisebox{0.5pt}{$)$}}}
\newcommand{\ofsub}[1]{\mbox{\small \raisebox{0.0pt}{$(#1)$}}}
%\newcommand{\ofsubb}[1]{\mbox{\footnotesize \raisebox{0.5pt}{$(#1)$}}}
%\newcommand{\ofsub}[1]{(#1)}
%\newcommand{\ofsub}[1]{\mbox{\tiny$|$\hspace{-0.5pt}\raisebox{-0.5pt}{$#1$}}}
\newcommand{\of}[2]{{#1{\!\ofsub{#2}}}}
\newcommand{\oft}[2]{{#1{\!\ofsubt{#2}}}}
\newcommand{\fof}[2]{{#1({#2})}}
\newcommand{\yof}[2]{{#1{\ofsub{#2}}}}
%\newcommand{\yofb}[2]{{#1{\ofsubb{#2}}}}
\newcommand{\lazy}{FastRegVol}
\newcommand{\volsamp}{RegVol}

\newcommand{\Sm}{{S_{-i}}}
\newcommand{\Sp}{{S_{+i}}}
\ifx\BlackBox\undefined
\newcommand{\BlackBox}{\rule{1.5ex}{1.5ex}}  % end of proof
\fi
%\renewcommand{\dagger}{+}
\DeclareMathOperator*{\argmin}{\mathop{\mathrm{argmin}}}
\DeclareMathOperator*{\argmax}{\mathop{\mathrm{argmax}}}
\DeclareMathOperator*{\diag}{\mathop{\mathrm{diag}}}
\def\x{\mathbf x}
\def\y{\mathbf y}
\def\ybh{\widehat{\mathbf y}}
\def\ybb{\bar{\mathbf y}}
\def\xbb{\bar{\mathbf x}}
\def\yb{{\bar y}}
\def\ybt{\widetilde{\mathbf y}}
\def\yh{\widehat{y}}
\def\yhb{\widehat{\y}}
\def\yt{\widetilde{y}}
\def\z{\mathbf z}
\def\a{\mathbf a}
\def\b{\mathbf b}
\def\w{\mathbf w}
\def\v{\mathbf v}
\def\m{\mathbf m}
\def\wbh{\widehat{\mathbf w}}
\def\wh{\widehat{\mathbf w}}
\def\vbh{\widehat{\mathbf v}}
\def\wbt{\widetilde{\mathbf w}}
\def\e{\mathbf e}
\def\zero{\mathbf 0}
\def\one{\mathbf 1}
\def\u{\mathbf u}
\def\ubbar{\bar{\mathbf u}}
\def\f{\mathbf f}
\def\ellb{\boldsymbol\ell}

\def\X{\mathbf X}
\def\Xs{\widetilde{\X}}
\def\B{\mathbf B}
\def\A{\mathbf A}
\def\C{\mathbf C}
\def\U{\mathbf U}
\def\Ubt{\widetilde{\mathbf U}}
\def\Ubh{\widehat{\mathbf U}}
\def\Ubbar{\bar{\mathbf U}}
\def\F{\mathbf F}
\def\D{\mathbf D}
\def\V{\mathbf V}
\def\M{\mathbf M}
\def\Mh{\widehat{\mathbf M}}
%\def\S{\mathbf S}
\def\Stb{\widetilde{\mathbf{S}}}
\def\Sbh{\widehat{\mathbf{S}}}
\def\St{\widetilde{\S}}
\def\Sh{\widehat{S}}
\def\Sc{\mathcal{S}}
\def\Fc{\mathcal{F}}
\def\Vc{\mathcal{V}}
\def\Bc{\mathcal{B}}
\def\Dc{\mathcal{D}}
\def\Z{\mathbf Z}
\def\Zbh{\widehat{\mathbf Z}}
\def\Zbt{\widetilde{\mathbf Z}}
\def\Abh{\widehat{\mathbf A}}
\def\I{\mathbf I}
\def\Ic{\mathcal I}
\def\II{\mathbf {I \!\,I}}
%\def\II{\boldsymbol {\mathbb I}}
\def\A{\mathbf A}
\def\P{\mathbf P}
\def\Ph{\widehat{\mathbf P}}
\def\cP{\mathcal P}
\def\cR{\mathcal R}
\def\Xt{\widetilde{\mathbf{X}}}
\def\Xh{\widehat{\mathbf{X}}}
\def\Rh{\widehat{R}}
\def\Ot{\widetilde{O}}
\def\At{\widetilde{\A}}


\def\E{\mathbb E}
\def\R{\mathbb R}
\def\N{\mathbb N}
\def\Pr{\mathrm{Pr}}
%\def\C{\mathbb C}
\def\tr{\mathrm{tr}}
\def\Sbar{{\bar{S}}}
\def\cS{{\mathcal{S}}}
\def\Tbar{{\bar{T}}}
\def\Tt{{\widetilde{T}}}
\def\rank{\mathrm{rank}}
\def\Prob{\mathrm{Prob}}
\def\Var{\mathrm{Var}}
\def\Xinv{(\X^\top\X)^{-1}}
\def\XinvS{(\X_S\X_S^\top)^{-1}}
\def\ABinvS{(\A_S\B_S^\top)^{-1}}
\def\ABinv{(\A\B^\top)^{-1}}
\def\xinv{\x_i^\top\Xinv\x_i}
\def\Xinvr{(\lambda\I+\X_{-1}^\top\X_{-1})^{-1}}
\def\pdet{\mathrm{pdet}}
\newcommand{\vol}{\mathrm{vol}}
%\newcommand{\defeq}{:=}
\newcommand{\defeq}{\stackrel{\textit{\tiny{def}}}{=}}
\newcommand{\di}{{[d+1]_{-i}}}
\newcommand{\cov}{\mathrm{cov}}
\let\origtop\top
\renewcommand\top{{\scriptscriptstyle{\origtop}}} % this makes transpose not so big

\definecolor{silver}{cmyk}{0,0,0,0.3}
\definecolor{yellow}{cmyk}{0,0,0.9,0.0}
\definecolor{reddishyellow}{cmyk}{0,0.22,1.0,0.0}
\definecolor{black}{cmyk}{0,0,0.0,1.0}
\definecolor{darkYellow}{cmyk}{0.2,0.4,1.0,0}
\definecolor{orange}{cmyk}{0.0,0.7,0.9,0}
\definecolor{darkSilver}{cmyk}{0,0,0,0.1}
\definecolor{grey}{cmyk}{0,0,0,0.5}
\definecolor{darkgreen}{cmyk}{0.6,0,0.8,0}
\newcommand{\Red}[1]{{\color{red}  {#1}}}
\newcommand{\Purple}[1]{{\color{purple}  {#1}}}
\newcommand{\Magenta}[1]{{\color{magenta}{#1}}}
\newcommand{\Green}[1]{{\color{darkgreen}  {#1}}}
\newcommand{\Blue}[1]{\color{blue}{#1}\color{black}}
\newcommand{\Orange}[1]{\textcolor{orange}{#1}\color{black}}
\newcommand{\Brown}[1]{{\color{brown}{#1}\color{black}}}
\newcommand{\Grey}[1]{{\color{grey}{#1}\color{black}}}
\newcommand{\white}[1]{{\textcolor{white}{#1}}}
\newcommand{\yellow}[1]{{\textcolor{reddishyellow}{#1}}}
\newcommand{\darkYellow}[1]{{\textcolor{darkYellow}{#1}}}
\newcommand{\grey}[1]{{\textcolor{grey}{#1}}}

\DeclareMathOperator{\half}{\frac{1}{2}}

\ifx\proof\undefined
\newenvironment{proof}{\par\noindent{\bf Proof\ }}{\hfill\BlackBox\\[2mm]}
\fi

\ifx\theorem\undefined
\newtheorem{theorem}{Theorem}
\fi

\ifx\example\undefined
\newtheorem{example}{Example}
\fi

\ifx\condition\undefined
\newtheorem{condition}{Condition}
\fi
\ifx\property\undefined
\newtheorem{property}{Property}
\fi

\ifx\lemma\undefined
\newtheorem{lemma}{Lemma}
\fi

\ifx\proposition\undefined
\newtheorem{proposition}{Proposition}
\fi

\ifx\remark\undefined
\newtheorem{remark}{Remark}
\fi

\ifx\corollary\undefined
\newtheorem{corollary}{Corollary}
\fi

\ifx\definition\undefined
\newtheorem{definition}{Definition}
\fi

\ifx\conjecture\undefined
\newtheorem{conjecture}{Conjecture}
\fi

\ifx\axiom\undefined
\newtheorem{axiom}{Axiom}
\fi

\ifx\claim\undefined
\newtheorem{claim}{Claim}
\fi

\ifx\assumption\undefined
\newtheorem{assumption}{Assumption}
\fi

\ifx\condition\undefined
\newtheorem{condition}{Condition}
\fi


\edef\polishl{\l}
\setlength{\columnsep}{0.7em}
\setlength{\columnseprule}{0mm}
\setlength{\arrayrulewidth}{1pt} 

\newcommand{\svr}[1]{{\textcolor{darkSilver}{#1}}}
\definecolor{brightyellow}{cmyk}{0,0,0.7,0.0}
\definecolor{lightyellow}{cmyk}{0,0,0.3,0.0}
\definecolor{lighteryellow}{cmyk}{0,0,0.1,0.0}
\definecolor{lightestyellow}{cmyk}{0,0,0.05,0.0}
\AtBeginSection[]
{
\begin{frame}<beamer>
\frametitle{Outline}
\tableofcontents[currentsection]
\end{frame}
}
\def\layersep{2.5cm}


%  \fboxsep=3pt
% %\fboxsep=0mm%padding thickness
% \fboxrule=2pt%border thickness

\setkeys{Gin}{width=0.7\textwidth}

\title[]{Random Projections in Hilbert Spaces}

\author[]{Micha{\l} Derezi\'{n}ski\\
UC Berkeley}

\begin{document}
\begin{frame}
  \titlepage
\end{frame}

\linespread{1.3}

\begin{frame}
  \frametitle{Quadrature in RKHS}

  Let $\Fc\subseteq L^2(\Xc,\mu)$ be a reproducing kernel Hilbert space (RKHS) with a
  positive definite kernel $k:\Xc\times\Xc\rightarrow \R$
  and probability measure $\mu$ on $\Xc$. \\
  Define the integral covariance operator $\Sigma:\Fc\rightarrow \Fc$
as follows:
  \begin{align*}
        \Sigma f(\cdot) = \int_xk(\cdot,x)f(x)\mu(dx) = \E_{x\sim\mu}[k(\cdot,x)f(x)].
  \end{align*}  
  Consider a set of inducing points $X=\{x_1,...,x_n\}\subseteq \Xc$.
A quadrature of $g\in \Fc$ induced by $X$ and
weights $(w_i)_{i=1}^n$  is given by:
  \begin{align*}
    \tilde g_w(\cdot) = \sum_{i=1}^nw_ik(\cdot,x_i).
  \end{align*}
Optimizing the weights of the quadrature w.r.t. the norm in $\Fc$, we
get:
\begin{align*}
  \inf_w \|g - \tilde g_w\|_{\Fc} = \|g - \hat g_X\|_{\Fc},\quad
 \text{for } \hat g_X := P_Xg,
\end{align*}
where $P_X$ is the orthogonal projection in $\Fc$ onto the span of $X$.
\end{frame}

\begin{frame}
  \frametitle{Quadrature error and expected projections}
  Suppose that the inducing points are drawn i.i.d.\ from the measure
  $\mu$.\\
  Then, we can write the expected quadrature error as follows:
  \begin{align*}
    \E_{X\sim\mu^n}\big[\|g-\hat g_X\|_{\Fc}^2\big]
    &=\E_{X\sim\mu^n}\big[\|(I-P_X)g\|_{\Fc}^2\big]
    \\
    &= \big\langle\,
      g,\E_{X\sim\mu^n}[I-P_X]\,g\,\big\rangle_{\Fc}
    \\
    &=\|g\|_{\Fc}^2 -  \big\langle\,
      g,\E_{X\sim\mu^n}[P_X]\,g\,\big\rangle_{\Fc}
  \end{align*}
Random matrix theory suggests the following deterministic equivalents:
  \begin{align*}
    \E_{X\sim\mu^n}[I-P_X] \simeq (\tfrac1\lambda\Sigma+I)^{-1},\quad
    \E_{X\sim\mu^n}[P_X]\simeq \Sigma(\Sigma + \lambda I)^{-1},
  \end{align*}
  where $\lambda>0$ is defined so that $\tr\,\Sigma(\Sigma+\lambda
  I)^{-1}=n$.\\[5mm]

 \emph{Note}: If $\Sigma$ is trace-class, then so is
 $\Sigma(\Sigma+\lambda I)^{-1}$,  but not $(\frac1\lambda\Sigma+I)^{-1}$.
\end{frame}

\begin{frame}
  \frametitle{Motivations and related work}
  The above (or closely related) setup arises in many related works, e.g.:
  \begin{itemize}
  \item \cite{bach2017equivalence} (random feature models)
  \item \cite{smola2007hilbert} (property testing and density estimation)
  \item \cite{sparse-variational-gp} (Gaussian processes)
  \item \cite{muandet2017kernel} (kernel mean embeddings)
  \item \cite{belhadji2019kernel} (numerical integration)
  \item \cite{BLLT19_TR} (linear regression)
  \end{itemize}
  \vspace{5mm}
  
We borrow the notation of \cite{bach2017equivalence}, and use
observations from \cite{belhadji2019kernel}
\end{frame}

\begin{frame}
  \frametitle{ Direct extension of the current result}
  \emph{Recall}: $P_X$ denotes the projection onto the span of $X$
  with respect to the norm induced by the RKHS $\Fc$ with kernel $k$
  and covariance operator $\Sigma$.\\[5mm]

 \emph{Notation}: We use $A\simeq_{\epsilon}B$ to denote
 $(1-\epsilon)B\leq A\leq(1+\epsilon)B$
 \\[5mm]

  \textbf{Conjecture}.
Let $\mu$ be sub-gaussian,
  $r=\tr\,\Sigma/\|\Sigma\|$ and fix $\rho=r/n>1$. Then:
  \begin{align*}
\E_{X\sim\mu^n}[I-P_X]\ \simeq_{O(1/\sqrt r)}\ (\tfrac1\lambda\Sigma+ I)^{-1}.
  \end{align*}
  \vspace{5mm}
  
Slightly stronger claim that may require additional
  assumptions:
  \begin{align*}
\E_{X\sim\mu^n}[P_X]\ \simeq_{O(1/\sqrt r)}\ \Sigma(\Sigma+ \lambda I)^{-1}.
  \end{align*}
\end{frame}

\begin{frame}
  \frametitle{Conjecture: Extension to $\rho<1$}
 \emph{Definition}. Let $\lambda_1,\lambda_2,...$ be the eigenvalues
 of $\Sigma$.
 For any $s\geq 0$, define
 \begin{align*}
   r_s = \frac1{\lambda_{s+1}}\sum_{i>s}\lambda_i.
 \end{align*}
\emph{Note}. The effective dimension of $\Sigma$ is recovered as $r_0=\tr\,\Sigma/\|\Sigma\|$.
\\[10mm]

  \textbf{Conjecture}.
Let $\mu$ be sub-gaussian, and $s<n<s+r_s$ for some $s\geq 0$. Then:
\begin{align*}
\E_{X\sim\mu^n}[I-P_X]\ \simeq_{O(1/\sqrt{r_s})}\ (\tfrac1\lambda\Sigma+ I)^{-1},
\end{align*}
where the constant may depend on $n/s$ and $r_s/n$.\\[5mm]


\emph{Intuition}: The error should not depend on the magnitude of the
top $s$ eigenvalues when $s$ is sufficiently smaller than $n$, because
those directions are accurately captured by the projection.

\end{frame}

\begin{frame}
  \frametitle{Implications for exponential decay}
  Suppose that $\lambda_i\asymp \alpha^i$ for $\alpha<1$. Then:
  \begin{align*}
    r_s \simeq_{O(1)}  \tr\,\Sigma / \|\Sigma\|\quad\text{for all $n$.}
  \end{align*}
  So, we may be able to show that if the eigenvalues of $\Sigma$
  exhibit exponential decay, then with $r=\tr\,\Sigma / \|\Sigma\|$:
  \begin{align*}
    \E_{X\sim\mu^n}[I-P_X]\ \simeq_{O(1/\sqrt{r})}\
    (\tfrac1\lambda\Sigma+ I)^{-1}\quad\text{for all $n$.}
    \end{align*}
\end{frame}

\begin{frame}
  \frametitle{Implications for polynomial decay}
  Suppose that $\lambda_i\asymp i^{-\beta}$ for $\beta>1$. Then:
  \begin{align*}
    r_s \simeq_{O(1)}  s/\beta\quad\text{for all $n$.}
  \end{align*}
  So, we may be able to show that if the eigenvalues of $\Sigma$
  exhibit polynomial decay, then:
  \begin{align*}
    \E_{X\sim\mu^n}[I-P_X]\ \simeq_{O(1/\sqrt{n})}\
    (\tfrac1\lambda\Sigma+ I)^{-1}\quad\text{for all $n$.}
    \end{align*}

\emph{Note}: This would imply that the deterministic equivalent is
asymptotically consistent for $n\rightarrow\infty$, with a fixed
covariance operator $\Sigma$.
\vspace{5mm}

\emph{Remark}: This claim also implies that the quadrature
error of $X\sim\mu^n$ is
within a constant factor of the best $n$-dimensional quadrature (following the analysis of
\cite{nystrom-multiple-descent}), and we can precisely characterize
that factor. 
    
\end{frame}

\begin{frame}[allowframebreaks]
  \frametitle{References}
  \tiny
  \bibliographystyle{alpha}
  \bibliography{../pap}
\end{frame}


\end{document}