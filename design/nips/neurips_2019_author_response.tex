\documentclass{article}

\usepackage{neurips_2019_author_response}

\usepackage[utf8]{inputenc} % allow utf-8 input
\usepackage[T1]{fontenc}    % use 8-bit T1 fonts
\usepackage{hyperref}       % hyperlinks
\usepackage{url}            % simple URL typesetting
\usepackage{booktabs}       % professional-quality tables
\usepackage{amsfonts}       % blackboard math symbols
\usepackage{nicefrac}       % compact symbols for 1/2, etc.
\usepackage{microtype}      % microtypography

\usepackage{mwe}

\bibliographystyle{unsrtnat}

\begin{document}

We would like to first stress that the main contribution of this
paper is theoretical (Theorems 1 and 3). For a long time, the area of
experimental design has been dominated by heuristic approaches (such
as greedy bottom-up and Fedorov's exchange method), and obtaining efficient
algorithms which guarantee a $(1+\epsilon)$-approximate solution is
considered a challenging task. For this task, our results
significantly improve on previous works that were published at top
machine learning and computational theory venues such as ICML'17
(\cite{pmlr-v70-allen-zhu17e}), SODA'19 (\cite{Nikolov:2019}) and COLT'19
(\cite{pmlr-v99-madan19a}). Those improvements are not limited to
approximation guarantees but also involve significantly faster
runtimes and a broader scope of applicable optimality criteria. While we do also
present numerical experiments comparing our algorithm to heuristic
approaches (Figure 1 in the submission), our primary point of comparison is with other methods that
provide theoretical guarantees for this task (Table 1 in the submission).

We also note that in both Bayesian and classical experimental design, A/C/D/V-optimality criteria have
historically been the standard evaluation metric (see \cite{chaloner1984optimal}) and
are still used in recent works published at NeurIPS
(\cite{mariet2017elementary}, Figure 1).  Improvements in algorithms for
optimal Bayesian design, such as our paper, immediately imply improved
performance guarantees for practical experimental design applications
ranging from clinical trials \cite{flournoy1993clinical,berry2002adaptive}
to materials science \cite{terejanu2012bayesian}. 

In the final version, we will add a conclusion section which
consolidates our results for ease of reference and differentiation
from prior work (as requested by Reviewers 1 and 4). 

\subsubsection*{Comments of Reviewer 3}
\begin{quote}
  \emph{``There are many Theorems/Lemmas and it is not always clear
    which are new results, and which are previous work.''} 
\end{quote}
As discussed above, our primary contribution is theoretical, and in
fact \emph{all} the lemmas and theorems presented in this paper are
novel results. We will clarify this in the final version.

\begin{quote}
\emph{``It would be much more convincing to add another figure that
  shows mean/sd of test error in k-fold CV.''}
\end{quote}
Typically, in experimental design literature we use the optimality
criteria as the performance metric. Nevertheless, we acknowledge that evaluating
mean-squared error under cross-validation may be more meaningful for NeurIPS
readers and we are happy to include this in the final version.%Figure~\ref{fig:mse} in our paper.

\subsubsection*{Comments of Reviewer 4}
\begin{quote}
  \emph{``The main theorems (Theorem 1-2) rely on sampling from DPPs,
    but the statement does not involve probabilistic modifications (in
    expectation, with high probability, etc.).''}
\end{quote}
Even though the output of our algorithms is randomized, the
approximation guarantees in Theorems 1 and 3 are in fact
deterministic. We achieve this by running the randomized procedure
repeatedly until the guarantee is satisfied. Because of this, the time
complexity guarantee holds with high probability, as explained in the
proof of Lemma 13 (page 7). We will clarify this further in the final version.

\begin{quote}
  \emph{``It would be helpful if the algorithms are summarized as
  pseudo-code and pointed from Introduction.''}
\end{quote}
We provide pseudo-code for our algorithm on page 5, document the
SDP being solved in equation (5), and describe the algorithm used to solve
the SDP in the experiments section (Section 5). We will try to
consolidate this information into one pseudo-code in the final version.

% \begin{figure}[h]
%   \centering
%   \includegraphics[width=0.30\textwidth]{example-image-a}
%   \caption{Regularized mean-squared error loss computed on remaining $n-k$
%     points.}
%   \label{fig:mse}
% \end{figure}

% While reviewer 4 is correct in observing that the greedy bottom-up method
% performs similarly to our method, we do not believe this is a significant
% concern because (1) most previous work in Bayesian experimental design
% (\cite{singh2018approximate,mariet2017elementary}) also do not significantly
% outperform greeedy bottom up / Fedorov exchange algorithms in experiments and
% (2) the theoretical guarantees for greedy methods (which rely on weak
% submodularity) are much looser and only applicable to D-optimality
% \cite{bouhtou2010submodularity}) whereas our theoretical results apply to all
% of the A/C/D/V-optimality criteria.

% Finally, we do not believe reviewer 4's concerns on pseudo-code are valid
% because we provide pseudo-code for our algorithm on page 5, document the
% SDP being solved in equation 5, and describe the algorithm used to solve
% the SDP in the experiments section (section 5).


\renewcommand\refname{\normalsize References}
\setlength{\bibsep}{0pt plus 0.2ex}
\tiny
\bibliography{refs}

\end{document}
