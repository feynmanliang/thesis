\section{Proof of Lemma~\ref{l:size}}
\label{appx: proof-of-l-size}

We first record an important property of the design $S_\mu^d$
which can be used to construct an over-determined design for any $n>d$. A similar
version of this result was also previously shown by
\cite{correcting-bias-journal} for a different determinantal design.

\begin{lemma}\label{l:decomposition}
  Let $\Xb\sim S_\mu^d$ and $\X\sim \mu^K$, where
  $K\sim\Poisson(\gamma)$. Then the matrix composed of a random
  permutation of the rows from $\Xb$ and $\X$ is distributed according to
  $S_\mu^{d+\gamma}$.
\end{lemma}

\begin{proof}
Let $\Xt$ denote the matrix constructed from the permuted rows of
$\Xb$ and $\X$.  Letting $\Z\sim\mu^{K+d}$, we derive the probability
$\Pr\big\{\Xt\!\in\! E\big\}$ by summing over the possible index subsets  $S\subseteq
[K+d]$ that correspond to the rows coming from $\Xb$:
\begin{align*}
  \Pr\big\{\Xt\in E\big\} &= \E\bigg[\frac{1}{\binom{K+d}{d}}
  \sum_{S:\,|S|=d}\frac{\E[\det(\Z_{S,*})^2\one_{[\Z\in E]}\mid
  K]}{d!\det(\Sigmab_\mu)}\bigg]\\
  &=\sum_{k=0}^\infty
    \frac{\gamma^k\ee^{-\gamma}}{k!}\,\frac{\gamma^dk!}{(k+d)!}\,
    \frac{\E\big[\sum_{S:\,|S|=d}\det(\Z_{S,*})^2\one_{[\Z\in E]}\mid
    K=k\big]}{\det(\gamma\Sigmab_\mu)}\\
  &\overset{(*)}{=} \sum_{k=0}^\infty
    \frac{\gamma^{k+d}\ee^{-\gamma}}{(k+d)!}
    \,\frac{\E[\det(\Z^\top\Z)\one_{[\Z\in E]}\mid K=k]}{\det(\gamma\Sigmab_\mu)},
\end{align*}
where $(*)$ uses the Cauchy-Binet formula to sum over all subsets $S$
of size $d$. Finally, since the sum shifts from $k$
to $k+d$, the last expression can be rewritten as
$\E[\det(\X^\top\X)\one_{[\X\in E]}]/\det(\gamma\Sigmab_\mu)$, where recall that
$\X\sim\mu^K$ and $K\sim\Poisson(\gamma)$, matching the definition of $S_\mu^{d+\gamma}$.
\end{proof}

We now proceed with the proof of Lemma \ref{l:size}, where we establish
that the expected sample size of $S_\mu^n$ is indeed $n$.

\begin{proofof}{Lemma}{\ref{l:size}}
  The result is obvious when $n=d$, whereas
  for $n>d$ it is an immediate consequence
  of Lemma \ref{l:decomposition}.
  Finally, for $n<d$ the expected sample
  size follows as a corollary of Lemma \ref{l:proj}, which states that
  \begin{align*}
\text{(Lemma \ref{l:proj})} \qquad\E\big[\I - \Xb^\dagger\Xb\big] =
    (\gamma_n\Sigmab_\mu + \I)^{-1},
  \end{align*}
  where $\Xb^\dagger\Xb$ is the orthogonal projection onto
  the subspace spanned by the rows of $\Xb$. Since the rank of this
  subspace is equal to the number of the rows, we have
  $\#(\Xb)=\tr(\Xb^\dagger\Xb)$, so
  \begin{align*}
    \E\big[\#(\Xb)\big] = d - \tr\big((\gamma_n\Sigmab_\mu +
    \I)^{-1}\big) =
    \tr\big(\gamma_n\Sigmab_\mu(\gamma_n\Sigmab_\mu+\I)^{-1}\big) = n,
  \end{align*}
  which completes the proof.
\end{proofof}

\section{Proofs for Section \ref{s:dp}}
\label{a:dp}

\begin{proofof}{Lemma}{\ref{t:ring}} \
 First, we show that $\A+\u\v^\top$ is d.p.~for any fixed
 $\u,\v\in\R^d$. Below, we use a standard identity for the rank one
 update of a determinant:
 $\det(\A+\u\v^\top)=\det(\A)+\v^\top\!\adj(\A)\u$. It follows that
 for any $\Ic$ and $\Jc$ of the same size,
  \begin{align*}
\E\big[\!\det(\A_{\Ic,\Jc}\!+\u_{\Ic}\v_{\Jc}^\top)\big] &=
    \E\big[\!\det(\A_{\Ic,\Jc}) +
    \v_{\Jc}^\top\adj(\A_{\Ic,\Jc}) \u_{\Ic}\big]\\
    &\overset{(*)}{=}\det\!\big(\E[\A_{\Ic,\Jc}]\big) +
      \v_{\Jc}^\top\adj\!\big(\E[\A_{\Ic,\Jc}]\big) \u_{\Ic}\\
    &=\det\!\big(\E[\A_{\Ic,\Jc} \!+ \u_{\Ic}\v_{\Jc}^\top]\big),
  \end{align*}
  where $(*)$ used \eqref{eq:adj}, i.e., the fact that for d.p.~matrices, adjugate commutes
  with expectation. Crucially, through the definition of an adjugate
  this step implicitly relies on the assumption that all the square
  submatrices of $\A_{\Ic,\Jc}$ are also  determinant preserving.
  Iterating this, we get that $\A+\Z$ is d.p.~for any fixed
  $\Z$. We now show the same for $\A+\B$:
  \begin{align*}
\E\big[\!\det(\A_{\Ic,\Jc}\!+\B_{\Ic,\Jc})\big]
    &=
      \E\Big[\E\big[\!\det(\A_{\Ic,\Jc}\!+\B_{\Ic,\Jc})\mid\B\big]\Big]\\
    &\overset{(*)}{=}\E\Big[\!\det\!\big(\E[\A_{\Ic,\Jc}]\!+\B_{\Ic,\Jc}\big)\Big]\\
      &= \det\!\big(\E[\A_{\Ic,\Jc}\!+\B_{\Ic,\Jc}]\big),
  \end{align*}
  where $(*)$  uses the fact that after conditioning on $\B$ we can
  treat it as a fixed matrix. Next, we show that $\A\B$ is determinant preserving via the Cauchy-Binet formula:
  \begin{align*}
    \E\big[\!\det\!\big((\A\B)_{\Ic,\Jc}\big)\big]
    &= \E\big[\!\det(\A_{\Ic,*}\B_{*,\Jc})\big]\\
    &=\E\bigg[\sum_{S:\,|S|=|\Ic|}\!\!\det\!\big(\A_{\Ic,S}\big)
      \det\!\big(\B_{S,\Jc}\big)\bigg]\\
&=\!\!\sum_{S:\,|S|=|\Ic|}\!\!\det\!\big(\E[\A]_{\Ic,S}\big)
                                                \det\!\big(\E[\B]_{S,\Jc}\big)\\
    &=\det\!\big(\E[\A]_{\Ic,*}\, \E[\B]_{*,\Jc}\big)\\
      &= \det\!\big(\E[\A\B]_{\Ic,\Jc}\big),
  \end{align*}
  where recall that $\A_{\Ic,*}$ denotes the submatrix of $\A$
  consisting of its (entire) rows indexed by $\Ic$.
  \end{proofof}

To prove Lemma \ref{l:poisson}, we will use the following
lemma, many variants of which appeared in the literature
\cite[e.g.,][]{expected-generalized-variance}. We use the one given by
\cite{correcting-bias}.
\begin{lemma}[\citeauthor{correcting-bias}, \citeyear{correcting-bias}]\label{l:cb}
If the rows of random $k\times d$ matrices $\A,\B$
  are sampled as an i.i.d.~sequence of $k\geq d$ pairs of joint random vectors, then
\begin{align}
  k^d\,\E \big[\det(\A^\top\B)\big]
  &= \ktd\,\det\!\big(\E[\A^\top\B]\big).
     \end{align}
 \end{lemma}

\noindent
Here, we use the following standard shorthand: $\ktd =
\frac{k!}{(k-d)!} = k\,(k-1)\dotsm(k-d+1)$. Note that the above result
almost looks like we are claiming that the matrix $\A^\top\B$ is d.p.,
but in fact it is not because $k^d\neq \ktd$. The difference
in those factors is precisely what we are going to correct with the
Poisson random variable. We now present the proof of Lemma
\ref{l:poisson}.
\begin{proofof}{Lemma}{\ref{l:poisson}}
Without loss of generality, it suffices to check Definition \ref{d:main} with both $\Ic$ and
$\Jc$ equal $[d]$. We first expand the expectation by
conditioning on the value of $K$ and letting $\gamma=\E[K]$:
    \begin{align*}
      \E\big[\!\det(\A^\top\B)\big]
      &= \sum_{k=0}^\infty
\E\big[\det(\A^\top\B)\mid K\!=\!k\big]\
\Pr(K\!=\!k)\\
      \text{(Lemma \ref{l:cb})}
      \quad&=
        \sum_{k=d}^\infty\frac{k! k^{-d}}{(k-d)!}\det\!\big(\E[\A^\top\B\mid
        K\!=\!k]\big)
        \frac{\gamma^k\ee^{-\gamma}}{k!}\\
      &=\sum_{k=d}^\infty
\Big(\frac\gamma k\Big)^d\det\!\big(\E[\A^\top\B\mid K\!=\!k]\big)
        \frac{\gamma^{k-d}\ee^{-\gamma}}{(k-d)!}.
     %  \\
     %  &=\sum_{k=0}^\infty \det\!\big(\E[\A^\top\B]\big)\,\Pr(K\!=\!k)
     % \ =\ \det\!\big(\E[\A^\top\B]\big).
    \end{align*}
    Note that $\frac\gamma k\,\E[\A^\top\B\mid K\!=\!k]=\E[\A^\top\B]$,
    which is independent of $k$. Thus we can rewrite the above
    expression as:
    \begin{align*}
\det\!\big(\E[\A^\top\B]\big)\sum_{k=d}^\infty\frac{\gamma^{k-d}\ee^{-\gamma}}{(k-d)!}
      =
      \det\!\big(\E[\A^\top\B]\big)\sum_{k=0}^\infty
      \frac{\gamma^{k}\ee^{-\gamma}}{k!}=\det\!\big(\E[\A^\top\B]\big),
    \end{align*}
    which concludes the proof.
  \end{proofof}

To prove Lemma \ref{l:normalization}, we use the following standard
determinantal formula which is used to derive the normalization
constant of a discrete determinantal point process.
\begin{lemma}[\citeauthor{dpp-ml}, \citeyear{dpp-ml}]\label{l:det-standard}
  For any $k\times d$ matrices $\A,\B$ we have
  \[\det(\I+\A\B^\top)=\sum_{S\subseteq[k]}\det(\A_{S,*}\B_{S,*}^\top).\]
\end{lemma}

\begin{proofof}{Lemma}{\ref{l:normalization}}
By Lemma \ref{l:poisson}, the matrix $\B^\top\A$ is determinant
preserving. Applying Lemma \ref{t:ring} we conclude that
$\I+\B^\top\A$ is also d.p., so
\begin{align*}
  \det\!\big(\I+\E[\B^\top\A]\big) = \E\big[\det(\I+\B^\top\A)\big] =\E\big[\det(\I+\A\B^\top)\big],
\end{align*}
where the second equality is known as Sylvester's Theorem.
We rewrite the expectation of $\det(\I+\A\B^\top)$ by applying Lemma
\ref{l:det-standard}.  Letting $\gamma=\E[K]$, we obtain:
\begin{align*}
\E\big[\det(\I+\A\B^\top)\big]  &=\E\bigg[\sum_{S\subseteq [K]}\E\big[\det(\A_{S,*}\B_{S,*}^\top)\mid
    K\big]\bigg]\\
  &\overset{(*)}{=}\sum_{k=0}^\infty\frac{\gamma^k\ee^{-\gamma}}{k!}
  \sum_{i=0}^k\binom{k}{i} \E\big[\det(\A\B^\top)\mid K=i\big]\\
  &=\sum_{i=0}^\infty \E\big[\det(\A\B^\top)\mid K=i\big]
  \sum_{k\geq i}^\infty \binom{k}{i}
  \frac{\gamma^k\ee^{-\gamma}}{k!}\\
  &=\sum_{i=0}^\infty
    \frac{\gamma^i\ee^{-\gamma}}{i!}\E\big[\det(\A\B^\top)\mid K=i\big]
    \sum_{k\geq i}^\infty\frac{\gamma^{k-i}}{(k-i)!} = \E\big[\det(\A\B^\top)\big]\cdot\ee^\gamma,
\end{align*}
where $(*)$ follows from the exchangeability of the rows of $\A$ and
$\B$, which implies that the distribution of $\A_{S,*}\B_{S,*}^\top$ is the
same for all subsets $S$ of a fixed size $k$.
\end{proofof}

\section{Proof of Theorem \ref{t:mse}}
\label{a:mse-proof}
In this section we use $Z_\mu^n$ to denote the normalization
constant that appears in \eqref{eq:cases} when computing an expectation for surrogate design
$S_\mu^n$.
We first prove Lemma \ref{l:sqinv-all} by splitting the under- and
over-determined cases and start with proving the former.
% Note that those
% results require $\mu$ to satisfy general position (Assumption \ref{a:general-position}),
% which implies that if $\X\sim\mu^k$ for $k\leq d$ then $\rank(\X)=k$.
\begin{lemma}\label{l:sqinv-under}
If  $\Xb\sim S_\mu^n$ for $n<d$, then we have
\begin{align*}
    \E\big[\tr\big((\Xb^\top\Xb)^{\dagger}\big)\big]
    &={\gamma_n}\big(1- \det\!\big((\tfrac1{\gamma_n}\I+\Sigmab_\mu)^{-1}\Sigmab_\mu\big)\big).
\end{align*}
\end{lemma}
\begin{proof}
Let $\X\sim\mu^K$ for $K\sim\Poisson({\gamma_n})$. Note that if
$\det(\X\X^\top)>0$ then using the fact that
$\det(\A)\A^{-1}=\adj(\A)$ for any invertible matrix $\A$, we can write:
  \begin{align*}
    \det(\X\X^\top)\tr\big((\X^\top\X)^{\dagger}\big)
    &= \det(\X\X^\top)\tr\big((\X\X^\top)^{-1}\big) \\
    &= \tr(\adj(\X\X^\top)) \\[-1mm]
    &= \sum_{i=1}^K\det(\X_{-i}\X_{-i}^\top),
  \end{align*}
  where $\X_{-i}$ is a shorthand for $\X_{[K]\backslash\{i\},*}$.
Assumption \ref{a:general-position} ensures that
$\Pr\big\{\det(\X\X^\top)>0\big\}=1$, which allows us to write:
  \begin{align*}
Z_\mu^n\cdot \E\big[\tr\big((\Xb^\top\Xb)^{\dagger}\big)\big]
    &=\E\bigg[
    \sum_{i=1}^K\det(\X_{-i}\X_{-i}^\top)\ \big|\
    \det(\X\X^\top)>0\bigg]\cdot\overbrace{\Pr\big\{\det(\X\X^\top)>0\big\}}^{1}\\
    &=\sum_{k=0}^d\frac{\gamma_n^{k}\ee^{-\gamma_n}}{k!}\E\Big[
      \sum_{i=1}^k\det(\X_{-i}\X_{-i}^\top)\ \big|\  K=k\Big]\\
    &=\sum_{k=0}^d\frac{\gamma_n^{k}\ee^{-\gamma_n}}{k!}\, k\
      \E\big[\det(\X\X^\top)\mid K=k-1\big]\\
    &=\gamma_n\sum_{k=0}^{d-1}\frac{\gamma_n^{k}\ee^{-\gamma_n}}{k!}
      \E\big[\det(\X\X^\top)\mid K=k\big]\\
    &=\gamma_n\Big(\E\big[\det(\X\X^\top)\big]\ -\
      \frac{\gamma_n^{d}\ee^{-\gamma_n}}{d!}\E\big[\det(\X)^2\mid K=d\big]
      \Big) \\
    &\overset{(*)}{=}\gamma_n\big(\ee^{-\gamma_n}\det(\I +\gamma_n\Sigmab_\mu) -
      \ee^{-\gamma_n}\det(\gamma_n\Sigmab_\mu)\big),
  \end{align*}
  where $(*)$ uses Lemma \ref{l:normalization} for the first term and
  Lemma \ref{l:cb} for the second term. We obtain the desired result by
  dividing both sides by
  $Z_\mu^n=\ee^{-\gamma_n}\det(\I+\gamma_n\Sigmab_\mu)$.
\end{proof}
In the over-determined regime, a more general matrix expectation
formula can be shown (omitting the trace). The following result is
related to an expectation formula derived by
\cite{correcting-bias-journal}, however they use a slightly
different determinantal design so the results are incomparable.
\begin{lemma}\label{l:sqinv-over}
If $\Xb\sim S_\mu^n$ and $n>d$, then we
have
\begin{align*}
  \E\big[ (\Xb^\top\Xb)^{\dagger}\big] =
  \Sigmab_\mu^{-1}\cdot \frac{1-\ee^{-\gamma_n}}{\gamma_n}.
\end{align*}
\end{lemma}
\begin{proof}
Let $\X\sim\mu^K$ for $K\sim\Poisson(\gamma_n)$. Assumption
\ref{a:general-position} implies that for $K\neq d-1$ we have
\begin{align}
  \det(\X^\top\X)(\X^\top\X)^\dagger=\adj(\X^\top\X),\label{eq:adj-over}
  \end{align}
however when $k=d-1$ then \eqref{eq:adj-over} does not hold because
$\det(\X^\top\X)=0$ while $\adj(\X^\top\X)$ may be non-zero. It
follows that:
  \begin{align*}
Z_\mu^n\cdot
    \E\big[ (\Xb^\top\Xb)^{\dagger}\big]
    &=\E\big[\det(\X^\top\X)(\X^\top\X)^\dagger\big]\\
    &=\E\big[\adj(\X^\top\X)\big]-
\frac{\gamma_n^{d-1}\ee^{-\gamma_n}}{(d-1)!}
      \E\big[\adj(\X^\top\X)\mid K=d-1\big]\\
    &\overset{(*)}{=}\adj\!\big(\E[\X^\top\X]\big) -
      \frac{\gamma_n^{d-1}\ee^{-\gamma_n}}{(d-1)^{d-1}}
      \adj\!\big(\E[\X^\top\X\mid K=d-1]\big)\\
    &=\adj(\gamma_n\Sigmab_\mu) - \ee^{-\gamma_n}\adj(\gamma_n\Sigmab_\mu)\\
    &=\det(\gamma_n\Sigmab_\mu)\,(\gamma_n\Sigmab_\mu)^{-1}(1-\ee^{-\gamma_n})\\
    &=\det(\gamma_n\Sigmab_\mu)\,\Sigmab_\mu^{-1}\cdot\frac{1-\ee^{-\gamma_n}}{\gamma_n},
  \end{align*}
  where the first term in $(*)$ follows from Lemma
  \ref{l:normalization} and \eqref{eq:adj}, whereas the second term comes
  from Lemma 2.3 of \cite{correcting-bias-journal}.
Dividing both sides by $Z_\mu^n=\det(\gamma_n\Sigmab_\mu)$ completes the proof.
\end{proof}

Applying the closed form expressions from Lemmas
\ref{l:proj} and \ref{l:sqinv-all}, we derive
the formula for the MSE and prove Theorem \ref{t:mse} (we defer the
proof of Lemma \ref{l:proj} to Appendix \ref{s:unbiased-proof}).
\begin{proofof}{Theorem}{\ref{t:mse}}
  First, assume that $n<d$, in which case we have
  $\gamma_n=\frac1{\lambda_n}$ and moreover
  \begin{align*}
    n &= \tr\big(\Sigmab_\mu(\Sigmab_\mu+\lambda_n\I)^{-1}\big)\\
      &=\tr\big((\Sigmab_\mu+\lambda_n\I-\lambda_n\I)(\Sigmab_\mu+\lambda_n\I)^{-1}\big)\\
    &=d - \lambda_n\tr\big((\Sigmab_\mu+\lambda_n\I)^{-1}\big),
    %\frac{\tr\big((\Sigmab_\mu+\lambda_n\I)^{-1}\big)}{d-n},
  \end{align*}
so we can write $\lambda_n$ as $(d-n)/\tr((\Sigmab_\mu+\lambda_n\I)^{-1})$.
  From this and Lemmas \ref{l:proj} and \ref{l:sqinv-under}, we
obtain the desired expression, where recall
  that $\alpha_n = \det\!\big(\Sigmab_\mu (\Sigmab_\mu+\frac1{\gamma_n})^{-1}\big)$:
  \begin{align*}
    \MSE{\Xb^\dagger\ybb} &= \sigma^2\,\gamma_n(1-\alpha_n) +
    \tfrac1{\gamma_n} \,\w^{*\top}(\Sigmab_\mu+\tfrac1{\gamma_n}\I)^{-1}\w^*
    \\
    &\overset{(a)}{=}\sigma^2\,\frac{1-\alpha_n}{\lambda_n} +
    \lambda_n\,\w^{*\top}(\Sigmab_\mu+\lambda_n\I)^{-1}\w^*\\
    &\overset{(b)}{=}\sigma^2\tr\big((\Sigmab_\mu+\lambda_n\I)^{-1}\big)\frac{1-\alpha_n}{d-n}
      +
      (d-n)\frac{\w^{*\top}(\Sigmab_\mu+\lambda_n\I)^{-1}\w^*}
      {\tr\big((\Sigmab_\mu+\lambda_n\I)^{-1}\big)}.
  \end{align*}
  While the expression given after $(a)$ is simpler than the one
after $(b)$, the latter better illustrates how the MSE depends on
the sample size $n$ and the dimension $d$.
  Now, assume that $n>d$. In this case, we have $\gamma_n=n-d$ and apply Lemma
  \ref{l:sqinv-over}:
  \begin{align*}
    \MSE{\Xb^\dagger\ybb}
    &= \sigma^2\,\tr(\Sigmab_\mu^{-1})\,
\frac{1-\ee^{-\gamma_n}}{\gamma_n}
=\sigma^2\,\tr(\Sigmab_\mu^{-1})
\,\frac{1-\beta_n}{n-d}.
  \end{align*}
The case of $n=d$ was shown in Theorem~2.12 of \cite{correcting-bias-journal}.
This concludes the proof.
\end{proofof}

\section{Proof of Theorem \ref{t:unbiased}}
\label{s:unbiased-proof}

As in the previous section, we use $Z_\mu^n$ to denote the normalization
constant that appears in \eqref{eq:cases} when computing an expectation
for surrogate design $S_\mu^n$.
Recall that our goal is to compute the expected value of
$\Xb^\dagger\ybb$ under the surrogate design $S_\mu^n$. Similarly as for Theorem
\ref{t:mse}, the case of $n=d$ was shown in Theorem 2.10 of
\cite{correcting-bias-journal}. We break the rest down into the
under-determined case $(n<d)$ and the over-determined case ($n>d$),
starting with the former. Recall that we do \emph{not} require any
modeling assumptions on the responses.
\begin{lemma}\label{l:ridge-under}
If $\Xb\sim S_\mu^n$ and $n<d$, then for any $y(\cdot)$
such that $\E_{\mu,y}[y(\x)\,\x]$ is well-defined,
denoting $\yb_i$ as $y(\xbb_i)$, we have
\begin{align*}
  \E\big[\Xb^\dagger \ybb\big]
  &=
    \big(\Sigmab_\mu+\tfrac1{\gamma_n}\I\big)^{-1}\E_{\mu,y}[y(\x)\,\x].
\end{align*}
\end{lemma}
\begin{proof}
   Let $\X\sim\mu^K$ for $K\sim\Poisson(\gamma_n)$ and denote
   $y(\x_i)$ as $y_i$.
  Note that when $\det(\X\X^\top)>0$, then
  the $j$th entry of $\X^\dagger\y$ equals
  $\f_j^\top(\X\X^\top)^{-1}\y$, where $\f_j$ is the $j$th
  column of $\X$, so:
\begin{align*}
  \det(\X\X^\top)\,(\X^\dagger\y)_j
  &= \det(\X\X^\top)\, \f_j^\top(\X\X^\top)^{-1}\y \\
  &=
  \det(\X\X^\top+\y\f_j^\top) - \det(\X\X^\top).
\end{align*}
If $\det(\X\X^\top)=0$, then also
$\det(\X\X^\top+\y\f_j^\top)=0$, so we can write:
\begin{align*}
Z_\mu^n\cdot\E\big[(\Xb^\dagger\ybb)_j\big]
  &=  \E\big[\det(\X\X^\top)(\X^\dagger\y)_j\big] \\
  &= \E\big[\det(\X\X^\top+\y\f_j^\top)-\det(\X\X^\top)\big]  \\
  &=\E\big[\det\!\big([\X,\y][\X,\f_j]^\top\big)\big] - \E\big[\det(\X\X^\top)\big]\\
  &\overset{(a)}{=}\ee^{-\gamma_n}\det\!\bigg(\I +
    \gamma_n\,\E_{\mu,y}\bigg[\begin{pmatrix}\x\x^\top&
      \!\!\x\, y(\x)\\ x_j\,\x^\top&\!\!
      x_j\,y(\x)\end{pmatrix}\bigg]\bigg)
                          -\ee^{-\gamma_n}\det(\I+\gamma_n\Sigmab_\mu)\\
  &\overset{(b)}{=}\ee^{-\gamma_n}\det(\I+\gamma_n\Sigmab_\mu) \\
    &\qquad \times \Big(\E_{\mu,y}\big[\gamma_n
    x_j\,y(\x)\big] - \E_{\mu}\big[\gamma_n
    x_j\,\x^\top\big](\I+\gamma_n\Sigmab_\mu)^{-1}\E_{\mu,y}\big[\gamma_n\x\,
    y(\x)\big]\Big),
\end{align*}
where $(a)$ uses Lemma \ref{l:normalization} twice, with the first
application involving two different matrices $\A=[\X,\y]$ and
$\B=[\X,\f_j]$, whereas $(b)$ is a standard determinantal identity
\cite[see Fact 2.14.2 in][]{matrix-mathematics}.
  Dividing both sides by $Z_\mu^n$ and letting $\v_{\mu,y}=\E_{\mu,y}[y(\x)\,\x]$, we obtain that:
  \begin{align*}
    \E\big[\Xb^\dagger\ybb\big]
    &= \gamma_n\v_{\mu,y} - \gamma_n^2\Sigmab_\mu(\I+\gamma_n\Sigmab_\mu)^{-1}\v_{\mu,y}\\
&=\gamma_n\big(\I - \gamma_n\Sigmab_\mu (\I+\gamma_n\Sigmab_\mu)^{-1}\big)\v_{\mu,y}
=\gamma_n(\I+\gamma_n\Sigmab_\mu)^{-1}\v_{\mu,y},
  \end{align*}
  which completes the proof.
\end{proof}
We return to Lemma \ref{l:proj}, regarding the expected orthogonal
projection onto the complement of the row-span of $\Xb$, i.e.,
$\E[\I-\Xb^\dagger\Xb]$, which follows as a corollary of Lemma~\ref{l:ridge-under}.
\begin{proofof}{Lemma}{\ref{l:proj}}
  We let $y(\x)=x_j$ where $j\in[d]$ and apply Lemma
  \ref{l:ridge-under} for each $j$, obtaining:
  \begin{align*}
    \I - \E\big[\Xb^\dagger\Xb] = \I -
    (\Sigmab_\mu+\tfrac1{\gamma_n}\I)^{-1}\Sigmab_\mu,
  \end{align*}
  from which the result follows by simple algebraic manipulation.
\end{proofof}

We move on to the over-determined case, where the ridge regularization
of adding the identity to $\Sigmab_\mu$ vanishes. Recall that we
assume throughout the paper that $\Sigmab_\mu$ is invertible.
\begin{lemma}\label{l:ridge-over}
  If $\Xb\sim S_\mu^n$ and $n>d$, then for any real-valued random function $y(\cdot)$
  such that $\E_{\mu,y}[y(\x)\,\x]$ is well-defined,
denoting $\yb_i$ as $y(\xbb_i)$, we have
 \begin{align*}
  \E\big[\Xb^\dagger \ybb\big]
  &=\Sigmab_\mu^{-1}\E_{\mu,y}\big[y(\x)\,\x\big].
\end{align*}
\end{lemma}
\begin{proof}
   Let $\X\sim\mu^K$ for $K\sim\Poisson(\gamma_n)$ and denote
   $y_i=y(\x_i)$. Similarly as in the proof of
   Lemma~\ref{l:ridge-under}, we note that when $\det(\X^\top\X)>0$,
   then
  the $j$th entry of $\X^\dagger\y$ equals
  $\e_j^\top(\X^\top\X)^{-1}\X^\top\y$, where $\e_j$ is the $j$th
standard basis vector, so:
\begin{align*}
  \det(\X^\top\X)\,(\X^\dagger\y)_j =
  \det(\X^\top\X)\, \e_j^\top(\X^\top\X)^{-1}\X^\top\y =
  \det(\X^\top\X+\X^\top\y\e_j^\top) - \det(\X^\top\X).
\end{align*}
If $\det(\X^\top\X)=0$, then also
$\det(\X^\top\X+\X^\top\y\e_j^\top)=0$. We proceed to compute the
expectation:
\begin{align*}
Z_\mu^n\cdot\E\big[(\Xb^\dagger\ybb)_j\big]
  &=  \E\big[\det(\X^\top\X)(\X^\dagger\y)_j\big] \\
  &= \E\big[\det(\X^\top\X+\X^\top\y\e_j^\top)-\det(\X^\top\X)\big]  \\
  &=\E\big[\det\!\big(\X^\top(\X+\y\e_j^\top)\big)\big] - \E\big[\det(\X^\top\X)\big]\\
  &\overset{(*)}{=}\det\!\Big(
    \gamma_n\,\E_{\mu,y}\big[\x(\x+ y(\x)\e_j)^\top\big]\Big)
    -\det(\gamma_n\Sigmab_\mu)\\
  &=\det\!\big(\gamma_n\Sigmab_\mu + \gamma_n\E_{\mu,y}[\x\,y(\x)]\e_j^\top\big)
    -\det(\gamma_n\Sigmab_\mu)\\
  &=\det(\gamma_n\Sigmab_\mu)\cdot
    \gamma_n\e_j^\top(\gamma_n\Sigmab_\mu)^{-1}\E_{\mu,y}\big[y(\x)\,\x\big],
\end{align*}
where $(*)$ uses Lemma \ref{l:poisson} twice (the first time, with
$\A=\X$ and $\B=\X+\y\e_j^\top$). Dividing both sides by
$Z_\mu^n=\det(\gamma_n\Sigmab_\mu)$ concludes the proof.
\end{proof}

\noindent
We combine Lemmas \ref{l:ridge-under} and \ref{l:ridge-over} to obtain
the proof of Theorem \ref{t:unbiased}.
\begin{proofof}{Theorem}{\ref{t:unbiased}}
The case of $n=d$ follows directly from Theorem~2.10 of
\cite{correcting-bias}.
Assume that $n<d$. Then we have
$\gamma_n=\frac1{\lambda_n}$, so the result follows
from Lemma \ref{l:ridge-under}.
% , noting that unlike in the lemma,
% Theorem \ref{t:unbiased} allows for the response model $y(\cdot)$ to be randomized:
% \begin{align*}
%   \E[\Xb^\dagger\ybb] = \E\big[\E[\Xb^\dagger\ybb\mid\ybb]\big] =
% % \w_{\lambda_n}^* = \argmin_\w\E_\mu\big[(\x^\top\w-y(\x))^2\big] +
% %   \lambda_n\|\w\|^2 =
% %   \big(\Sigmab_\mu+\lambda_n\I\big)^{-1}
% %   \E_{\mu}\big[y(\x)\,\x\big].
% \end{align*}
If $n>d$, then the result follows from Lemma \ref{l:ridge-over}.
\end{proofof}

\section{Proof of Theorem \ref{t:asymptotic}}
\label{sec:proof-of-t-asymptotic}
Once again, we break down the proof into under- and
over-determined cases, starting with the former. Note that in this
case we require that the covariance be equal to identity.
\begin{lemma}\label{l:asymptotic-under}
  Let $\rho=n/d$, $\X\sim \Nc_{n,d}(\zero,\I_n \otimes \I_d)$ and $y_i=y(\x_i)$ under Assumption
  \ref{a:linear}. If $d>n+1$ then
  \begin{align*}
    &0\ \leq\ \frac{\MSE{\X^\dagger\y}-\Mc(\I,
\w^*,\sigma^2,n)}{\Mc(\I,\w^*,\sigma^2,n)}\ \leq\
\frac1d\cdot \frac{1}{1-\rho-\frac1d} + 3\rho^d.
  \end{align*}
\end{lemma}

\begin{proof}
  We first recall the standard decomposition of $\MSE{\X^\dagger \y}$:
  \begin{align*}
    \MSE{\X^\dagger \y}
    &=\sigma^2 \E\big[\tr\big((\X^\top \X)^\dagger \big)\big]+
      \w^{*\top}\!\big( \I - \E[\X^\dagger \X]\big) \w^*.
  \end{align*}
  Since the rows of $\X$ are standard normal random variables,
  $\X\X^\top$ is an $n\times n$ Wishart random matrix with $d>n+1$ degrees
  of freedom. Using the formula for the mean of the Inverse-Wishart
  distribution, it follows that
  \begin{align*}
\E\big[\tr((\X^\top\X)^\dagger)\big] = \E\big[\tr((\X\X^\top)^{-1})\big]= \frac{n}{d - n - 1}.
  \end{align*}
  % and therefore
  % \begin{align*}
  %   \MSE{\X^\dagger \y}
  %   &= \|\w\|_2^2 - \w^\top \E[\X^\dagger \X] \w
  %   + \frac{\sigma^2 n}{d - n - 1}
  % \end{align*}
  % Let $\X = \U \S \V^\top$ be its full singular value decomposition. Notice
  % \begin{align*}
  %   \X^\dagger \X
  %   &= \V \S^\dagger \U^\top \U \S \V^\top \\
  %   &= \V \S^\dagger \S \V^\top \\
  %   &= \V \begin{bmatrix}
  %     \I_n & 0 \\
  %     0 & 0
  %   \end{bmatrix} \V^\top \\
  %   &= \V_n \V_n^\top \\
  %   \w^\top \E[\X^\dagger \X] \w
  %   &= \E[\|\V_n^\top \w\|_2^2]
  % \end{align*}
  % where $\V_n$ denotes the $d \times n$ matrix with columns equal to the
  % $n$ eigenvectors of $\X^\top \X$ with non-zero eigenvalue.
  % But since $\Sigmab = \I$, by rotation invariance
  % $\V_n^\top$ is a change of basis onto a uniformly random
Note that $\X^\dagger\X$ is a uniformly random projection matrix so by
rotational symmetry it follows that
\begin{align*}
  \w^{*\top}\E\big[\X^\dagger\X]\w^* =
  \E\big[\|\X^\dagger\X\w^*\|^2\big] = \frac nd \,\|\w^*\|^2.
\end{align*}
Putting this together we obtain that
  \begin{align*}
    \MSE{\X^\dagger \y}
    &=\frac{\sigma^2 n}{d - n - 1}+ \|\w^*\|^2\,\frac{d - n}{d}.
  \end{align*}
  On the other hand, the surrogate MSE expression can be derived by
observing that for $\Sigmab=\I$ we have
$\tr((\Sigmab+\lambda_n\I)^{-1})=d/(1+\lambda_n)=n$ (see definition of
$\lambda_n$ in Theorem \ref{t:mse}):
  \begin{align*}
    \Mc(\I, \w^*,\sigma^2,n) =
    \sigma^2n\cdot\frac{1-\alpha_n}{d-n} + \|\w^*\|^2\,\frac{d-n}{d}.
  \end{align*}
  Note that the second term is the same in both cases, even though
  this may not be true for non-isotropic Gaussians. We now compute the
  normalized difference between the
  expressions,
  \begin{align*}
 \frac{\MSE{\X^\dagger\y}-\Mc(\I, \w^*,\sigma^2,n)}{\Mc(\I,
    \w^*,\sigma^2,n)}
    &= \frac{\sigma^2n\cdot(\frac1{d-n-1} -
    \frac{1-\alpha_n}{d-n})}{\sigma^2n\cdot\frac{1-\alpha_n}{d-n} +
    \|\w^*\|^2\,\frac{d-n}{d}}\\
&\leq \frac {d-n}{1-\alpha_n}
\Big(\frac1{d-n-1}-\frac{1-\alpha_n}{d-n}\Big)\\
    &=\frac{d-n}{d-n-1}\,\frac{1}{1-\alpha_n}- 1\\
    &=\frac1{d-n-1} + \frac{\alpha_n}{1-\alpha_n}\Big(1+\frac1{d-n-1}\Big).
  \end{align*}
Let $\rho=n/d$. Then $\frac1{d-n-1} = \frac1d\cdot
\frac1{1-\rho-\frac1d}$ and moreover
$\alpha_n = (\frac1{1+\lambda_n})^d=(\frac
nd)^d=\rho^d$. From the assumption that $d>n+1$, we
conclude that $\alpha_n \leq (\frac{d-2}{d})^d\leq \ee^{-2}$
so that $\frac{\alpha_n}{1-\alpha_n}(1+\frac1{d-n-1})\leq
  \frac{2 \alpha_n}{1-\ee^{-2}}\leq 3\rho^d$. This shows the right-hand-side inequality of the
  theorem. That fact that $\MSE{\X^\dagger\y}\geq\Mc(\I,
  \w^*,\sigma^2,n)$ follows easily.
\end{proof}
\begin{lemma}\label{l:asymptotic-over}
  Let $\rho=n/d$, $\X\sim \Nc_{n,d}(\zero,\I_n \otimes \Sigmab)$ and $y_i=y(\x_i)$ under Assumption
  \ref{a:linear}. If $n>d+1$ then
  \begin{align*}
    &0\ \leq\ \frac{\MSE{\X^\dagger\y}-\Mc(\Sigmab,
\w^*,\sigma^2,n)}{\Mc(\Sigmab,\w^*,\sigma^2,n)}\ \leq\
\frac1d\cdot \frac{1}{\rho-1-\frac1d} + 3(\ee^{1-\rho})^d.
  \end{align*}
\end{lemma}
\begin{proof}
The MSE for the over-determined Gaussian design can be derived by
using the formula for the mean of the Inverse-Wishart distribution:
\begin{align*}
  \MSE{\X^\dagger\y}=\sigma^2\tr\big(\E[(\X^\top\X)^{-1}]\big) =
  \frac{\sigma^2\tr(\Sigmab^{-1}) }{n-d-1}.
\end{align*}
To compute the normalized difference we follow similar derivations as
in the proof of Lemma \ref{l:asymptotic-under}:
\begin{align*}
 \frac{\MSE{\X^\dagger\y}-\Mc(\Sigmab, \w^*,\sigma^2,n)}{\Mc(\Sigmab,
    \w^*,\sigma^2,n)}
    &=  \frac {n-d}{1-\beta_n}
      \Big(\frac1{n-d-1}-\frac{1-\beta_n}{n-d}\Big)\\
  &\leq\frac1d\cdot\frac1{\rho-1-\frac1d} + \frac{2\beta_n}{1-\beta_n}.
\end{align*}
Recall that $\beta_n = \ee^{d-n} = (\ee^{1-\rho})^d$ and for $n-d\geq
2$ we have $\frac2{1-\beta_n}\leq 3$. The desired inequalities follow immediately.
\end{proof}
Theorem \ref{t:asymptotic} now follows as a consequence of
Lemmas \ref{l:asymptotic-under} and \ref{l:asymptotic-over}.
\begin{proofof}{Theorem}{\ref{t:asymptotic}}
  Since $\frac1d\leq \frac13|1-\rho|$ and $\ee^{-|1-\rho|d}\leq
  \frac1{2d|1-\rho|}$, it follows that
  \begin{align*}
    \frac1d\cdot \frac{1}{|1-\rho|-\frac1d} + 3(\ee^{-|1-\rho|})^d
    &\leq \frac1d\cdot \frac{1}{|1-\rho|-\frac13|1-\rho|} +
      \frac1d\cdot\frac{3}{2|1-\rho|}=\frac {c_\rho}{d}.
  \end{align*}
  The case of $n>d+1$ now follows from Lemma
  \ref{l:asymptotic-over}. Also, since $\rho\leq \ee^{\rho-1}$, for
$n<d-1$ we have $3\rho^d\leq 3(\ee^{-|1-\rho|})^{d}$, so the same bound follows
from Lemma \ref{l:asymptotic-under}.
\end{proofof}

\section{Empirical analysis of the asymptotic consistency conjectures}
\label{sec:asymp-conj-details}

Conjectures \ref{c:wishart} and \ref{c:projection} are related to open problems
which have been extensively studied in the literature.
With respect to Conjecture~\ref{c:wishart}, \cite{srivastava2003} first derived the probability
density function of the pseudo-Wishart
distribution (also called the singular Wishart), and \cite{cook2011}
computed the first and second moments of 
generalized inverses for the distribution. However, for the
Moore-Penrose inverse and arbitrary covariance $\Sigmab$,
\cite{cook2011} claims that the quantities required
to express the mean ``do not have tractable closed-form representation.''
%
Conjecture~\ref{c:projection} has connections to directional statistics.
Using the SVD, we have the equivalent representation $\X^\dagger \X = \V \V^\top$
where $\V$ is an element of the Stiefel manifold $V_{n,d}$ (i.e., orthonormal
$n$-frames in $\R^d$).
The distribution of $\V$ is known as the matrix angular central
Gaussian (MACG) distribution \citep{chikuse1990matrix}. While prior work
has considered high dimensional limit theorems \citep{CHIKUSE1991145}
as well as density estimation and hypothesis testing \citep{CHIKUSE1998188}
on $V_{n,d}$, they only analyzed the invariant measure
(which corresponds in our setting to $\Sigmab = \I$),
and to our knowledge a closed form expression of $\E[\V\V^\top]$ where
$\V$ is distributed according to MACG with
arbitrary $\Sigmab$ remains an open question.

For verifying these two conjectures, it suffices to only consider
diagonal covariance matrices $\Sigmab$.  This is because if $\Sigmab = \Q \D
\Q^\top$ is its eigendecomposition and $\X\sim\Nc_{n,d}(\zero, \I_n \otimes \Q\D\Q^\top)$, then
we have for $\W \sim \Pc\Wc(\Sigmab, n)$ that $\W
\overset{d}{=} \X^\top \X$ and hence, defining
$\Xt\sim\Nc_{n,d}(\zero,\I_n \otimes \D)$, by linearity and unitary invariance of trace,
\begin{align*}
  \E[\tr(\W^\dagger)]
  &= \tr\big( \E[(\X^\top\X)^\dagger] \big)
  = \tr\Big( \Q\E\big[(\Xt^\top\Xt)^\dagger\big]\Q^\top \Big)
  = \tr\Big( \E\big[(\Xt^\top\Xt)^\dagger\big] \Big)
  = \E\left[\tr \big((\Xt^\top\Xt)^\dagger\big) \right].
\end{align*}
Similarly, we have that
$\E[\X^\dagger\X]=\Q\E\big[\Xt^\dagger\Xt\big]\Q^\top$, and a simple
calculation shows that the expression in Conjecture~\ref{c:projection}
is also independent of the choice of matrix $\Q$.

\subsection{Empirical analysis of asymptotic consistency}
\label{s:empirical}

Theorem \ref{t:asymptotic} states that our surrogate expressions for the MSE under certain Gaussian designs are asymptotically consistent with the multiplicative error rate of $O(1/d)$. 
In this section, we show strong empirical evidence that this fact extends to
the setting not covered by the theorem: under-determined regime (i.e.,
$n<d$) with a non-isotropic Gaussian distribution (i.e., $\Sigmab\neq
\I$). We break down our analysis into verifying two
conjectures which are of independence interest to multivariate
Gaussian analysis. The first conjecture addresses the variance term in
the MSE and postulates an asymptotically
consistent formula for the expected Moore-Penrose pseudo-inverse
of the singular Wishart distribution. Recall that matrix
$\W\sim\Wc(\Sigmab,n)$ is distributed according to the Wishart
distribution with $n$ degrees of freedom if it can be decomposed as
$\W=\X^\top\X$, where $\X\sim\Nc(\zero,\Sigmab)^n$.
\begin{conjecture}[Pseudo-inverse of singular Wishart]\label{c:wishart}
  Fix $n/d<1$ and let $\W\sim\Wc(\Sigmab,n )$, where
$\Sigmab$ is $d\times d$ positive definite with condition number bounded by a constant.
  Then:
\begin{align}
\bigg|\frac{\E\big[\tr(\W^\dagger)\big]}{\Vc(\Sigmab,n)} -1\bigg|=
  O(1/d)\qquad\text{for}\quad\Vc(\Sigmab,n)=\tr\big((\Sigmab+\lambda_n\I)^{-1}\big)\frac{1-\alpha_n}{d-n},\label{eq:wishart}
\end{align}
where $\lambda_n\geq 0$ satisfies $n=\tr(\Sigmab(\Sigmab+\lambda_n\I)^{-1})$ and
$\alpha_n=\det(\Sigmab(\Sigmab+\lambda_n\I)^{-1})$.
\end{conjecture}
Our second conjecture involves the projection onto the
orthogonal complement of a Gaussian sample $\X$, i.e., the matrix
$\I-\X^\dagger\X$, and addresses the bias term in the MSE.
\begin{conjecture}[Gaussian orthogonal projection]\label{c:projection}
  Fix $n/d<1$ and let $\X\sim \Nc(\zero,\Sigmab)^n$, where
$\Sigmab$ is $d\times d$ positive definite with condition number bounded by a constant. Then:
\begin{align}
\sup_{\w\in\R^d\backslash\{\zero\}}\bigg|\frac{\w^\top\E[\I-\X^\dagger\X]\w}{\w^\top
  \Bc(\Sigmab,n)\w} - 1\bigg| = O(1/d)\qquad\text{for}\quad
  \Bc(\Sigmab,n) =
  (\Sigmab+\lambda_n\I)^{-1}\frac{d-n}{\tr((\Sigmab+\lambda_n\I)^{-1})},\label{eq:projection}  
\end{align}
where $\lambda_n\geq 0$ satisfies $n=\tr(\Sigmab(\Sigmab+\lambda_n\I)^{-1})$.
\end{conjecture}
Note that the surrogate expression for the mean squared
error can be written as:
\[
\Mc(\Sigmab_\mu,\w^*,\sigma^2,n)=\sigma^2 \Vc(\Sigmab_\mu,n)
+ \w^{*\top}\Bc(\Sigmab_\mu,n)\w^*.
\]
So, if the conjectures are true,
this would immediately imply that the asymptotic consistency claim
given in Theorem \ref{t:asymptotic} for
the surrogate Gaussian MSE extends to
the under-determined setting with arbitrary covariance $\Sigmab$.

Furthermore, both of these conjectures are related to open problems
which have been extensively studied in the literature. 
With respect to Conjecture~\ref{c:wishart}, \cite{srivastava2003} first derived the probability
density function of a singular Wishart
distribution, and \cite{cook2011} computed the first and second moments of
generalized inverses of a singular Wishart distribution. However, for the
Moore-Penrose pseudo-inverse and arbitrary covariance $\Sigmab$,
\cite{cook2011} claims that the quantities required
to express the mean ``do not have tractable closed-form representation.''
%
Conjecture~\ref{c:projection} has connections to directional statistics.
Using the SVD, we have the equivalent representation $\X^\dagger \X = \V \V^\top$
where $\V$ is an element of the Stiefel manifold $V_{n,d}$ (i.e., orthonormal
$n$-frames in $\R^d$).
The distribution of $\V$ is known as the matrix angular central
Gaussian (MACG) distribution \cite{chikuse1990matrix}. While prior work
has considered high dimensional limit theorems \cite{CHIKUSE1991145}
as well as density estimation and hypothesis testing \cite{CHIKUSE1998188}
on $V_{n,d}$, they only analyzed the invariant measure
(which corresponds in our setting to $\Sigmab = \I$),
and to our knowledge a closed form expression of $\E[\V\V^\top]$ where
$\V$ is distributed according to MACG with
arbitrary $\Sigmab$ remains an open question.

For verifying these two conjectures, it suffices to only consider
diagonal covariance matrices $\Sigmab$.  This is because if $\Sigmab = \Q \D
\Q^\top$ is its eigendecomposition and $\X\sim\Nc(\zero, \Q\D\Q^\top)^n$, then
we have for $\W \sim \Wc(\Sigmab, n)$ that $\W
\overset{d}{=} \X^\top \X$ and hence, defining
$\Xt\sim\Nc(\zero,\D)^n$, by linearity and unitary invariance of trace,
\begin{align*}
  \E[\tr(\W^\dagger)]
  &= \tr\big( \E[(\X^\top\X)^\dagger] \big)
  = \tr\Big( \Q\E\big[(\Xt^\top\Xt)^\dagger\big]\Q^\top \Big)
  = \tr\Big( \E\big[(\Xt^\top\Xt)^\dagger\big] \Big)
  = \E\left[\tr \big((\Xt^\top\Xt)^\dagger\big) \right].
\end{align*}
Similarly, we have that
$\E[\X^\dagger\X]=\Q\E\big[\Xt^\dagger\Xt\big]\Q^\top$, and a simple
calculation shows that the expression in Conjecture~\ref{c:projection}
is also independent of the choice of matrix $\Q$.
Thus, we empirically validate our conjectures for diagonal matrices $\Sigmab$
with several different eigenvalue decay profiles. Denoting
$\lambda_1,...,\lambda_d$ as the eigenvalues of $\Sigmab$, we consider
the following decays:
\begin{itemize}
  \item \texttt{diag\_linear}: linear decay, $\lambda_{i} = b-a i$;
  \item \texttt{diag\_exp}: exponential decay, $\lambda_{i} = b\,10^{-
      a i} $;
  \item \texttt{diag\_poly}: fixed-degree polynomial decay, $\lambda_{i} = (b-a i)^2$;
  \item \texttt{diag\_poly\_2}: variable-degree polynomial decay, $\lambda_i = b i^{-a}$.
\end{itemize}
The constants $a$ and $b$ are chosen to ensure $\lambda_{\text{max}}(\Sigmab) = 1$ and
$\lambda_{\text{min}}(\Sigmab) = 10^{-4}$ (i.e., the condition number
$\kappa(\Sigmab) = 10^{4}$ remains constant).
Figure~\ref{fig:eig-decays} illustrates an example of these decay profiles
for $d=100$.

\begin{figure}[H]
    \includegraphics[width=\textwidth]{continuous_figures/decays.pdf}
  \caption{Scree-plots of $\Sigmab$ for the eigenvalue decays examined
    in our empirical valuations. Here $d=100$ for visualization, whereas
    our experiments increase $d$ while preserving the ratio $n/d$ and
    the decay profile,
    with $\lambda_{\text{max}}(\Sigmab) = 1$ to
    $\lambda_{\text{min}}(\Sigmab) = 10^{-4}$.}
  \label{fig:eig-decays}
\end{figure}


We verify our conjectures by incrasing $d$ while keeping the aspect ratio
$n/d$ fixed and examining the rate of decay of the quantities asserted
in the conjectures. As no closed form expressions are available for
the expectations in the conjectures, we estimate $\E\big[\tr(\W^\dagger)\big]$ (for
Conjecture~\ref{c:wishart}) and $\E[\I-\X^\dagger\X]$ (for
Conjecture~\ref{c:projection}) through Monte Carlo sampling. To ensure that
estimation noise is sufficiently small, we continually increase the number
of Monte Carlo samples until the bootstrap confidence intervals are
within $\pm 12.5\%$ of the quantities in \eqref{eq:wishart} and
\eqref{eq:projection}. We found that while
Conjecture~\ref{c:wishart} required a relatively small number of
trials (up to one thousand),
estimation noise was much larger in Conjecture~\ref{c:projection} and
necessitated over two million trials to obtain good estimates
near $d=100$.

% For Conjecture~\ref{c:wishart}, sampling $\W\sim\Wc(\Sigmab,n)$ is
% elementary and the only nontrivial aspect of computing $\Vc(\Sigmab, n)$
% is solving for $\lambda_n$ such that
% $n = \tr(\Sigmab(\Sigmab + \lambda_n \I)^{-1})$. But notice that
% if we let $\tau_i > 0$ be the eigenvalues of $\Sigmab \succ 0$,
% then
% \begin{align*}
%   \tr(\Sigmab(\Sigmab + \lambda \I)^{-1})
%   &= \sum_{i=1}^d \frac{\tau_i}{\tau_i + \lambda}
% \end{align*}
% which is equal to $d > n$ when $\lambda = 0$ and monotonically decreasing in
% $\lambda$. Hence, $\lambda_n$ is unique and can be found using bisection.

\begin{figure}[h]
    \includegraphics[width=\textwidth]{continuous_figures/wishart_var.pdf}
  \caption{
    Empirical verification of Conjecture~\ref{c:wishart}. We show
    the quantity $\big|\E[\tr(\W^\dagger)]\,\Vc(\Sigmab,n)^{-1} -1\big|$
    as $d$ increases for various aspect ratios $n/d$. Consistent with our conjecture,
    a $O(1/d)$ decay (linear with slope $-1$ on a log-log plot) is
    exhibited across all eigenvalue decay profiles and
    aspect ratios investigated.
  }
  \label{f:conj-wishart}
\end{figure}

The results of empirically validating Conjecture~\ref{c:wishart} are
illustrated in Figure~\ref{f:conj-wishart}, where we performed
Monte Carlo estimation of $\E\big[\tr(\W^\dagger)\big]$ and
plot $\big|\E[\tr(\W^\dagger)]\,\Vc(\Sigmab,n)^{-1} -1\big|$
as $d$ increases from $10$ to $1000$, across a range of aspect ratios $n/d$ and eigenvalue
decay profiles for $\Sigmab$. Confidence intervals are estimated by
bootstrapping. We observe that on log-log axes all of the plots are decreasing with
a linear $-1$ slope, consistent with the $O(1/d)$ rate predicted
by Conjecture~\ref{c:wishart}.


Conjecture \ref{c:projection} is handled similarly, by sampling
$\X\sim\mu^n$ where $\mu=\Nc(\zero,\Sigmab )$ to obtain a Monte Carlo estimate of
$\E[\I-\X^\dagger\X]$. To handle the supremum over $\w$, notice that
we can rewrite \eqref{eq:projection} as a spectral norm
\begin{align}
  \sup_{\w\in\R^d\backslash\{\zero\}}\bigg|\frac{\w^\top\E[\I-\X^\dagger\X]\w}{\w^\top
  \Bc(\Sigmab,n)\w} - 1\bigg|
\  =\ \big\|\E[\I-\X^\dagger\X]\Bc(\Sigmab,n)^{-1}
  - \I\big\|.
  \label{eq:proj-to-operator-norm}
\end{align}
Confidence intervals can now be constructed
using existing methods for constructing operator norm
confidence intervals, and our results use the bootstrapping
method described in \cite{lopes2019bootstrapping}.

Figure~\ref{f:conj-bias} shows
how $\big\|\E[\I-\X^\dagger\X]\Bc(\Sigmab,n)^{-1} - \I\big\|$ decays as we
hold the aspect ratio $n/d$ fixed and increase $d$ between $10$ and
$100$ across the listed
eigenvalue decay profiles and aspect ratios. Again, we observe on
log-log axes a linear decay with slope $-1$ consistent with the $O(1/d)$
rate posed by Conjecture~\ref{c:projection}. Note that the range of
$d$ is smaller than in Figure \ref{f:conj-wishart} because the large
number of Monte Carlo samples (up to two million) required for this
experiment made the computations much more expensive.

\begin{figure}[H]
  \begin{center}
    \includegraphics[width=\textwidth]{continuous_figures/proj_bias.pdf}
  \end{center}
  \caption{
    Empirical results validating Conjecture~\ref{c:projection}.
We show how
    $\big\|\E[\I-\X^\dagger\X]\Bc(\Sigmab,n)^{-1} - \I\big\|$
    (which by Equation~\ref{eq:proj-to-operator-norm}
    is equal to the quantity controlled by Conjecture~\ref{c:projection})
    decays as $d$ increases and observe a linear slope consistent with the conjectured $O(1/d)$ rate.
  }
  \label{f:conj-bias}
\end{figure}

