\begin{figure*}
    \centering
    \hspace{-0.35cm}
    \includegraphics[width=\textwidth]{Figures/mg_combined.pdf}
    % \includegraphics[width=0.51\textwidth]{bayesian_figures/objectives-mg_scale-full.pdf}\hspace{-2mm}\nobreak\includegraphics[width=0.51\linewidth]{bayesian_figures/ratios-mg_scale.pdf}
    % \includegraphics[width=0.49\textwidth]{bayesian_figures/objectives-mg_scale-zoom.pdf}
    \caption{(left) A-optimality value obtained by the various methods on
        the \texttt{mg\_scale} dataset \citep{libsvm} with
        prior precision $\A = 10^{-5}\, \I$,\quad (right)
        A-optimality value for our method (with and without SDP) divided by
        $f_{\A}(\frac kn\Sigmab_{\X})$, the baseline estimate suggested by Theorem \ref{t:q1}.}
    \label{f:experiments}
\end{figure*}

\section{Experiments}\label{s:bayesian:experiments}
We confirm our theoreticala
results with experiments on real world data from \texttt{libsvm} datasets
\citep{libsvm} (more details in Appendix~\ref{a:experiments}).
For all our experiments, the prior precision matrix is set to $\A = n^{-1} \I$
and we consider sample sizes $k \in [d, 5d]$. Each experiment is
averaged over 25 trials and bootstrap 95\% confidence intervals are shown.
% We provide open source implementations for efficient RDPP sampling
% and reproducing our results at \Red{TODO: fill in after blind review}.
The quality of our method, as measured by the A-optimality
criterion,
\[f_{\A}(\X_S^\top \X_S) = \tr \left((\X_S^\top \X_S + \A)^{-1}\right),\]
is compared against several baselines and recently proposed methods
for A-optimal design that have been shown to perform well in
practice. Note that none of these algorithms come with theoretical
guarantees as strong as those offered by our approach. The list of
implemented methods is as follows:
\begin{description}
    \item[Our method (with SDP)] uses the efficient algorithms
        developed in proving Theorem~\ref{t:q2} to sample
        $\DPPreg{p}(\X,\A)$ constrained to subset size $k$
        with $p = w^*$, see \eqref{eq:sdp},
        obtained using a         recently developed first order convex cone solver called Splitting
        Conical Solver \citep[SCS, see][]{o2016conic}.
        We chose SCS because it can handle the SDP constraints in
        \eqref{eq:sdp} and has provable termination guarantees, while
        also finding solutions faster \citep{o2016conic} than alternative
        off-the-shelf optimization software libraries such as SDPT3
        and Sedumi.

    \item[Our method (without SDP)] samples $\DPPreg{p}(\X,\A)$ with uniform
        probabilities $p \equiv \frac{k}{n}$.
        % Compared to uniform and predictive length
        %   sampling, there is an additional DPP sampling overhead which is independent
        %   of sample size $k$ (but does depend on $d$).

    \item[Greedy bottom-up] adds an index $i \in [n]$ to the sample $S$
        maximizing the increase in A-optimality criterion
        \citep{greedy-supermodular,chamon2017approximate}.
        %  This method posesses well-understood theoretical bounds
        %  (however they
        % require additional assumptions on the matrix $\X$) and has strong
        %  empirical performance \cite{tractable-experimental-design}.

    \item[Uniform] samples every size $k$ subset $S \subseteq [n]$
        with equal probability.

    \item[Predictive length] sampling \citep{zhu2015optimal} samples
        each row $\x_i$ of $\X$ with probability $\propto\|\x_i\|$.
\end{description}


Figure~\ref{f:experiments} reveals that our method (without SDP) is superior
to both uniform and predictive length sampling, producing designs which
achieve lower $A$-optimality criteria values for all sample sizes.
As Theorem~\ref{t:algorithm} shows that our method (without SDP) only differs
from uniform sampling by an additional DPP sample with controlled
expected size (see Lemma~\ref{l:size}), we may conclude
that adding even a small DPP sample can improve a uniformly sampled design.

Consistent with prior observations
\citep{tractable-experimental-design,chamon2017approximate}, the greedy bottom up
method achieves surprisingly good performance, despite the limited
theoretical guarantees it offers. However, if our method is used
in conjunction with an SDP solution, then we are able to match and
even slightly exceed the performance of the greedy bottom up
method. Furthermore, the overall run-time costs (see Appendix~\ref{a:experiments})
between the two are comparable. As the majority of the runtime of our
method (with SDP) is occupied by solving the SDP, an interesting future direction
is to investigate alternative solvers such as interior point methods as well
as terminating the solvers early once an approximate solution is reached.

Figure~\ref{f:experiments} (right) numerically evaluates the tightness of the
bound obtained in Theorem \ref{t:q1} by plotting the ratio:
\[\frac{f_{\A}(\X_S^\top \X_S)}{ f_{\A}(\frac{k}{n}\Sigmab_\X)}\]
for subsets returned by our method (with and without SDP). Note that the
line for our method with SDP on
Figure~\ref{f:experiments} (right) shows that the ratio never goes
below 0.5, and we saw similar behavior across all examined datasets
(see Appendix~\ref{a:experiments}). This evidence suggests that for
many real datasets $\opt$ is within only a small constant factor
away from $f_{\A}(\frac{k}{n}\Sigmab_\X)$, matching the upper bound of
Theorem~\ref{t:q1}.
% As sample size $k \to n$, this should converge towards $1$ because $\X_S^\top
% \X_S \to \X^\top \X = \Sigma_X$.
% Across all datasets we examined, we see that this ratio is close to
% its asymptotic value of $1$. Similar to how condition
% numbers are the proper problem dependent constants in matrix inversion
% \cite{nocedal2006numerical}
% and $\tr (\frac{k}{n} \X^\top \X)^{-1}$
% is the problem-dependent constant which appear in results on non-Bayesian
% experimental design \feynman{Citation for this COLT paper?}, our results here
% suggest that $f_{\A}\left(\frac{k}{n} \X^\top \X\right)$
% is the proper problem-dependent constant for Bayesian optimal experimental
% design with a $k$ cardinality constraint.
% \feynman{Does this make sense?}


% \section{Additional details for the experiments}
% \label{a:experiments}
% This section presents additional details and experimental results omitted from
% the main body of the paper. 
In addition to the \texttt{mg\_scale} dataset presented in
Section~\ref{s:experiments}, we also benchmarked on three other data sets
described in Table~\ref{tab:libsvm-datasets}.
\begin{table}[ht]
  \centering
  \caption{Datasets used in the experiments \citep{libsvm}.}
  \label{tab:libsvm-datasets}
  \begin{tabular}[t]{lcccc}
    \toprule
        & \texttt{mg\_scale} & \texttt{bodyfat\_scale} & \texttt{mpg\_scale} & \texttt{housing\_scale} \\
    \midrule
    $n$ & 1385               & 252                     & 392                 & 506                     \\
    $d$ & 6                  & 14                      & 7                   & 13                      \\
    \bottomrule
  \end{tabular}
\end{table}

The A-optimality values obtained are illustrated in Figure~\ref{f:obj-grid}.
The general trend observed in Section~\ref{s:experiments} of our method
(without SDP) outperforming independent sampling methods (uniform and
predictive length) and our method (with SDP) matching the performance of the
greedy bottom up method continues to hold across the additional datasets considered.

\begin{figure}[htpb]
  \centering
  \includegraphics[width=\textwidth]{design/bayesian_figures/obj_grid.pdf}
  \caption{A-optimality values achieved by the methods compared. In all cases
    considered, we found our method (without SDP) to be superior to independent
    sampling methods like uniform and predictive length sampling. After paying the price
    to solve an SDP, our method (with SDP) is able to consistently match the performance
    of a greedy method which has been noted
    \citep{chamon2017approximate} to work well empirically.}
  \label{f:obj-grid}
\end{figure}

The relative ranking and overall order of magnitude differences
between runtimes (Figure~\ref{f:runtimes}) are also similar across the various
datasets. An exception to the rule is on $\texttt{mg\_scale}$, where we see
that our method (without SDP) costs more than the greedy method
(whereas everywhere else it costs~less).

\begin{figure}[htpb]
  \centering
  \includegraphics[width=\textwidth]{design/bayesian_figures/runtime_grid.pdf}
  \caption{Runtimes of the methods compared. Our method (without SDP) is
    within an order of magnitude of greedy bottom up and faster in 3 out of
    4 cases. The gap between our method with and without SDP is
    attributable to the SDP solver, making investigation of more efficient
    solvers and approximate solutions an interesting direction for future
    work.
  }
  \label{f:runtimes}
\end{figure}

The claim that $f_{\A}(\frac{k}{n} \Sigmab_\X)$ is an appropriate
quantity to summarize the contribution of problem-dependent factors
on the performance of Bayesian A-optimal designs is further evidenced in Figure~\ref{f:ratios}.
Here, we see that after normalizing the A-optimality values by this
quantity, the remaining quantities are all on the same scale and close to $1$.

\begin{figure}[htpb]
  \centering
  \includegraphics[width=\textwidth]{design/bayesian_figures/ratios_grid.pdf}
  \caption{The ratio controlled by Lemma~\ref{l:guarantees}. This ratio converges
    to $1$ as $k \to n$ and is close to $1$ across all
    real world datasets,
    suggesting that $f_{\A}(\frac{k}{n} \Sigmab_\X)$
    is an appropriate problem-dependent scale for Bayesian A-optimal
    experimental design.
  }
  \label{f:ratios}
\end{figure}

% The choice of $\lambda=43$ is motivated by keeping 95\% of the spectral
% mass. Figure~\ref{fig:scree} shows a scree plot for \texttt{mpg\_scale}
% after applying a polynomial kernel, where we see that the majority
% of the spectral mass is contained in the top few eigenvalues.

% \begin{figure}[htbp]
%     \centering
%     % \hspace{-0.55cm}
%     \includegraphics[width=0.5\textwidth]{Figures/screeplot.pdf}
%     \caption{
%         To determine the amount of regularization $\lambda$, a scree
%         plot of the eigenvalues of the polynomially expanded
%         covariance matrix $\Phi(X)^\top \Phi(X)$.
%     }
%     \label{fig:scree}
% \end{figure}

