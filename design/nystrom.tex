\documentclass[12pt]{sty/colt2019/colt2018-arxiv}
% \else
% \documentclass[anon,12pt]{sty/colt2019/colt2019} % Anonymized submission
% \documentclass[12pt]{colt2019} % Include author names
% \fi

% The following packages will be automatically loaded:
% amsmath, amssymb, natbib, graphicx, url, algorithm2e

% \title[Short Title]{Full Title of Article}
\title{Row subset selection and the Nystr\"om method}
\usepackage{times}
 

\usepackage{color}
\usepackage{cancel}
\usepackage{wrapfig}
\usepackage{caption}
\usepackage{algorithm}
\usepackage{xfrac}
\usepackage{algorithmic}
\def\opt{{\textsc{OPT}_k}}
\def\const{{\mathrm{const}}}
\def\nnz{{\mathrm{nnz}}}
\def\r{\sfrac{\sigma_{\w}^2}{\sigma_{\xib}^2}}
\def\rm{\sfrac{\sigma_{\xib}^2}{\sigma_{\w}^2}}
\def\cmark{\Green{\checkmark}}
\def\xmark{\Red{\large\sffamily x}}
\newcommand{\pdet}{{\mathrm{pdet}}}
\newcommand{\MSPE}[1] {{\mathrm{MSPE}\big[#1\big]}}
\newcommand{\MSE}[1] {{\mathrm{MSE}\big[#1\big]}}
\def\Poisson{{\operatorname{Poisson}}}
\def\PB{{\operatorname{PB}}}
\newcommand{\DP}[1]{\mathcal{DP}^{#1}}
\def\Ic{\mathcal{I}}
\def\Jc{\mathcal{J}}
\def\Mc{\mathcal M}
\def\Ec{\mathcal E}
\def\sr{{\mathrm{sr}}}
\def\ktd{{k^{\underline{d}}}}
\def\Det{{\mathrm{Det}}}
\def\detu{{\widecheck{\mathrm{Det}}_\mu^\gamma}}
\def\deto{{\widehat{\mathrm{Det}}_\mu^\gamma}}
\def\Zu{{\widecheck{Z}_\mu^{\gamma}}}
\def\Zo{{\widehat{Z}_\mu^{\gamma}}}
\def\Zun{{\widecheck{Z}_\mu^{\gamma_n}}}
\def\Zon{{\widehat{Z}_\mu^{\gamma_n}}}
\newcommand{\Er}{\mathrm{Er}}
\newif\ifDRAFT
\DRAFTtrue
\ifDRAFT
\newcommand{\marrow}{\marginpar[\hfill$\longrightarrow$]{$\longleftarrow$}}
\newcommand{\niceremark}[3]
   {\textcolor{red}{\textsc{#1 #2:} \marrow\textsf{#3}}}
\newcommand{\ken}[2][says]{\niceremark{Ken}{#1}{#2}}
\newcommand{\manfred}[2][says]{\niceremark{Manfred}{#1}{#2}}
\newcommand{\michael}[2][says]{\niceremark{Michael}{#1}{#2}}
\newcommand{\michal}[2][says]{\niceremark{Michal}{#1}{#2}}
\newcommand{\feynman}[2][says]{\niceremark{Feynman}{#1}{#2}}
%\usepackage[inline]{showlabels}
\else
\newcommand{\ken}[1]{}
\newcommand{\michael}[1]{}
\newcommand{\michal}[1]{}
\newcommand{\feynman}[1]{}
\fi
\newcommand{\norm}[1]{{\| #1 \|}}

\newcommand{\deff}{d_{\textnormal{eff}}}
\def\ee{\mathrm{e}}
\newcommand\mydots{\makebox[1em][c]{.\hfil.\hfil.}}
\def\Sd{\mathscr{S}_{\!d}}
\newcommand{\dx}{\dxy_{\!\cal X}}
\newcommand{\dxk}{\dxy_{\!\cal X}^k}
\newcommand{\dk}{\dxy^k}
\newcommand{\dxy}{\mathrm{D}}
\def\simiid{\overset{\textnormal{\fontsize{6}{6}\selectfont
i.i.d.}}{\sim}}
%\newcommand{\Dxy}{D_{\!\cal X\!,\cal Y}}
\def\vskx{{\mathrm{VS}_{\!\dx}^k}}
\def\vsk{{\mathrm{VS}_{\!D}^k}}
\def\vskxm{{\mathrm{VS}_{\!\dx}^{k-1}}}
\def\vskm{{\mathrm{VS}_{\!D}^{k-1}}}
\def\vsdx{{\mathrm{VS}_{\!\dx}^d}}
\def\vsd{{\mathrm{VS}_{\!D}^d}}
\newcommand{\vs}[1]{{\mathrm{VS}_{\!D}^{#1}}}
\newcommand{\sigd}{\boldsymbol\Sigma_{\!\dx}}
\def\wols{\w_{\mathrm{LS}}}
\def\wds{\boldsymbol\w_{\!D}^*}
\def\kd{K_{\!\dx}}

\def\poly{{\mathrm{poly}}}
\def\polylog{{\mathrm{polylog}}}
\def\DPP{{\mathrm{DPP}}}
\def\DPPcor{{\DPP_{\!\mathrm{cor}}}}
\def\DPPens{{\DPP_{\!\mathrm{ens}}}}
\newcommand{\DPPreg}[1]{{\DPP_{\!\mathrm{reg}}^{#1}}}
\def\Vol{{\mathrm{VS}}}
\def\Lev{{\mathrm{Lev}}}
\newcommand\todod[1]{\Red{\# DH: #1}}
\newcommand{\explain}[2]{\mathrel{\overset{\makebox[0pt]{\text{\tiny
#1}}}{#2}}}
\def\tot {{\mathrm{tot}}}
\def\checkmark{\tikz\fill[scale=0.4](0,.35) -- (.25,0) --
(1,.7) -- (.25,.15) -- cycle;}
\newcommand{\mnote}[1]{{\bf\large \Magenta{*}}\marginpar{\small \Magenta{#1}}}
\newcommand{\bnote}[1]{{\bf #1}}

\newcommand{\sqrtshort}[1]{{\sqrt{\white{\Big|}\!\!\smash{\text{\fontsize{9}{9}\selectfont$#1$}}}}}
\newenvironment{proofof}[2]{\par\vspace{2mm}\noindent\textbf{Proof of {#1} {#2}}\ }{\hfill\BlackBox}
\newcommand{\sets}[2]
{{\hspace{-0.3mm}[\hspace{-0.3mm}#1\hspace{-0.3mm}]\hspace{-0.3mm}\choose
\hspace{-0.3mm}#2\hspace{-0.3mm}}}
\DeclareMathOperator{\sgn}{\textnormal{sgn}}
\DeclareMathOperator{\adj}{\textnormal{adj}}
\def\Rb{{\mathbf{R}}}
\DeclareMathOperator{\ws}{\widetilde{\w}}
\newcommand{\inote}[1]{{\bf {#1}}}
\def\xib{\boldsymbol\xi}
\def\Sigmab{\mathbf{\Sigma}}
\def\Sigmabh{\widehat{\Sigmab}}
\def\Sigmabt{\widetilde{\Sigmab}}
\def\S{\mathbf{S}}
\def\T{\mathbf{T}}
\def\xt{\tilde{x}}
\def\xbt{\widetilde{\x}}
\def\xbh{\widehat{\x}}
\def\ubh{\widehat{\u}}
\def\dom {{\mathrm{dom}}}
\def\val {{\mathrm{val}}}
\def\out {{\mathrm{out}}}
\def\iin  {{\mathrm{iin}}}
\def\s {\mathbf{s}}
\def\q {\mathbf{q}}
\def\qt{\tilde{q}}
\def\itld {j}
\def\ubt {\tilde{\u}}
\def\n{\{1..n\}}
\def\cb {\mathbf{c}}
\def\cW{\mathcal W}
\def\Xt{\widetilde{X}}
\def\Dbt{\widetilde{\D}}
\def\xtb{\tilde{\mathbf{x}}}
\def\ytb{\tilde{\mathbf{y}}}
\def\Xtb{\widetilde{\mathbf{X}}}
\def\Xbb{\overline{\X}}
\def\Xb{{\bar{\X}}}
\def\ybb{\overline{\y}}
\def\f{{\mathbf{f}}}
\def\g{{\mathbf{g}}}
\def\fbb{{\overline{\f}}}
\def\fb{{\overline{f}}}
\def\Xc{\mathcal{X}}
\def\W{\mathbf W}
\def\L{\mathbf{L}}
\def\Rb{\mathbf R}
\def\Pc{\mathcal{P}}
\def\Nc{\mathcal{N}}
\def\Pt{\widetilde{P}}
\def\Hc{\mathcal{H}}
\def\Wc{\mathcal{W}}
\def\Cc{\mathcal{C}}
\def\p{\mathbf p}
%\def\r{\mathbf r}
\def\Y{\mathbf Y}
\def\H{\mathbf H}
\def\K{\mathbf K}
\def\Kh{\widehat{K}}
\def\Kbh{{\widehat{\K}}}
\def\Q{\mathbf Q}
\def\Qbar{{\bar{\mathbf Q}}}
\def\Ytb{\widetilde{\mathbf{Y}}}
\def\c{{n-d\choose s-d}}
\DeclareMathOperator{\Proj}{Proj}
\newcommand{\Span}{\mathrm{span}}
\newcommand{\ofsubt}[1]{\mbox{\scriptsize \raisebox{0.25pt}{$(#1)$}}}
%\raisebox{0.5pt}{$($}}#1\mbox{\tiny \raisebox{0.5pt}{$)$}}}
\newcommand{\ofsub}[1]{\mbox{\small \raisebox{0.0pt}{$(#1)$}}}
%\newcommand{\ofsubb}[1]{\mbox{\footnotesize \raisebox{0.5pt}{$(#1)$}}}
%\newcommand{\ofsub}[1]{(#1)}
%\newcommand{\ofsub}[1]{\mbox{\tiny$|$\hspace{-0.5pt}\raisebox{-0.5pt}{$#1$}}}
\newcommand{\of}[2]{{#1{\!\ofsub{#2}}}}
\newcommand{\oft}[2]{{#1{\!\ofsubt{#2}}}}
\newcommand{\fof}[2]{{#1({#2})}}
\newcommand{\yof}[2]{{#1{\ofsub{#2}}}}
%\newcommand{\yofb}[2]{{#1{\ofsubb{#2}}}}
\newcommand{\lazy}{FastRegVol}
\newcommand{\volsamp}{RegVol}

\newcommand{\Sm}{{S_{-i}}}
\newcommand{\Sp}{{S_{+i}}}
\ifx\BlackBox\undefined
\newcommand{\BlackBox}{\rule{1.5ex}{1.5ex}}  % end of proof
\fi
%\renewcommand{\dagger}{+}
\DeclareMathOperator*{\argmin}{\mathop{\mathrm{argmin}}}
\DeclareMathOperator*{\argmax}{\mathop{\mathrm{argmax}}}
\DeclareMathOperator*{\diag}{\mathop{\mathrm{diag}}}
\def\x{\mathbf x}
\def\y{\mathbf y}
\def\ybh{\widehat{\mathbf y}}
\def\ybb{\bar{\mathbf y}}
\def\xbb{\bar{\mathbf x}}
\def\yb{{\bar y}}
\def\ybt{\widetilde{\mathbf y}}
\def\yh{\widehat{y}}
\def\yhb{\widehat{\y}}
\def\yt{\widetilde{y}}
\def\z{\mathbf z}
\def\a{\mathbf a}
\def\b{\mathbf b}
\def\w{\mathbf w}
\def\v{\mathbf v}
\def\m{\mathbf m}
\def\wbh{\widehat{\mathbf w}}
\def\wh{\widehat{\mathbf w}}
\def\vbh{\widehat{\mathbf v}}
\def\wbt{\widetilde{\mathbf w}}
\def\e{\mathbf e}
\def\zero{\mathbf 0}
\def\one{\mathbf 1}
\def\u{\mathbf u}
\def\ubbar{\bar{\mathbf u}}
\def\f{\mathbf f}
\def\ellb{\boldsymbol\ell}

\def\X{\mathbf X}
\def\Xs{\widetilde{\X}}
\def\B{\mathbf B}
\def\A{\mathbf A}
\def\C{\mathbf C}
\def\U{\mathbf U}
\def\Ubt{\widetilde{\mathbf U}}
\def\Ubh{\widehat{\mathbf U}}
\def\Ubbar{\bar{\mathbf U}}
\def\F{\mathbf F}
\def\D{\mathbf D}
\def\V{\mathbf V}
\def\M{\mathbf M}
\def\Mh{\widehat{\mathbf M}}
%\def\S{\mathbf S}
\def\Stb{\widetilde{\mathbf{S}}}
\def\Sbh{\widehat{\mathbf{S}}}
\def\St{\widetilde{\S}}
\def\Sh{\widehat{S}}
\def\Sc{\mathcal{S}}
\def\Fc{\mathcal{F}}
\def\Vc{\mathcal{V}}
\def\Bc{\mathcal{B}}
\def\Dc{\mathcal{D}}
\def\Z{\mathbf Z}
\def\Zbh{\widehat{\mathbf Z}}
\def\Zbt{\widetilde{\mathbf Z}}
\def\Abh{\widehat{\mathbf A}}
\def\I{\mathbf I}
\def\Ic{\mathcal I}
\def\II{\mathbf {I \!\,I}}
%\def\II{\boldsymbol {\mathbb I}}
\def\A{\mathbf A}
\def\P{\mathbf P}
\def\Ph{\widehat{\mathbf P}}
\def\cP{\mathcal P}
\def\cR{\mathcal R}
\def\Xt{\widetilde{\mathbf{X}}}
\def\Xh{\widehat{\mathbf{X}}}
\def\Rh{\widehat{R}}
\def\Ot{\widetilde{O}}
\def\At{\widetilde{\A}}


\def\E{\mathbb E}
\def\R{\mathbb R}
\def\N{\mathbb N}
\def\Pr{\mathrm{Pr}}
%\def\C{\mathbb C}
\def\tr{\mathrm{tr}}
\def\Sbar{{\bar{S}}}
\def\cS{{\mathcal{S}}}
\def\Tbar{{\bar{T}}}
\def\Tt{{\widetilde{T}}}
\def\rank{\mathrm{rank}}
\def\Prob{\mathrm{Prob}}
\def\Var{\mathrm{Var}}
\def\Xinv{(\X^\top\X)^{-1}}
\def\XinvS{(\X_S\X_S^\top)^{-1}}
\def\ABinvS{(\A_S\B_S^\top)^{-1}}
\def\ABinv{(\A\B^\top)^{-1}}
\def\xinv{\x_i^\top\Xinv\x_i}
\def\Xinvr{(\lambda\I+\X_{-1}^\top\X_{-1})^{-1}}
\def\pdet{\mathrm{pdet}}
\newcommand{\vol}{\mathrm{vol}}
%\newcommand{\defeq}{:=}
\newcommand{\defeq}{\stackrel{\textit{\tiny{def}}}{=}}
\newcommand{\di}{{[d+1]_{-i}}}
\newcommand{\cov}{\mathrm{cov}}
\let\origtop\top
\renewcommand\top{{\scriptscriptstyle{\origtop}}} % this makes transpose not so big

\definecolor{silver}{cmyk}{0,0,0,0.3}
\definecolor{yellow}{cmyk}{0,0,0.9,0.0}
\definecolor{reddishyellow}{cmyk}{0,0.22,1.0,0.0}
\definecolor{black}{cmyk}{0,0,0.0,1.0}
\definecolor{darkYellow}{cmyk}{0.2,0.4,1.0,0}
\definecolor{orange}{cmyk}{0.0,0.7,0.9,0}
\definecolor{darkSilver}{cmyk}{0,0,0,0.1}
\definecolor{grey}{cmyk}{0,0,0,0.5}
\definecolor{darkgreen}{cmyk}{0.6,0,0.8,0}
\newcommand{\Red}[1]{{\color{red}  {#1}}}
\newcommand{\Purple}[1]{{\color{purple}  {#1}}}
\newcommand{\Magenta}[1]{{\color{magenta}{#1}}}
\newcommand{\Green}[1]{{\color{darkgreen}  {#1}}}
\newcommand{\Blue}[1]{\color{blue}{#1}\color{black}}
\newcommand{\Orange}[1]{\textcolor{orange}{#1}\color{black}}
\newcommand{\Brown}[1]{{\color{brown}{#1}\color{black}}}
\newcommand{\Grey}[1]{{\color{grey}{#1}\color{black}}}
\newcommand{\white}[1]{{\textcolor{white}{#1}}}
\newcommand{\yellow}[1]{{\textcolor{reddishyellow}{#1}}}
\newcommand{\darkYellow}[1]{{\textcolor{darkYellow}{#1}}}
\newcommand{\grey}[1]{{\textcolor{grey}{#1}}}

\DeclareMathOperator{\half}{\frac{1}{2}}

\ifx\proof\undefined
\newenvironment{proof}{\par\noindent{\bf Proof\ }}{\hfill\BlackBox\\[2mm]}
\fi

\ifx\theorem\undefined
\newtheorem{theorem}{Theorem}
\fi

\ifx\example\undefined
\newtheorem{example}{Example}
\fi

\ifx\condition\undefined
\newtheorem{condition}{Condition}
\fi
\ifx\property\undefined
\newtheorem{property}{Property}
\fi

\ifx\lemma\undefined
\newtheorem{lemma}{Lemma}
\fi

\ifx\proposition\undefined
\newtheorem{proposition}{Proposition}
\fi

\ifx\remark\undefined
\newtheorem{remark}{Remark}
\fi

\ifx\corollary\undefined
\newtheorem{corollary}{Corollary}
\fi

\ifx\definition\undefined
\newtheorem{definition}{Definition}
\fi

\ifx\conjecture\undefined
\newtheorem{conjecture}{Conjecture}
\fi

\ifx\axiom\undefined
\newtheorem{axiom}{Axiom}
\fi

\ifx\claim\undefined
\newtheorem{claim}{Claim}
\fi

\ifx\assumption\undefined
\newtheorem{assumption}{Assumption}
\fi

\ifx\condition\undefined
\newtheorem{condition}{Condition}
\fi


\coltauthor{\Name{Micha{\l } Derezi\'{n}ski}
  \Email{mderezin@berkeley.edu}\\
  \addr Department of Statistics, UC Berkeley
} 

\begin{document}

\maketitle

\section{Introduction}

\subsection{Nystr\"om}
Let $\L$ be a p.s.d.~$n\times n$ matrix with eigendecomposition
$\U\D\U^\top$, where the columns of $\U$ are the eigenvectors and
$\D=\diag(\lambda_1,...,\lambda_n)$ contains the eigenvalues. For any
unitarily invariant norm $\|\cdot\|$, the optimal rank $k$
approximation to $\L$ is
\begin{align*}
  \L_k =\U\diag(\lambda_1,...,\lambda_k,0,...,0)\U^\top,
\end{align*}
in the sense of minimizing $\|\L-\L_k\|$ among all rank $k$
matrices. To avoid the full eigendecomposition, the Nystr\"om method
was introduced, which constructs a rank $k$ approximation of $\L$
by only using the decomposition of a small $k\times k$ submatrix
$\L_{S,S}$ for some index subset $S$. 
\begin{definition}
 We define the Nystr\"om approximation of $\L$ based on a subset $S$
 as the $n \times n$ matrix $\widehat\L(S)=\L_{\cdot,S}\L_{S,S}^\dagger\L_{S,\cdot}$.
\end{definition}
An example application is a variational Gaussian Process approximation
$\widehat\L(S)$ for which \cite{sparse-variational-gp} showed that the
Kullback-Leibler divergence measuring the quality of the 
approximation can be represented (up to a factor of 2) by the nuclear (trace)
norm of the difference: $\|\L-\widehat\L(S)\|_*$.

\subsection{Connecting projections to Nystr\"om}
Given an $n\times d$ matrix $\X$ and subset
$S\subseteq [n]$ of size $k\leq d$, let $\P_S =
\X_S^\top(\X_S\X_S^\top)^{-1}\X_S$ denote the projection onto
the span of $\{\x_i:i\in S\}$.
The goal in row subset selection is often to minimize $L(S) =
\|\X-\X\P_S\|_F^2$ subject to $|S|=k$ (other norms are also
considered). We show that this problem is 
equivalent to minimizing the nuclear norm for the Nystr\"om
approximation of $\L=\X\X^\top$.
\begin{lemma}
For any $\X$ and $S\subseteq [n]$, we have $L(S) = \|\L -
\widehat\L(S)\|_*$ where $\L=\X\X^\top$.
\end{lemma}
\begin{proof}
\begin{align*}
L(S)&=\|\X(\I-\P_S)\|_F^2 = \tr((\I - \P_S)\X^\top\X(\I-\P_S))=
  \tr(\X(\I-\P_S) \X^\top)\\
  & = \tr(\X\X^\top - \X\X_S^\top(\X_S\X_S^\top)^{-1}\X_S\X^\top) =
    \tr(\L-\widehat\L(S)) = \|\L-\widehat\L(S)\|_*.
\end{align*}  
\end{proof}
\subsection{Optimal bounds using DPPs}
%From now on, w.l.o.g., we use the $\X$ notation instead of the $\L$ notation.
Denoting $\lambda_1,\dots,\lambda_d$ as the decreasing eigenvalues of
$\L$, we define the optimal rank-$k$ projection loss (not based on a subset) as
follows:
\begin{align*}
  E_k^*\defeq \min_{\text{proj. }\P:\,\rank(\P)=k}\|\X-\X\P\|_F^2 = \|\L-\L_k\|_* =
  \sum_{i=k+1}^d\lambda_i.
\end{align*}
\begin{definition} Define $S\sim \DPP(\L)$
  over all subsets $S\subseteq [n]$ so
  that \[\Pr(S)=\frac{\det(\L_{S,S})}{\det(\L+\I)}.\]
  \end{definition}
Prior work showed the following result when sampling from a
cardinality constrained DPP. The below result  was first shown by
\cite{pca-volume-sampling}, then later adapted to the context of
Nystr\"om approximations by \cite{belabbas-wolfe09}. 
\begin{theorem}[\cite{pca-volume-sampling}]\label{t:simple}
If $S\sim \DPP(\L)$, then
  \begin{align*}
    \E\big[L(S)\mid |S|=k\big]\leq (k+1)\cdot E_k^*.
  \end{align*}
\end{theorem}
Moreover, \cite{pca-volume-sampling} showed that  the factor $k+1$
cannot in general be improved when comparing to $E_k^*$. 
\begin{theorem}[\cite{pca-volume-sampling}]\label{t:lower}
For any $\epsilon>0$, there is a $k+1\times k+1$  matrix $\L$ such
that for every subset $S$ of size $k$ we have
\begin{align*}
  L(S)\geq (1-\epsilon)(k+1)\cdot E_k^*.
\end{align*}
\end{theorem}
We show a result closely related to Theorem \ref{t:simple}, where DPP
is not cardinality constrained but rather rescaled by a parameter
$\alpha$ which controls the \emph{expected} subset size.
\begin{theorem}\label{t:main}
For any $\X\in\R^{n\times d}$ and $\alpha>0$, if $S\sim
\DPP(\frac1\alpha\L)$, then
\begin{align*}
  \E\big[L(S)\big] =
  \alpha\cdot\E\big[|S|\big],\quad\E\big[|S|\big]=\tr\big(\L(\alpha\I+\L)^{-1}\big). 
\end{align*}
For $\alpha=E_k^*$ we have $\E\big[|S|\big]\leq k+1$ and the bound
becomes: $\E\big[L(S)\big]\leq (k+1)\cdot E_k^*$.
\end{theorem}

\subsection{Related work for selecting $r>k$ rows}

A result related to Theorem \ref{t:simple} comparing the expected
error of subsets of size $|S|=r>k$ to $E_k^*$ was given by
\cite{more-efficient-volume-sampling}. A more comprehensive analysis
of deterministic methods of row subset selection with $r>k$ for a variety of
norms was given by \cite{near-optimal-columns}. From the perspective
of the Nystr\"om method, this problem was also studied by
\cite{revisiting-nystrom} who used leverage score sampling as one of
the approaches.



\subsection{Open problem}
Denote the loss of the best subset of size $k$ as $L_k^* \defeq
\min_{S:|S|=k} L(S)$. Naturally we have $E_k^*\leq L_k^*$, so the
existing results show that:
\begin{align*}
  \E\big[L(S)\big] \leq (k+1)\cdot L_k^*
\end{align*}
However, the lower bound of Theorem \ref{t:lower} does not apply
here. So  the following remains open.

\textbf{Question:} When is it possible to efficiently find $S$ of size
$k$ such that $L(S)\leq (1+\epsilon)\cdot L_k^*$? 

\section{Application of Theorem \ref{t:main}}

\subsection{Identical eigenvalues}

Suppose that all $d$ non-zero eigenvalues  of $\L=\X\X^\top$ are 1. In
this case we have $E_k^* = d-k$. Now, for some $\epsilon>0$ we choose
$\alpha>0$ such that
\begin{align*}
(1-\epsilon) k = \E\big[|S|\big] = \frac{d}{1+\alpha}.
\end{align*}
So, we have $\alpha = \frac d{(1-\epsilon)k}-1$. By Theorem
\ref{t:main},
\begin{align*}
  \E\big[L(S)\big] = \alpha\, \E\big[|S|\big] =
\Big(\frac d{(1-\epsilon)k}-1\Big)\,(1-\epsilon) k = d - (1-\epsilon)k
  = E_k^*(1 + \epsilon\tfrac k{d-k}).
\end{align*}
Note that for $\epsilon=0$, we have $\E[|S|]=k$ and $\E[L(S)]=E_k^*$,
however we introduce $\epsilon$ to ensure that $|S|\leq k$ with
sufficient probability. From the concentration bound for a Poisson binomial r.v.~we get
\begin{align*}
  \Pr(|S| > k) \leq \ee^{-\frac{(k-(1-\epsilon)k))^2}{2k}} =
  \ee^{-\epsilon^2 k/2}
\end{align*}
Now, it follows that:
\begin{align*}
\E\big[L(S)\mid |S|\leq k\big] \leq \frac{\E[L(S)]}{\Pr(|S|\leq k)}
  \leq \frac{1+\epsilon\frac k{d-k}}{1-\ee^{-\epsilon^2k/2}}\cdot E_k^*.
\end{align*}
In particular, this implies that for any $k$ we can find $S$ of size
$k$ such that $L(S)-E_k^* = O(1) \cdot E_k^*$. 

\subsection{Hard example}
Suppose that $d-1$ eigenvalues are 1 and last eigenvalue is $\delta\ll
1$. Then $E_k^*=\delta+d-k-1$. We proceed similarly (omitting
$\epsilon$ for simplicity): 
\begin{align*}
k = \E\big[|S|\big] = \frac{d-1}{1+\alpha} +
  \frac{\delta}{\delta+\alpha} = \frac{d-1+\delta}{1+\alpha} +
  \frac{\delta(1-\delta)}{(\delta+\alpha)(1+\alpha)}.
\end{align*}
From this it follows that $\alpha=\frac{E_k^*}{k} +
\frac{\delta(1-\delta)}{k(\delta+\alpha)}$ and so we obtain that:
\begin{align*}
  \E\big[L(S)] = E_k^* +  \frac{\delta(1-\delta)}{\delta+\alpha}
  \leq E_k^* + 1 = \delta +d - k.
\end{align*}
So if $k\leq d/2$ then $\E[L(S)] = O(E_k^*)$. However, let us
consider $k=d-1$. Then $E_k^*=\delta$ and 
\begin{align*}
  \E[L(S)] = E_k^*(1 +  \tfrac{1-\delta}{\delta+\alpha}).
\end{align*}
Note that $\delta$ and $\alpha$ are defined in an entangled way. We
use this to get:
\begin{align*}
  \frac{\delta}{\delta+\alpha} = k - \frac{d-1}{1+\alpha} = (d-1)\frac{\alpha}{1+\alpha}
\end{align*}
and further transformation shows that:
\begin{align*}
\frac1{\delta+\alpha} = \frac1\alpha\Big(1 -
  (d-1)\frac\alpha{1+\alpha}\Big) = \frac1\alpha -
  \frac{d-1}{1+\alpha}.
  \end{align*}
  Putting the above two equations together we obtain:
  \begin{align*}
    \frac{1-\delta}{\delta+\alpha} = \frac1\alpha -
    \frac{d-1}{1+\alpha} - (d-1)\frac\alpha{1+\alpha} = \frac1\alpha
   - (d-1).
  \end{align*}
By making $\delta$ arbitrarily small we can also force $\alpha$ to be
arbitrarily small so:
\begin{align*}
  \frac{\E[L(S)]}{E_k^*}= 1+  \big(\tfrac1\alpha - (d-1)\big)
  \underset{\alpha\rightarrow 0}{\longrightarrow} \infty.
\end{align*}
This eigenvalue structure matches the construction used to obtain the
lower bound from Theorem \ref{t:lower}. So as we can see, our bound is
weaker than that of Theorem \ref{t:simple} in this very minor corner case.

  
\subsection{Improvement for exponentially decaying eigenvalues}
Suppose that the eigenvalues of $\L=\X\X^\top$
exhibit exponential decay: $\lambda_i=\gamma^{i}$, and for
simplicity assume that $d\rightarrow\infty$ so that the stable rank is 
$r=\sum_i\frac{\lambda_i}{\lambda_1} = \frac1{1-\gamma}$. Setting $\alpha =
\gamma^t$, if $S\sim\DPP(\frac1\alpha\L)$ then
\begin{align*}
  \E\big[|S|\big] \leq t + \gamma^{-t}\sum_{i>t}\gamma^i =
  t + \frac\gamma{1-\gamma} = t +\gamma\cdot r,
\end{align*}
Letting $k=\E[|S|\big]\geq r$ and noting that
$E_k^*=\gamma^k\cdot\gamma r$, Theorem \ref{t:main} obtains the following bound: 
\begin{align*}
  \E\big[L(S)\big] = k\cdot \gamma^{k-\frac\gamma{1-\gamma}}\leq \ee
  k\gamma^k = \frac{\ee k}{\gamma r}E_k^*,
\end{align*}
better than the bound obtained by \cite{pca-volume-sampling}
when $r\gg 1$ (i.e., decay is slow). In particular, we
achieve that if $k = O(r)$ then $\E[L(S)] = O(E_k^*)$.


 
 Concentration argument seems straight forward. We take $\E(|S|) = (1-\epsilon)k$, so that $t \geq (1-\epsilon)k - \gamma r$. We have, 
 for the expected loss, 
 
 \begin{align*}
 \E[L(S)] = \alpha \E[|S|] =  (1- \epsilon) k \gamma^t \leq  (1- \epsilon)k\cdot \gamma^{(1-\epsilon)k-\frac\gamma{1-\gamma}}\leq \ee
  (1- \epsilon)k\gamma^{(1-\epsilon)k} = \frac{\ee  (1- \epsilon)k}{\gamma^(\epsilon k) r}E_k^* .
 \end{align*}
 
 Then, 
 \begin{align*}
 \E\big[L(S)\mid |S|\leq k\big] \leq \frac{\E[L(S)]}{\Pr(|S|\leq k)}
 \leq \frac{\ee  (1- \epsilon)k}{\gamma^{k\epsilon}\gamma r(1-\ee^{-\epsilon^2k/2})}\cdot E_k^*.
 \end{align*}

 
\subsection{General result for stable rank} 
Based on the above examples, we formulate the following result.

\begin{theorem}
For any $\lambda_1\geq...\geq\lambda_d\geq 0$, $\alpha>0$ and $k\in[d]$, define
\begin{align*}
  k_\alpha=\sum_i\frac{\lambda_i}{\lambda_i+\alpha},\quad
  r =\sum_i\frac{\lambda_i}{\lambda_1},\quad\text{and}\quad E_{k}^* =
  \sum_{i>k}\lambda_i.
  \end{align*}
For any $\epsilon>0$, if $k\leq \frac{\epsilon}{1+\epsilon}\cdot r$, then there is $\alpha$ such that
    $k_\alpha\leq k$ and  $\alpha k_\alpha\leq (1+\epsilon) E_k^*$.
 \end{theorem}
\begin{proof}
We have
  \begin{align*}
E_{k}^*=
    \frac{\lambda_1}{\lambda_1}\sum_{i>k}\lambda_i =\lambda_1\Big( r -
 \frac{\sum_{i\leq k}\lambda_i}{\lambda_1}\Big)\geq \lambda_1\Big( r -  \sum_{i\leq k}\frac{\lambda_1}{\lambda_1}\Big) =\lambda_1( r -
    k).
  \end{align*}
  Now we let $\alpha = (1+\epsilon) E_k^*/k$ so that $\alpha k
  =(1+\epsilon)E_k^*$ and moreover: 
  \begin{align*}
    k_\alpha=\sum_i\frac{\lambda_i}{\lambda_i+(1+\epsilon)E_{k}^*/k}
    \leq \sum_i\frac{\lambda_i}{\lambda_i+(1+\epsilon)\lambda_1(\frac rk-1)}
    \leq\sum_i\frac{\lambda_i}{(1+\epsilon)\lambda_1\frac{r-k}k}=\frac{r}{r-k}\cdot
    \frac k{1+\epsilon}.
  \end{align*}
  Note that $k$ is chosen so that $\frac
  r{r-k}\,\frac1{1+\epsilon}\leq 1$.
\end{proof}



\subsubsection{Concentration}


\begin{lemma} With $r$ as the stable rank and any $\epsilon > 0$, for $ \frac 2{\epsilon^2} \ln \frac1{\epsilon} \leq k \leq  \frac {\epsilon }{(1+\epsilon)}  r $, we can find a set $S$ of size $|S|\leq k$ with $L(S)  \leq (1+3\epsilon)  E^*_k$ with $O(\frac1{\epsilon})$ rounds of rejection sampling on average. 
\end{lemma}


Use $\alpha = \frac{(1+\epsilon)}{1-\delta} \frac{E^*_k}{k}$. This makes sure that, 

\begin{align*}
\E(|S|) = \sum_i\frac{\lambda_i}{\lambda_i+(1+\epsilon)E_{k}^*/k(1-\delta)}
\leq \sum_i\frac{\lambda_i}{\lambda_i+\frac{(1+\epsilon)}{1-\delta}\lambda_1(\frac rk-1)}
\leq\sum_i\frac{\lambda_i}{\frac{(1+\epsilon)}{1-\delta}\lambda_1\frac{r-k}k}=\frac{(1-\delta)r}{r-k}\cdot
\frac k{1+\epsilon},
\end{align*}
	
which implies $\E(|S|) \leq (1-\delta)k$, since $k$ is chosen to ensure that $\frac{r}{r-k}.\frac{1}{1+\epsilon} \leq 1$. This gives $\E(L(S)) = \alpha \E(|S|) = \frac{(1+\epsilon)}{1-\delta} \frac{E^*_k}{k} \E(|S|) \leq \frac{(1+\epsilon)}{1-\delta} \frac{E^*_k}{k} (1-\delta) k = (1+\epsilon) E^*_k$.

 For concentration , 

\begin{align*}
	& \E\big[L(S)\mid |S|\leq k\big] \leq \frac{\E[L(S)]}{\Pr(|S|\leq k)} \leq \frac {(1+\epsilon)E^*_k}{(1-\ee^{-\delta^2k/2}) }.  
\end{align*}

With $\delta \leq \epsilon$, and $k \geq \frac{2}{\epsilon^2} \ln \left(2 + \frac1{\epsilon} \right)$, which ensures $\ee^{-\delta^2 k/2} \leq \frac{\epsilon}{1+2\epsilon}$, we get, $\E\big[L(S)\mid |S|\leq k\big] \leq (1+2\epsilon) E^*_k$.

Note that for $k \geq \frac{2}{\epsilon^2} \ln \left(2 + \frac1{\epsilon} \right)$, we have $|S| \leq k$ with probability $\frac{1+\epsilon}{1+2\epsilon}$. Using shorthand $E_k=\E\big[L(S)\mid |S|\leq k\big]$ in the Markov inequality, we have,

\[ P\big[L(S) \geq \eta E_k \mid |S| \leq k \big] \leq \frac{1}{\eta}. \]

Thus, with probability $\geq  \frac{1+\epsilon}{1+2\epsilon} (1 - \frac1{\eta})$, we have $|S| \leq k$ and  $L(S) \leq \eta E_k$. Choosing $\eta = (1+\frac \epsilon {1+2\epsilon})$ ensures $L(S) \leq (1+3\epsilon)E^*_k$ with probability $\geq  \frac{(1+\epsilon)\epsilon}{(1+2\epsilon)(1+3\epsilon)}$, which means on average $\frac{(1+2\epsilon)(1+3\epsilon)}{(1+\epsilon)\epsilon} = O(1/\epsilon)$ rounds of rejection sampling suffice.



\subsubsection{Longer tail}
Consider the case when $\frac{d}{2}$ eigenvalues are 1 and the remaining $\frac{d}{2}$ are all equal to $\delta \ll 1$. In this case, for $k>\frac{d}{2}$ we have, 

\begin{align*}
\frac{\E(L(S)) } {E^*_k} &= \frac{\alpha k}{(d-k)\delta}.
\end{align*}   


With $\alpha = \delta$, we get $k = \sum \frac{\lambda_i}{\lambda_i + alpha}= \frac{d}{2}(\frac{1}{1+\alpha} + \frac{\delta}{\delta + \alpha}) = \frac{d}{2} (\frac{1}{1+\alpha} + \frac{1}{2}) \approx \frac{3d}{4}$ for $\delta = \alpha \rightarrow 0$.

So, for $k \approx 3d/4$ in the above example, we see $\E(L(S))$ is constant factor approximation of $E^*_k$.

 

\subsubsection{General tail} 
Say $m$ eigenvalues are $1$ and $(d-m)$ eigenvalues are $\delta$. We want to consider the case when $k>m$. And, we would like to have a $\beta$-constant factor approximation for $\E(L(S))$ i.e. $\frac{\E(L(S))}{E^*_k} = \beta \implies \frac{\alpha k}{(d-k)\delta} = \beta \implies \frac{\alpha}{\delta} = \frac{\beta(d-k)}{k}$. 



\begin{align*}
k &= \frac{m}{1+ \alpha} + \frac{\delta(d-m)}{\alpha + \delta} \\ 
& = \frac{m}{1+\alpha} + \frac{(d-m)}{1 + \frac{\alpha}{\delta}} \\
& = \frac{m}{1+\alpha} + \frac{(d-m)}{1 + \beta (\frac{d}{k} - 1)}.
% & = m \left( \frac{1}{1+ \alpha} - \frac{1}{\delta + \alpha} \right) + \frac{d}{\alpha + \delta}, \\ 
\end{align*}

Assuming $\frac{m}{d} = c$ (constant), and $\alpha, \delta \rightarrow 0$,we get 

\begin{align*}
\frac{k}{d} &= c + \frac{1-c}{1 + \beta (\frac{d}{k} - 1)}.
\end{align*}

The above expression is quadratic in $\frac{k}{d}$. Let $x = \frac{d}{k}$. We can write, 

\begin{align*}
\frac{1}{x} &= c+ \frac{1-c}{1 + \beta x - \beta}\\
 & = \frac{c + \beta c x - \beta c + 1 -c }{1 + \beta x - \beta}\\      
 \implies & \beta c x^2 + x (1-\beta - \beta c) +(\beta -1) = 0.
\end{align*}

For specific values, say $c=1/2$ and $\beta=2$, we get the quadratic $x^2 - 2x +1 = 0 \implies x =1$. 

\subsection{Lower bound}

 \begin{lemma}
	For any $\epsilon>0$, $\exists$ a matrix $\mathbf{X}$ with stable rank $r$ such that using DPP sampling with $E(|S|) = k$ with $k \leq \frac{\epsilon}{1+\epsilon} r$, we can lower bound $E(L(S)) \geq \left(\frac1{2} + \sqrt{\epsilon}\right)E^*_k $.
\end{lemma}
\begin{proof}
	
Using the same setup as above, with $m=k$, and say we want $r=(1+\rho) k$.  To calculate $r$, 

 
 \begin{align*}
r = \sum_i \frac {\lambda_i}{ \lambda_1} = m + \delta (d-m) = k + \delta(d-k).
 \end{align*}
 
 

Since  $r=(1+\rho) k$, we get $\delta(d-k) = \rho k$. 

Furthermore, since we want $\frac{\E(L(S))}{E^*_k} = \beta$, 

\begin{align*}
&\frac{\alpha k }{\delta(d-k) } = \beta \\ & \implies \alpha k = \beta \rho k \implies \alpha = \beta \rho.
\end{align*}

For $k$, 

\begin{align*}
& k = \sum_i \frac{\lambda_i}{\lambda_i + \alpha} 
 = \frac{k}{1+\alpha} + \frac{\delta(d-k)}{\delta+\alpha} \\
& \implies k =  \frac{k}{1+\alpha} + \frac{\rho k}{\delta+\alpha}\\
&\implies 1 =  \frac{1}{1+\alpha} + \frac{\rho}{\delta+\alpha} \\
&\implies \delta = \rho + \frac{\rho}{\alpha} - \alpha = \rho + \frac1 {\beta} - \beta \rho.
\end{align*}

For the bound, we need $0 < \delta < 1$. While $\delta < 1$ is trivially satisfied, for $\delta > 0$, we need $\rho < \frac{1}{\beta (\beta - 1)}$.

With $\rho = \frac1 {\epsilon}$, we have $k = \frac{\epsilon}{1+ \epsilon}r$. And with $\beta =1+\frac{\epsilon}{1+ \epsilon} $, the condition $\rho < \frac{1}{\beta (\beta - 1)}$ is satisfied. Hence, we get for any $\epsilon>0$, with $k = \frac{\epsilon}{1+ \epsilon}r $, we are ensured that $\frac{E(L(S))}{E^*_k} \geq \left( 1 + \frac{\epsilon}{1+\epsilon} \right)$.                                                                                                                                                                                             

We can also write the condition as $\beta (\beta -1) -\epsilon < 0$, which implies
 \[\beta \in \left(\frac{( 1- \sqrt(1+ 4 \epsilon)}{2}, \frac{( 1+ \sqrt(1+ 4 \epsilon)}{2} \right)\]. $\beta = (1/2 + \sqrt{\epsilon})$ satisfies this condition.
 

\end{proof}

\section{Lower bound for subsets larger than stable rank}
We first extend the lower bound result of
\cite{pca-volume-sampling}. Their construction uses a set of vectors that
form a regular simplex centered at $\zero$ and than slightly
shifts it. We show that in fact the regularity of the simplex is not
necessary for their proof to work.
\begin{lemma}\label{l:simplex}
Fix $\epsilon\in(0,1)$.  Consider any $k+1$ vectors $\x_i$ spanning a
$k$-dimensional subspace with their centroid at $\zero$ and let $\v$
be a unit vector orthogonal to that subspace. For any sufficiently small
$\alpha>0$, if we construct
matrix $\X$ with $k+1$ rows so that the $i$th row vector is $\x_i^\top+\alpha\v^\top$, 
then for any subset $S\subseteq[k+1]$ of size $k$,
\begin{align*}
  L(S)\geq (1-\epsilon)(k+1) E_k^*.
  \end{align*}
\end{lemma}
\begin{proof}
Let $\P^*$ be the projection onto the span of
$\{\x_1,\dots,\x_{k+1}\}$. Then we have
\begin{align*}
  E_k^* \leq \|\X - \X\P^*\|_F^2 = (k+1)\cdot\alpha^2.
\end{align*}
We will use the shorthand $\xbt_i=\x_i+\alpha\v$. Consider any subset
of $k$ rows from $\X$. W.l.o.g. suppose 
$S=\{1,\dots,k\}$. Since the centroid of all the rows is $\alpha\v$,
we can rewrite the last vector as 
\begin{align*}
\xbt_{k+1} = (k+1)\alpha\v - \sum_{i\in S}\xbt_i.
\end{align*}
Since for any $i\in S$ we have $\P_S\xbt_i=\xbt_i$, it follows that:
\begin{align*}
  L(S) &= \|\xbt_{k+1} - \P_S\xbt_{k+1}\|^2 \\
  &=\bigg\| (k+1)\alpha\v - \sum_{i\in S}\xbt_i - \Big(\P_S (k+1)\alpha\v -
    \sum_{i\in S}\xbt_i\Big)\bigg\|^2\\
       &=(k+1)^2\alpha^2\|\v-\P_S\v\|^2.
\end{align*}
Note that $\lim_{\alpha\rightarrow 0}\P_S\v = \zero$ because $\v$
is orthogonal to the subspace spanned by $\{\x_i:i\in S\}$, so there is
$\alpha_S>0$ such that for any $\alpha\leq\alpha_S$ we have $\|\v-\P_S\v\|^2\geq 1-\epsilon$. This shows that
$L(S)\geq (1-\epsilon)(k+1)E_k^*$ for $\alpha=\alpha_S$. Taking
$\alpha = \min_{S:|S|=k}\alpha_S$ completes the proof.
\end{proof}
We next obtain our main hardness result which states that the lower bound
factor $k+1$ of \cite{pca-volume-sampling} cannot be improved by constraining to matrices of a
particular stable rank $r$ as long as $k>r$. This stands in contrast
to our upper bound which showed that when $k<r$, then the factor $k+1$
can be significantly improved.
\begin{theorem}
For any $\epsilon\in(0,1)$ and $n,d,k,r\in\N$ such that  $0<r< k<\min\{n,d\}$, there is a matrix
$\X\in\R^{n\times d}$ with stable rank in $[r,r+1]$ such that for any subset
$S$ of size $k$
\begin{align*}
  L(S)\geq (1-\epsilon)(k+1) E_k^*.
  \end{align*}
\end{theorem}
\begin{proof}
  We will construct a set of vectors by gradually modifying and augmenting it while
  preserving two invariants: (1) that the stable rank is in $[r,r+\frac23]$
  and (2) the centroid of the vectors is  $\zero$. 
  
First, construct matrix $\X_{(r+1)}$  from $r+1$ row vectors
$\x_1^\top,...,\x_{r+1}^\top$ that form 
  a regular simplex centered at $\zero$, scaled so that the $r$ non-zero
  singular values of $\X_{(r+1)}$ are all equal 1. Since all the non-zero
  singular values of the matrix are equal, the stable rank
is $r$. Next, we iteratively add vectors one by one (and modify the
existing ones) until there is $k+1$ in the set. Let $\x_{1},...,\x_{t-1}$ and
$\xbt_1,...,\xbt_t$ denote the 
sets of vectors \emph{before} and \emph{after} adding the $t$th vector,
with corresponding matrices $\X_{(t-1)}$ and $\X_{(t)}$. Also, let
$\tilde r_{t-1}$ denote the stable rank of $\X_{(t-1)}$.
Adding the $i$th vector proceeds as follows:
  \begin{enumerate}
    \item shift all $t-1$ vectors along a normal direction $\v_t$
      (orthogonal to their span) by some $\delta_t$,
    \item let the new vector be $\xbt_{t}=-t\,\delta_t\v_t$.
    \end{enumerate}
    After the transformation and adding the $i$th vector, the centroid
    stays at $\zero$, and moreover:
    \begin{align*}
      \X_{(t)}^\top\X_{(t)} &=
      \sum_{i=1}^{t-1}(\x_i+\delta_t\v_t)(\x_i+\delta_t\v_t)^\top +
      t^2\delta_t^2\v_t\v_t^\top\\
      &=\sum_{i=1}^{t-1}\x_i\x_i^\top +
        \delta_t\v_t\Big(\sum_{i=1}^{t-1}\x_i^\top\Big) +
        \Big(\sum_{i=1}^{t-1}\x_i\Big)\delta_t\v_t^\top +
 (t-1)\delta_t^2\v_t\v_t^\top+        t^2\delta_t^2\v_t\v_t^\top\\
      &=\X_{(t-1)}^\top\X_{(t-1)}+(t^2+t-1)\delta_t^2\v_t\v_t^\top.
    \end{align*}
Therefore, since $\v_i$ is orthogonal to the span of $\{\x_{1},...,\x_{t-1}\}$,
it becomes a new eigenvector of $\X_{(t)}^\top\X_{(t)}$, while the old
eigenvectors from $\X_{(t-1)}^\top\X_{(t-1)}$ remain the same in the
new matrix. Thus, as long as $\delta_t$ is sufficiently small (so that
the new eigenvalue does not become the largest), we have:
    \begin{align*}
\tilde r_{t} = \tilde r_{t-1} + \|\X_{(t)}\|_F^2
      -\|\X_{(t-1)}\|_F^2=\tilde r_{t-1} + \delta_t^2(t^2+t-1).
    \end{align*}
    To make sure that stable rank remains less than $r+\frac23$, we
    assume that $\delta_t\leq\frac1{t2^{t}}$ and bound the stable rank of the final set:
    \begin{align*}
 \tilde r_{k+1} = r + \sum_{t=r+2}^{k+1} \delta_t^2 (t^2+t-1) \leq r +
      \sum_{t=1}^{k+1}\frac2{4^t}\leq r + \frac23.
    \end{align*}
  Now, it remains to apply Lemma \ref{l:simplex} to the set
  of $k+1$ vectors, obtaining matrix $\Xt_{(k+1)}$ that satisfies the
desired lower bound. When choosing
  $\alpha$, we have to again ensure that it 
  is sufficiently small so that stable rank does not change by
  more than $\frac13$. Naturally, this can be done just as we did it
  in the construction of the set. We construct the final matrix $\X$
from $\Xt_{(k+1)}$ by padding with zeros to get the shape right,
which does not affect the loss $L(\cdot)$ or the stable rank. 
\end{proof}

\section{Lower bound for subsets smaller than stable rank}
Let $\x_1,...,\x_{k+1}$ be vectors forming a regular simplex centered
at $\zero$ which lies inside of the subspace spanned by the first $k$
coordinates. Denote $a=\|\x_i\|$ as the length of each vector. We
shift those vectors in the $(k+1)$-st coordinate by $\alpha$, getting
$\xbt_i=\x_i+\alpha\e_{k+1}$. Then, we add further vectors
$\x_{k+2},...,\x_n$ for some $n\gg k$ so that $\x_i=\delta\e_{i}$ for
$i>k+1$. Denote the entire $n\times d$ matrix as $\X$. The optimum
rank $k$ error is upper bounded as follows: 
\begin{align*}
  E_k^* \leq (k+1)\alpha^2 + (n-k-1)\delta^2.
\end{align*}
The top $k$ eigenvalues of $\X^\top\X$ associated with the simplex are all
$\lambda_i=\frac{k+1}{k}\,a^2$. This means that,
assuming that $\delta^2\leq(k+1)\alpha^2\leq \frac{k+1}{k}\,a^2$ (to
achieve the correct ordering of the eigenvalues), the
stable rank of $\X$ is given by:
\begin{align*}
  r = \frac1{\lambda_1}\sum_{i=1}^d\lambda_i = k +
  k\,\frac{\alpha^2}{a^2} + \frac{(n-k-1)\delta^2}{\frac{k+1}k\,a^2}.
\end{align*}
Finally, for any subset $S\subseteq[k+1]$, the loss is:
\begin{align*}
  L(S) &= (k+1)^2\frac{(\frac ak)^2\alpha^2}{(\frac ak)^2+\alpha^2} +
  (n-k-1)\delta^2\\
  &=(k+1)^2\frac{a^2\alpha^2}{a^2+k^2\alpha^2} +
  (n-k-1)\delta^2.
\end{align*}
Suppose that $\alpha^2=a^2/k^2$ and $(n-k-1)\delta^2 = (k+1)(\rho a^2-\alpha^2)$ for some
$\rho\in[\frac1{k^2},1]$. It follows that $r=(1+\rho)k$ and
$E_k^*=\rho(k+1)a^2$. Also we get:
\begin{align*}
  L(S)&=  (k+1)^2 \frac{a^2}{2k^2} + (k+1)\Big(\rho-\frac1{k^2}\Big) a^2\\
      &=\rho(k+1)a^2 + a^2\frac{k+1}{k^2}\Big(\frac{k+1}{2} - 1\Big)\\
  &=E_k^*\cdot\Big(1 + \frac{k-1}{2\rho k^2}\Big).
\end{align*}
So if we let $\epsilon = \frac{k-1}{2\rho k^2}$ then we can turn
this into the following lower bound. 
\begin{theorem}
  For any $\epsilon\in(0,k^2)$ and $n,d,k,r\in\N$ such
  that $k\leq\frac{\epsilon}{1+\epsilon}\,r\leq\min\{n,d\}$, there is
  an $n\times d$ matrix $\X$ with stable rank  
  $r$ such that for any subset $S$ of size $k$,
  \begin{align*}
    L(S) \geq \bigg(1+\epsilon\cdot \frac{k-1}{2k^2}\bigg)\cdot E_k^*.
  \end{align*}
\end{theorem}
Note that this lower bound is worse than our upper bound by a factor
of about $\frac1{2k}$. The key question is which bound is loose: the
upper or the lower bound? Also, not sure how to cleanly
write the lower bound
in a way that concludes with $L(S)\geq (1+\epsilon) E_k^*$.


\subsection{Const lower bound}
Choose $\rho \in [\frac1 {k^2}, \frac{(k-1)}{2k^2\epsilon} ]$. This still means $r=(1+\rho)k$ and
$E_k^*=\rho(k+1)a^2$. Following the steps for $L(S)$, we get,
\begin{align*}
L(S) \geq E_k^*\cdot\Big(1 + \frac{k-1}{2\rho k^2}\Big) \geq (1 + \epsilon) E_k^*.
\end{align*}

Note that we need $\delta^2 \leq (k+1) \alpha^2$ to hold, but  $(n-k-1)\delta^2 = (k+1)(\rho a^2-\alpha^2)$ can still hold simultaneously by choosing large enough $n$ for  $\rho \in [\frac1 {k^2},\frac{(k-1)}{2k^2\epsilon} ]$.

Now we need to find the condition for $k$ wrt $r$. By $r = (1+\rho)k$, we have $r \geq  (1+\frac1{k^2})k \iff k + \frac1 k \leq r \impliedby k < r$.

Finally the range of $\epsilon$ can only be such that the lower bound of $1/k^2$ is not violated.

\begin{theorem}(Informal) For any $\epsilon \in (0,(k-1)/2 ]$, $\exists$ a matrix $\mathbf{X} $ of stable rank $r$, such that for selecting $k$ rows with $k < r$, ensures 
	\begin{align*}
	L(S) \geq (1+\epsilon) E^*_k
	\end{align*}
	
\end{theorem}
\section{Proof of Theorem \ref{t:main}}
Our result is based on the following matrix expectation formula.
\begin{lemma}
  For any $\X\in\R^{n\times d}$ and non-negative weights $p_1,...,p_n$, let $\Xt$ denote
  $\X$ with $i$th row rescaled by $\sqrt{p_i}$. If $S\sim\DPP(\Xt\Xt^\top)$, then
  \begin{align*}
    \E\big[\I-\X_S^\top(\X_S\X_S^\top)^{-1}\X_S\big] & = (\I +
    \Xt^\top\Xt)^{-1} = \Big(\I + \sum p_i\x_i\x_i^\top\Big)^{-1}.
  \end{align*}
\end{lemma}
\begin{proofof}{Theorem}{\ref{t:main}}
  Let $S\sim\DPP(\frac1\alpha\X\X^\top)$. The expected subset size is given by:
\begin{align*}
  \E\big[|S|\big] =
  \tr\big(\tfrac1\alpha\X\X^\top(\I+\tfrac1\alpha\X\X^\top)^{-1}\big).
\end{align*}
On the other hand the expected loss is given by:
\begin{align*}
\E\big[L(S)\big]=  
  \tr(\X^\top\X \,\E[\I-\P_S]) =
  \tr\big(\X^\top\X(\I+\tfrac1\alpha\X^\top\X)^{-1}\big) =
  \E[|S|]\cdot \alpha.
\end{align*}
Now, setting $\alpha = E_k^*$, it follows that:
\begin{align*}
  \E\big[|S|\big] = \sum_{i=1}^d\frac{\lambda_i}{\alpha+\lambda_i}\leq
  \sum_{i=1}^k\frac{\lambda_i}{E_k^*+\lambda_i} +
  \frac1{E_k^*}\sum_{i=k+1}^d\lambda_i\leq k+1.
\end{align*}
\end{proofof}

Note that the scaling weights $p_i$ we use to prove Theorem
\ref{t:main} are all set to $\frac1\alpha$. Can we improve the result
by finding optimal values of the weights? 

\section{Algorithm using a weighted DPP}

\begin{definition}
Define the $\alpha$-regularized Nystr\"om approximation of $\L$ based on a subset $S$
 as the $n \times n$ matrix $\widehat\L_\alpha(S)=\L_{\cdot,S}(\alpha\I+\L_{S,S})^\dagger\L_{S,\cdot}$.
\end{definition}


Given some value $\alpha>0$, consider the following optimization problem:
\begin{align*}
  \min_p\quad
  &
  \tr\bigg[\X^\top\X\Big(\I+\frac1\alpha\sum_ip_i\x_i\x_i^\top\Big)^{-1}\bigg]\\
  \text{subject to}\quad
  &\sum_i p_i = k,
  % \sum_j p_j\,\x_j^\top\Big(\alpha\I+\sum_ip_i\x_i\x_i^\top\Big)^{-1}\!\x_j\leq k,
\quad 0\leq p_i\leq 1 \ \ \forall_i.
\end{align*}
Let $p^*(\alpha)$ denote the solution.%  and let $S\sim
% \DPP(\frac1\alpha\Xt\Xt^\top)$. Note that 
% \begin{align*}
% \E\big[|S|\big] =
%   \tr\big(\Xt^\top\Xt(\alpha\I+\Xt^\top\Xt)^{-1}\big)\leq \tr\big(\X^\top\X(\alpha\I+\X^\top\X)^{-1}\big)=d_\alpha.
%   \end{align*}
%Moreover, define the following
% secondary optimization:
% \begin{align*}
%   \alpha^* = \min_{\alpha>0} \ \ \alpha \quad\text{subject to}\quad
%   \tr\bigg[\Big(\sum_ip_i^*(\alpha)\x_i\x_i^\top\Big)\Big(\alpha\I + \sum_i
%   p_i^*(\alpha)\x_i\x_i^\top\Big)^{-1}\bigg]\leq k.
% \end{align*}

\begin{theorem}\label{t:weighted}
If $S\sim\DPP(\frac1{\alpha}\Xt\Xt^\top)$, where $i$th
row of $\Xt$ is rescaled by $\sqrt{p_i^*(\alpha)}$, we obtain:
\begin{align*}
  \E\,\big[\|\L-\widehat\L(S)\|_*\big] \leq \min_{S:|S|=k} \|\L -
  \widehat\L_{\alpha}(S)\|_*.
\end{align*}
\end{theorem}
\textbf{Problem:} It is not clear how to choose $\alpha$ to ensure
that $\E[|S|]\leq k$. However, a simple doubling strategy might work in
practice.

Another strategy would be to put the expected subset size as a
constraint in the optimization:
\begin{align*}
  \min_p\quad
  &
  \tr\bigg[\X^\top\X\Big(\I+\frac1\alpha\sum_ip_i\x_i\x_i^\top\Big)^{-1}\bigg]\\
  \text{subject to}\quad
  & \tr\bigg[\Big(\sum_ip_i\x_i\x_i^\top\Big)\Big(\alpha\I + \sum_i
 p_i\x_i\x_i^\top\Big)^{-1}\bigg]\leq k,
\quad 0\leq p_i\leq 1 \ \ \forall_i.
\end{align*}
Here, the problem is that we can no longer write the bound from
Theorem \ref{t:weighted}, and also the constraint becomes non-linear.


% \section{A limit argument for Theorem \ref{t:main}}\label{s:proof}
% \begin{definition}\label{d:r-dpp}
% Given matrix $\X\in\R^{n\times d}$, a sequence of Bernoulli probabilities $p=(p_1,\dots,p_n)\in[0,1]^n$
% and a p.s.d.~matrix $\A\in\R^{d\times d}$, let
% $\Vol_p(\X,\A)$ be a distribution over all sets $S\subseteq [n]$ s.t.:
%   \begin{align}
%   \Pr(S) = \frac{\det(
%   \X_S^\top\X_S+\A)}{\det\!\big(
%  \sum_ip_i\x_i\x_i^\top+ \A\big)}\
% \cdot \prod_{i\in S}p_{i}\cdot\prod_{i\not\in S}(1-p_i).\label{eq:poisson-prob}
% \end{align}
% %   \Pr(\pi)\ \propto\ \det\!\big(\A +
% %   \X_{\pi}^\top\X_{\pi}\big)
% %   \Pr\!\bigg(\Big[K\!=\!|\pi|\Big]\,\wedge\,\Big[\forall_{i=1}^{|\pi|}\,\pi_i\!=\!\pi_i\Big] \bigg).
% % \end{align*}
% \end{definition}
% Whenever all $p_i=\frac sn$, we will write $\Vol^s(\X,\A)$.
% The below theorem was shown in prior work.
% \begin{theorem}[\cite{bayesian-experimental-design}]\label{t:expectations}
% Given $\X\!\in\!\R^{n\times d}$, $\A\!\in\!\R^{d\times d}$ and $p$,
% if $S\sim \Vol_p(\X,\A)$, then
% \begin{align}
%   \E\Big[\big(\X_S^\top\X_S+\A\big)^{-1}\Big]
%   &\preceq \Big(
%   \sum_ip_i\x_i\x_i^\top +\A \Big)^{-1}.\label{eq:sqinv}
% \end{align}
% \end{theorem}

% We will use the following limit argument.
% \begin{lemma}
% For any $p\in[0,1]^n$, let $\Xt$ denote $\X$ with $i$th
%   row rescaled by $\sqrt{p_i}$ for all $i$. We have
%   \[\lim_{\epsilon\rightarrow 0}\Vol_{\epsilon
%     p}(\tfrac1{\sqrt\epsilon}\,\X,\I) =\DPP(\Xt\Xt^\top).\]
% \end{lemma}
% % \begin{proof}
% %   Denoting $p=\frac sn$ we have
% %   \begin{align*}
% %     \Pr(S)
% % &= \frac{\det(\frac1p\X_S^\top\X_S+\I)}{\det(\X^\top\X+\I)}\cdot
% %     p^{|S|}\cdot (1-p)^{n-|S|}\\
% % &=    \frac{\det(\X_S\X_S^\top+p\I)}{\det(\X\X^\top+\I)}\cdot
% %                                    (1-p)^{n-|S|}\\
% % &\rightarrow \frac{\det(\X_S\X_S^\top)}{\det(\I+\X\X^\top)}.
% %   \end{align*}
% % \end{proof}

% % \begin{theorem}
% %   For any $\X\in\R^{n\times d}$ and $p\in[0,1]^n$, if $S\sim\DPP(\Xt\Xt^\top)$, then
% %   \begin{align*}
% %     \E\big[\X_S^\top(\X_S\X_S^\top)^{-1}\X_S\big] &=
% %     \Xt^\top(\I+\Xt\Xt^\top)^{-1}\Xt,\\
% %     \E\big[\I-\X_S^\top(\X_S\X_S^\top)^{-1}\X_S\big] &= (\I +
% %     \Xt^\top\Xt)^{-1} = \Big(\I + \sum_ip_i\x_i\x_i^\top\Big)^{-1},
% %   \end{align*}
% %   where $\Xt$ here denotes $\X$ with $i$th row rescaled by $\sqrt{p_i}$.
% % \end{theorem}
% We use these two facts to show our new expectation inequality.
% \begin{proofof}{Theorem}{\ref{t:main}}
%   If $S \sim \Vol_{\epsilon p}(\frac1{\sqrt{\epsilon}}\,\X,\I)$ then
%   \begin{align*}
%     \E\big[(\I + \tfrac1\epsilon\X_S^\top\X_S)^{-1}\big] \preceq
%     (\I+\Xt^\top\Xt)^{-1}.
%   \end{align*}
%   On the other hand, we have
%   \begin{align*}
%     (\I + \tfrac 1\epsilon\X_S^\top\X_S)^{-1} = \I - \tfrac
%     1\epsilon\X_S^\top(\I+\tfrac 1\epsilon\X_S\X_S^\top)^{-1}\X_S = \I -
%     \X_S^\top(\epsilon\,\I+\X_S\X_S^\top)^{-1}\X_S.
%   \end{align*}
%   Taking the limit $\epsilon\rightarrow 0$ we get the result.
% \end{proofof}

\section{Backup}


\subsubsection{Concentration}

\begin{lemma} With $r$ as the stable rank and $\epsilon \in (0,1)$, for $ \frac 8{\epsilon^2} \ln \frac1{\epsilon} \leq k \leq  \frac {\epsilon (1-\epsilon)}{2(1+\epsilon)}  r $, there exists an algorithm to select a set of rows $S$ with $|S| \leq k$, so that $L(S)  \leq \frac{(1+\epsilon)}{(1- \epsilon) } E^*_k$. 
\end{lemma}
It suffices to show that $\E\big[L(S)\mid |S|\leq k\big]  \leq \frac{(1+\epsilon)}{(1- \epsilon) } E^*_k$ if $S$ is selected using DPP sampling. 

For concentration, we would need $k_\alpha \leq k$, which requires using similar arguments as above:

\begin{align*}
	& \frac r {r-k}\,\frac1{1+\epsilon} \leq 1 - \delta \iff \frac r {r-k} \leq (1-\delta) (1+\epsilon)   \iff 1 - \frac k{r} \geq \frac1{(1-\delta) (1+\epsilon)} \\
	& \iff \frac k{r} \leq \frac{\epsilon(1-\delta)- \delta}{(1+\epsilon)(1-\delta)}.
\end{align*}

For $\delta \leq \epsilon/2$, we have $\frac \delta{1-\delta} \leq \frac {\epsilon/2}{1- \epsilon/2}$. To satisfy the above condition on $\frac k{r}$,

\begin{align*}
	&	\frac k{r} \leq \frac{\epsilon(1-\delta)- \delta}{(1+\epsilon)(1-\delta)} \iff 	\frac  k{r} \leq \frac{\epsilon}{1+\epsilon} - \frac{\delta}{(1+\epsilon)(1-\delta)} \\
	&\impliedby 	\frac  k{r} \leq \frac{\epsilon}{1+\epsilon} - \frac{\epsilon/2}{(1+\epsilon)(1-\epsilon/2)} \iff 	\frac k{r} \leq \frac{\epsilon}{1+\epsilon} \left( 1- \frac{1/2}{1-\epsilon/2}\right)  \\
	&\iff \frac k{r} \leq \frac{\epsilon}{1+\epsilon} \left(  \frac{1/2 - \epsilon/2}{1-\epsilon/2}\right) \impliedby \frac kr \leq \frac {\epsilon (1-\epsilon)}{2(1+\epsilon)} 
\end{align*}	



For concentration, 

\begin{align*}
	& \E\big[L(S)\mid |S|\leq k\big] \leq \frac{\E[L(S)]}{\Pr(|S|\leq k)} \leq \frac {(1+\epsilon)E^*_k}{(1-\ee^{-\delta^2k/2}) } \leq \frac{(1+\epsilon)}{(1- \epsilon) } E^*_k,  
\end{align*}

for $\ee^{-\delta^2k/2} \leq \epsilon $, which gives $k \geq \frac 2{\delta^2} \ln \frac1{\epsilon} $ which is simplified to  $k \geq \frac 8{\epsilon^2} \ln \frac1{\epsilon} $ for $\delta \leq \epsilon/2$.
\bibliography{pap}

\end{document}

 