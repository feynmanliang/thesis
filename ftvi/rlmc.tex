\documentclass{scrartcl}
\usepackage[utf8]{inputenc}
\usepackage[english]{babel}

\usepackage{amsmath}
\usepackage{amsthm}
\usepackage{amssymb}
\usepackage[bold]{hhtensor}

\usepackage{natbib}
\bibliographystyle{apalike}

\usepackage{hyperref}
\usepackage{cleveref}

\usepackage{todonotes}

\DeclarePairedDelimiterX{\infdivx}[2]{(}{)}{%
  #1\;\delimsize\|\;#2%
}
\newcommand{\infdiv}{D\infdivx}

\DeclareMathOperator*{\cov}{cov}

\newcommand{\dd}{\mathrm{d}}

\newcommand{\cE}{\mathcal{E}}
\newcommand{\cL}{\mathcal{L}}
\newcommand{\cO}{\mathcal{O}}
\newcommand{\cP}{\mathcal{P}}
\newcommand{\cQ}{\mathcal{Q}}
\newcommand{\cS}{\mathcal{S}}

\newcommand{\eps}{\varepsilon}
\newcommand{\pd}{\partial}

\newcommand{\dist}{\mathrm{dist}}
\newcommand{\simiid}{\overset{\text{iid}}{\sim}}

\newcommand{\vb}{\vec{b}}
\newcommand{\vv}{\vec{v}}
\newcommand{\vx}{\vec{x}}
\newcommand{\vy}{\vec{y}}
\newcommand{\veta}{\vec{\eta}}

\newcommand{\mI}{\matr{I}}
\newcommand{\mM}{\matr{M}}

\newcommand{\RR}{\mathbb{R}}
\newcommand{\EE}{\mathbb{E}}
\newcommand{\PP}{\mathbb{P}}

\newcommand{\cN}{\mathcal{N}}

\newcommand{\feynman}[1]{{\color{blue}{(Feynman: {#1})}}}
\newcommand{\liam}[1]{{\color{green}{(Liam: {#1})}}}
\newcommand{\michael}[1]{{\color{purple}{(Michael: {#1})}}}


\title{Riemann Manifold Langevin Monte Carlo}


\begin{document}
\maketitle

XXX.  MWM TEST PUSH.

A number of samplers arise from discretizations of a stochastic differential equation (SDE) of the form
\[
    dX_t = \mu(X_t, t) dt + \sigma(X_t, t) d\eta_t
\]
where $\mu$ is a drift term and $\eta_t$ is a noise process.

To sample a target probability density $\pi(X)$, one can set $\mu(X_t) =
    \nabla \log \pi(\theta)$, $\sigma(X_t, t) \equiv 1$, and take $\eta_t = W_t$
a Wiener process. This corresponds to overdamped Langevin dynamics with potential $\log \pi(X)$,
which has $pi(X)$ as stationary solution. First-order Euler discretization of the dynamics
at scale $\eps$ yields
\[
    X_{t+1} \sim q(\cdot\mid X_t) = \cN\left(
    X_t + \frac{\eps^2}{2} \nabla \log \pi(X_t),
    \eps^2 \mI_k
    \right)
\]
When used in series with a Metropolis-Hastings accept/reject to correct for discretization error,
the result is known as the Metropolis Adjusted Langevin Algorithm (MALA, \cite{robert1996exponential}).
To accommodate  different variances between dimensions, \cite{roberts2002langevin} proposes using a
preconditioning ``mass'' matrix
\[
    q(\cdot\mid X_t) = \cN\left(
    X_t + \frac{\eps^2}{2} \nabla \log \pi(X_t),
    \eps^2 \mM
    \right)
\]
To choose $M$, \cite{roberts2002langevin} propose Riemannian-manifold
Langevin Monte Carlo (RLMC) via a Langevin diffusion on a Riemannian manifold
with metric tensor $\mM$ the of Fisher information matrix.
There are a number of choices:
\begin{description}
    \item[Expected Fisher Information] $\cov \log \nabla^2 p(\vy \mid \vx) = -E_{\vy \mid \vx} \nabla^2 \log p(\vy \mid \vx)$, which has the disadvantage that integration over the posterior is required
    \item[Empirical Fisher Information] $\widehat{\cov} \nabla^2 \log p(\vy \mid \vx)$, which only uses observed data
    \item[Observed Fisher Information] $\nabla^2 \log p(\vy \mid \vx) \mid_{\vx = \vx^{ML}}$, which requires only a single Hessian evaluation, is independent of the current particle position $\vx$, and is ensured to be positive semi-definite (hence PD after regularization).
\end{description}
Additionally, \cite{roberts2002langevin} suggest capturing prior informativeness in the metric tensor
by considering the Fisher information of the joint rather than the posterior, i.e.
\[
    -E_{\vy \mid \vx} \nabla^2 \log p(\vy, \vx)
    = \underbrace{-E_{\vy \mid \vx} \nabla^2 \log p(\vy \mid \vx)}_{\text{expected Fisher information}} - \underbrace{\nabla^2 \log p(\vx)}_{\text{Hessian of prior}}
\]

Newtonian Monte Carlo (NMC,\cite{arora2020newtonian}) is the default
inference method in a recently released open-source probabilistic programming
language (PPL) called \texttt{beanmachine} \citep{tehranibean}. Given
a 2-parameter proposal density $f(\cdot; \alpha, \beta)$, NMC performs a second-order approximation
by solving for $\alpha_t$ and $\beta_t$ in
\begin{alignat*}{1}
    \nabla \log \pi(X_t)   & = \nabla \log f(X_t; \alpha, \beta)   \\
    \nabla^2 \log \pi(X_t) & = \nabla^2 \log f(X_t; \alpha, \beta)
\end{alignat*}
and then setting $q(\cdot \mid X_t) = f(\cdot; \alpha_t, \beta_t)$. In the case of a Gaussian PDF $f$
with mean $\alpha$ and variance $\beta$, this becomes
\[
    q(\cdot \mid X_t)
    = \cN(X_t - (\nabla^2 \log \pi(X_t))^{-1} \nabla \log \pi(X_t), -(\nabla^2 \log \pi(X_t))^{-1})
\]
To relate this to RLMC, take $\pi(x) = p(x, y) \propto p(x \mid y)$ a
posterior distribution given (fixed) i.i.d. observations $y$. Then
\begin{alignat*}{1}
    - \nabla^2 \log \pi(x)
     & = - \nabla^2 \log p(x, y)                 \\
     & = - \sum_i^n \nabla^2 \log p(x, y_i)      \\
     & = -n E_{\hat{F}_n} \nabla^2 \log p(x, y)  \\
     & = -n \widehat{\cov} \nabla^2 \log p(x, y)
\end{alignat*}
from which we see that (after rescaling by $n$) NMC corresponds RLMC using the empirical
rather than expected Fisher information.

\todo{Heavy tailed $f$s? Can we show this discretizes a Levy-Flight vs Brownian Motion diffusion?}

There exists a major discrepancy between NMC as formualted and RLMC. From \cite{roberts2002langevin},
a diffusion which has invariant measure $\pi(X)$ on $\RR^d$ is given by the SDE
\begin{alignat}{1}
    dX_t & = \frac{1}{2} \mM(X_t)^{-1} (\nabla \log \pi(X)) dt + d\tilde{\vb}(t)
\end{alignat}
with the Brownian motion on the Riemannian manifold has as $i$th component
\begin{alignat}{1}
    d\tilde{\vb}_i(t)
     & = \underbrace{\lvert \mM(X_t) \rvert^{-1/2} \sum_{j=1}^d \frac{\pd}{\pd X_j} ( \mM(X_t)_{ij} \lvert \mM(X_t) \rvert^{1/2} )}_{A(X_t)} dt
    + (\mM(X_t)^{-1/2} d\vb(t))_i
\end{alignat}
where $\vb$ is standard Brownian motion.
Note that if $A(X_t) = 0$ (e.g. in the case $\mM$ is the observed Fisher information, which is constant),
then taking $\mM(X_t) = \nabla^2 \log \pi(X_t)$ and a first-order Euler discretization
at scale $1$ recovers NMC.

\begin{itemize}
    \item What effect does the missing drift term $A(X_t)$ have? Can we increase acceptance rates by including it?
    \item Is a scale of $1$ appropriate? Given that NMC uses $n$ times the empirical Fisher information, is there a ``better'' scaling involving $n$?
\end{itemize}

A diffusion which NMC does indeed represent the Euler discretization of is given by
\begin{alignat}{1}
    dX_t =\mM(X_t)^{-1} (\nabla \log \pi(X)) dt + \mM(X_t)^{-1/2} db_t
\end{alignat}

\begin{itemize}
    \item What is an (the?) invariant distribution of this distribution?
    \item Is there a bound on a divergence
          measure between this distribution and the target $\pi(X)$?
    \item Since this coincides with the previous diffusion in the case of
          $\mM$ the observed Fisher information, how does the performance of a MCMC
          sampling algorithm using that compare?
\end{itemize}

\bibliography{refs}

\end{document}
